%
% Copyright � 2014 Peeter Joot.  All Rights Reserved.
% Licenced as described in the file LICENSE under the root directory of this GIT repository.
%
\index{modified nodal analysis!inductor}
\index{modified nodal analysis!capacitor}
\index{modified nodal analysis!time dependence}
\makeproblem{Modeling inductors, capacitors and time dependence.}{multiphysics:problemSet3a:1}{
In this problem you will first extend the circuit simulator that you developed
in the previous problem sets to include capacitors \index{capacitor} and inductors \index{inductor}. Then, you
will use it to simulate a clock-distribution network.
\makesubproblem{}{multiphysics:problemSet3a:1a}
Modify the circuit simulator you developed for the previous
assignments to handle capacitors and inductors. The program should
read a file with the list of: resistors, currents sources, voltage sources,
capacitors, and inductors. The syntax for specifying a capacitor is:
\begin{center}
\textbf{Clabel node1 node2 val}
\end{center}
where label is an arbitrary label, node1 and node2 are integer circuit
node numbers, and val is the capacitance (a floating point number).
The syntax for specifying an inductor is:
\begin{center}
\textbf{Llabel node1 node2 val}
\end{center}
where label is an arbitrary label, node1 and node2 are integer circuit
node numbers, and val is the inductance (a floating point number).
Explain how you handle inductors, and which stamp can be proposed to
include them into the modified nodal analysis equations written in the
form
\begin{equation}\label{eqn:multiphysicsProblemSet3a:10}
\BG \Bx(t) + \BC \dot{\Bx}(t) = \BB \Bu(t),
\end{equation}
where the column vector \( \Bu(t) \) contains all sources.
\makesubproblem{}{multiphysics:problemSet3a:1b}
As test system, we consider the network in \cref{fig:problemSet3a:problemSet3aFig1} which distributes the clock signal to 8 blocks of an integrated circuit. The network is in the form of a binary tree with four levels. Each segment is
a transmission line with characteristics (length, per-unit-length parameters) given in \cref{tab:multiphysicsProblemSet3a:20}. Divide each transmission line into segments
of length \( \Delta z = 0.05 \text{mm} \), and model each segment with an RLC circuit
as the one shown in the figure.
\imageFigure{../figures/ece1254-multiphysics/problemSet3aFig1}{Network for the distribution of the clock to 8 IC blocks.}{fig:problemSet3a:problemSet3aFig1}{0.3}
\captionedTable{Characteristics of the clock tree network. Note that the resistance, inductance and capacitance values are per-unit-length (p.u.l.)}{tab:multiphysicsProblemSet3a:20}{
\begin{tabular}{|l||l|l|l|l|l|}
\hline
Level  & Len.      & Res. p.u.l    & Ind. p.u.l. & Cap. p.u.l. \\
       & \( [mm] \)  & \( R [\Omega/cm] \) & \( L [nH/cm] \)   & \( C [pF/cm] \) \\ \hline \hline
1      & 6           & 25                  & 5.00              & 2.00            \\ \hline
2      & 4           & 35.7                & 7.14              & 1.40            \\ \hline
3      & 3           & 51.0                & 10.2              & 0.98            \\ \hline
4      & 2           & 51.0                & 10.2              & 0.98            \\ \hline
\end{tabular}
}
Model the clock source with a periodic
voltage source with the following characteristics: amplitude 1 V, rise/fall
time 100 ps, period 2 ns, duty cycle 50\%, initial delay 100 ps. The clock
source voltage is depicted in \cref{fig:problemSet3a:problemSet3aFig2}.
\imageFigure{../figures/ece1254-multiphysics/problemSet3aFig2}{Clock signal.}{fig:problemSet3a:problemSet3aFig2}{0.3}
Model each chip block with a 5 k \( \Omega \) resistor in parallel with a 5 f F capacitor. Write a Matlab function that generates a spice-compatible netlist
of the clock distribution network. Report in a table the values of the
resistors, inductors and capacitors that you used in each section.
\makesubproblem{}{multiphysics:problemSet3a:1c}
Simulate the circuit with backward Euler (BE). Use a constant
time-step. Plot the voltage generated by the clock source and the voltage
received by one of the blocks for \( t \in [0,5] \) ns. Suggestion: since the clock
starts a few timesteps after \( t = 0 \), you can assume zero initial conditions
for your simulation.
\makesubproblem{}{multiphysics:problemSet3a:1d}
Explain how did you choose the timestep to be used in the simulation.
\makesubproblem{}{multiphysics:problemSet3a:1e}
Implement and compare two methods for solving differential equations: backward Euler (BE), and trapezoidal rule (TR). Simulate the
circuit with the two methods for different timestep values, ranging from
a coarse timestep to a fine time step. Report in a table the error and
the CPU time for each method. Since we don't have an exact solution
for this system, use as reference solution the output voltage computed
with TR with a very fine time step. In the table, report as error the
maximum absolute error between the computed output voltage and the
reference one.
\makesubproblem{}{multiphysics:problemSet3a:1f}
Plot the error versus the time step on a log-log scale for the two
methods, and comment the obtained results.
} % makeproblem
%
\makeanswer{multiphysics:problemSet3a:1}{
\withproblemsetsParagraph{
\makeSubAnswer{}{multiphysics:problemSet3a:1a}
%
To understand how to handle inductors \index{modified nodal analysis!inductor} in the MNA equations, consider a few representative examples
%
\index{modified nodal analysis!RLC circuit}
\paragraph{Example: Simple RLC circuit.}
%{example:multiphysicsProblemSet3a:30}{
First consider an RLC circuit with a current source as sketched in \cref{fig:problemSet3a:problemSet3aFig3}.
%
%\imageFigure{../figures/ece1254-multiphysics/problemSet3aFig3}{RLC circuit with current source}{fig:problemSet3a:problemSet3aFig3}{0.2}
\imageFigure{../figures/ece1254-multiphysics/RLC-circuit-with-current-source.pdf}{RLC circuit with current source.}{fig:problemSet3a:problemSet3aFig3}{0.2}
%
With \( Z = 1/R \), the KCLs plus the inductor voltage relation, are
\begin{enumerate}
\item \( -i_{\textrm{s}} + Z V_1 + C \frac{d(V_1 - V_2)}{dt} = 0 \)
\item \( C \frac{d(V_2 - V_1)}{dt} + i_{\textrm{L}} = 0 \)
\item \( -(V_2 - 0) + L \frac{di_{\textrm{L}}}{dt} = 0 \)
\end{enumerate}
%
This can be put into matrix form as
%
\begin{equation}\label{eqn:multiphysicsProblemSet3a:30}
\begin{bmatrix}
Z & 0 & 0 \\
0 & 0 & 1 \\
0 & -1 & 0 \\
\end{bmatrix}
\begin{bmatrix}
V_1 \\
V_2 \\
i_{\textrm{L}}
\end{bmatrix}
+
\begin{bmatrix}
C & -C & 0 \\
-C & C & 0 \\
0 & 0 & L \\
\end{bmatrix}
%\frac{d}{dt}
{\begin{bmatrix}
V_1 \\
V_2 \\
i_{\textrm{L}}
\end{bmatrix}}'
=
\begin{bmatrix}
i_{\textrm{s}} \\
0 \\
0
\end{bmatrix}.
\end{equation}
%}
%
\index{modified nodal analysis!voltage source}
\paragraph{Example: RLC circuit with voltage source.}
%{example:multiphysicsProblemSet3a:50}{
Now consider the more complex circuit of \cref{fig:problemSet3a:problemSet3aFig4}, with an inductor between two not ground nodes, as well as a voltage source, which will also require the MNA matrix to be augmented.  The equations, starting with the KCLs by node number, are
%
\begin{enumerate}
\item \( -i_{s_1} + Z (V_1 - V_2) = 0 \)
\item \( Z (V_2 - V_1) + C_1 \frac{d(V_2 - 0)}{dt} + C_2 \frac{d(V_2 - V_3)}{dt} = 0 \)
\item \( C_2 \frac{d(V_3 - V_2)}{dt} + i_{\textrm{L}_{3,4}} - i_{\textrm{L}_{0,3}} = 0 \)
\item \( -i_{\textrm{L}_{3,4}} - i_{s_2} = 0 \)
\item \( V_1 = V_{s_1} \)
\item \( -(V_3 - V_4) + L_2 \frac{d}{dt} i_{\textrm{L}_{3,4}} = 0 \)
\item \( -(0 - V_3) + L_1 \frac{d}{dt} i_{\textrm{L}_{0,3}} = 0 \)
\end{enumerate}
%
%\imageFigure{../figures/ece1254-multiphysics/problemSet3aFig4}{More complex RLC circuit}{fig:problemSet3a:problemSet3aFig4}{0.2}
\imageFigure{../figures/ece1254-multiphysics/more-complex-RLC-circuit.pdf}{More complex RLC circuit.}{fig:problemSet3a:problemSet3aFig4}{0.2}
%
In matrix form, this is
\begin{equation}\label{eqn:multiphysicsProblemSet3a:50}
G V + C V' = I,
\end{equation}
where
\begin{equation}\label{eqn:multiphysicsProblemSet3a:50b}
V =
{\begin{bmatrix}
V_1 &
V_2 &
V_3 &
V_4 &
i_{s_1} &
i_{\textrm{L}_{3,4}} &
i_{\textrm{L}_{0,3}} 
\end{bmatrix}}^\T,
\end{equation}
\begin{equation}\label{eqn:multiphysicsProblemSet3a:50d}
I =
{\begin{bmatrix}
0 &
0 &
0 &
i_{s_2} &
V_{s_1} &
0 &
0 &
\end{bmatrix}}^\T,
\end{equation}
\begin{equation}\label{eqn:multiphysicsProblemSet3a:50a}
Z = 
\begin{bmatrix}
Z & -Z & 0 & 0 & -1 & 0 & 0 \\
-Z & Z & 0 & 0 & 0 & 0 & 0 \\
0 & 0 & 0 & 0 & 0 & \mytikzmark{left}{1}1 & -1 \\
0 & 0 & 0 & 0 & 0 & -1\mytikzmark{right}{1} & 0 \\
1 & 0 & 0 & 0 & 0 & 0 & 0 \\
0 & 0 & \mytikzmark{left}{2}-1 & 1\mytikzmark{right}{2} & 0 & 0 & 0 \\
0 & 0 & 1 & 0 & 0 & 0 & 0 \\
\end{bmatrix},
\end{equation}
\begin{equation}\label{eqn:multiphysicsProblemSet3a:50c}
C = 
\begin{bmatrix}
0 & 0 & 0 & 0 & 0 & 0 & 0 \\
0 & C_1 + C_2 & -C_2 & 0 & 0 & 0 & 0 \\
0 & -C_2 & C_2 & 0 & 0 & 0 & 0 \\
0 & 0 & 0 & 0 & 0 & 0 & 0 \\
0 & 0 & 0 & 0 & 0 & 0 & 0 \\
0 & 0 & 0 & 0 & 0 & \mytikzmark{left}{3}L_2\mytikzmark{right}{3} & 0 \\
0 & 0 & 0 & 0 & 0 & 0 & L_1 \\
\end{bmatrix},
\end{equation}
%
\DrawMyBox[thick, Maroon, rounded corners]{1}
\DrawMyBox[thick, Maroon, rounded corners]{2}
\DrawMyBox[thick, Maroon, rounded corners]{3}
%}
%
\paragraph{Example: Inductor interior node.}
%{example:multiphysicsProblemSet3a:91}{
Finally consider a simple example with a single inductor in an interior node as sketched in \cref{fig:ps3aExample3:ps3aExample3Fig1}.
For this example a labeling convention compatible with the implementation to be will be chosen.  This is important to ensure the signs are correct.
%
%ps3aExample3Fig1
\imageFigure{../figures/ece1254-multiphysics/final-RLC-example-circuit.pdf}{A final RLC example circuit.}{fig:ps3aExample3:ps3aExample3Fig1}{0.2}
%
%
The equations are
%
\begin{enumerate}
\item \( -i_{\textrm{s}} + Z (V_1 - 0) -i_{\textrm{L}_{2,1}} = 0 \)
\item \( i_{\textrm{L}_{2,1}} + C \frac{d(V_2 - 0)}{dt} = 0 \)
\item \( -(V_2 - V_1) + L \frac{d}{dt} i_{\textrm{L}_{2,1}} = 0 \),
\end{enumerate}
%
which have the matrix form
%
\begin{equation}\label{eqn:multiphysicsL17:111}
\begin{bmatrix}
Z & 0 & \mytikzmark{left}{1}{-1} \\
0 & 0 & 1 \mytikzmark{right}{1} \\
\mytikzmark{left}{2} 1
& -1 \mytikzmark{right}{2} & 0 \\
\end{bmatrix}
\begin{bmatrix}
V_1 \\
V_2 \\
i_{\textrm{L}_{2,1}}
\end{bmatrix}
+
\begin{bmatrix}
0 & 0 & 0 \\
0 & C & 0 \\
0 & 0 & \mytikzmark{left}{3}L\mytikzmark{right}{3} \\
\end{bmatrix}
{
\begin{bmatrix}
V_1 \\
V_2 \\
i_{\textrm{L}_{2,1}}
\end{bmatrix}
}'
=
\begin{bmatrix}
i_{\textrm{s}} \\
0 \\
0
\end{bmatrix}
\end{equation}
%
\DrawMyBox[thick, Maroon, rounded corners]{1}
\DrawMyBox[thick, Maroon, rounded corners]{2}
\DrawMyBox[thick, Maroon, rounded corners]{3}
%}
%
From the above examples, the inductor stamp structure can be observed.  For an inductance \( L \) between nodes \( n_i, n_j \) and a current \( i_{\textrm{L}_{n_i, n_j}} \) directed from \( n_i \) to \( n_j \) that \( \BG \) portion of that stamp is
%
\begin{equation}\label{eqn:multiphysicsProblemSet3a:70}
\kbordermatrix{
                 & n_i & n_j  & i_{\textrm{L}_{n_i, n_j}} \\
n_i              &     &      & 1                \\
n_j              &     &      & -1               \\
i_{\textrm{L}_{n_i, n_j}} & -1  & 1    &                  \\
},
\end{equation}
%
and the \( \BC \) portion of that stamp with just
%
\begin{equation}\label{eqn:multiphysicsProblemSet3a:71}
\kbordermatrix{
                 & i_{\textrm{L}_{n_i, n_j}} \\
i_{\textrm{L}_{n_i, n_j}} & L                \\
}.
\end{equation}
%
The trickiest aspect of application of this stamp is likely determining the right rows and columns to allocate for each inductor current, leaving sufficient space for the voltage source currents rows and columns, as well as space for any voltage controlled voltage gain stamps.
%
See \nbcite{ps3a:NodalAnalysis.m}{ps3a/NodalAnalysis.m} for the Matlab code to generate MNA equations also including capacitor and inductor support.
%
\makeSubAnswer{}{multiphysics:problemSet3a:1b}
%
\captionedTable{RLC values per segment.}{tab:multiphysicsProblemSet3a:131}{
\begin{tabular}{|l||l|l|l|l|}
\hline
Level  & n        & R (\( \Omega \))  & Inductance (nH) & Capacitance (pF) \\ \hline
\hline
1      & 120.0000 & 0.1250            & 0.0250          & 0.0100 \\ \hline
2      & 80.0000  & 0.1785            & 0.0357          & 0.0070 \\ \hline
3      & 60.0000  & 0.2550            & 0.0510          & 0.0049 \\ \hline
4      & 40.0000  & 0.2550            & 0.0510          & 0.0049 \\ \hline
\end{tabular}
}
%
Since there is only one source in the system, the clock that feeds the transmission line network, the source vector \( \BB \Bu(t) \) has just one non-zero element.  This allows the system equations to be expressed in a simple fashion that requires no support of time dependent functions in the \matlabFuncPath{NodalAnalysis()}{ps3a:NodalAnalysis.m} code.  If the clock amplitude is \( v_{\textrm{clk}}(t) \), then the system equations have the form
%
\begin{equation}\label{eqn:multiphysicsProblemSet3a:151}
\BG \Bx(t) + \BC \dot{\Bx}(t) = \Bb v_{\textrm{clk}}(t),
\end{equation}
%
and \( \BG, \BC, \Bb \) are the MNA matrices for a constant unit voltage at the source (clock) node.
%
Code to implement such a netlist for this network, including the terminating chip loads, can be found in \matlabFuncPath{generateNetlistSegmentForLevel()}{ps3a:genNlSegmentForLevel.m},
\matlabFuncPath{generateNetlistSegment()}{ps3a:genNlSegment.m}, and
\matlabFuncPath{generateNetlist()}{ps3a:genNl.m}, the last of which takes the filename for the netlist output.
%
The function \matlabFuncPath{vclock(t)}{ps3a:vclock.m} implements the clock signal \( v_{\textrm{clk}}(t) \), for \( t \) in nanoseconds.
%
\makeSubAnswer{}{multiphysics:problemSet3a:1c}
%
The system equations for BE are
%
\begin{equation}\label{eqn:multiphysicsProblemSet3a:171}
\BG \Bx_n + \BC \frac{ \Bx_n - \Bx_{n-1} }{\Delta t} = \Bb v_{\textrm{clk}}(t_n),
\end{equation}
%
or
\begin{equation}\label{eqn:multiphysicsProblemSet3a:191}
\lr{ \BG + \inv{\Delta t} \BC } \Bx_n = \BC \frac{ \Bx_{n-1} }{\Delta t} + \Bb v_{\textrm{clk}}(t_n).
\end{equation}
%
This is the system to solve at each time step for \( \Bx_n \).  A one time LU factorization of \( \BG + \BC/\Delta t \) was used to hoist most of the inversion cost out of the time stepping loop.  The load voltage and the reference clock source is plotted in \cref{fig:ps3aBE:ps3aBEFig1}, calculated using %%\matlabFuncPath{calculateSolutionForTimestep(0.05, 1)}{ps3a:calculateSolutionForTimestep.m}
\matlabFuncPath{driver(0, 1)}{ps3a:driver.m}, using a \( 5 \) ps timestep.
%
\mathImageFigure{../figures/ece1254-multiphysics/ps3aBEFig1.pdf}{Load (chip) voltage calculated with BE.}{fig:ps3aBE:ps3aBEFig1}{0.4}{ps3a:driver.m}
%
\makeSubAnswer{}{multiphysics:problemSet3a:1d}
%
Iterative refinement of the timestep was used to determine a reasonable seeming stopping point.  With a rise and fall time of 100 ps, something smaller than that was clearly required, and 50 ps was tried first to get a rough idea of the output signal, checking against expectations for correctness.  A plot at this timestep clearly shows artifacts of the discretization, which is removed at a timestep of 10 ps.  Further reduction of the timestep to 5 ps sharpens the curve slightly, and can still be completed with a reasonable computation time.  A final reduction of the timestep to 1 ps adds very little additional correction,
all at the leading and trailing edges of the signal, but adds significantly to the time required for the calculation.  Plots with these range of timesteps are shown in \cref{fig:ps3aBE:ps3aBEFig2}.
%
\mathImageFigure{../figures/ece1254-multiphysics/ps3aBEFig2.pdf}{Chip voltage at a variety of timesteps.}{fig:ps3aBE:ps3aBEFig2}{0.5}{ps3a:driver.m}
%
\makeSubAnswer{}{multiphysics:problemSet3a:1e}
%
With
\begin{dmath}\label{eqn:multiphysicsProblemSet3a:231}
\begin{aligned}
\BA &= \frac{2}{\Delta t} \BC + \BG  \\
\BB &= \frac{2}{\Delta t} \BC - \BG  \\
\Bc &= \Bb \lr{v_{\textrm{clk}}(t_{n-1}) + v_{\textrm{clk}}(t_{n}) },
\end{aligned}
\end{dmath}
%
the TR system equations take the form
\begin{dmath}\label{eqn:multiphysicsProblemSet3a:251}
\BA \Bx' = \BB \Bx + \Bc.
\end{dmath}
%
%\begin{dmath}\label{eqn:multiphysicsProblemSet3a:211}
%\lr{ \frac{2}{\Delta t} \BC + \BG } \Bx_n
%=
%-
%\lr{ \frac{2}{\Delta t} \BC - \BF } \Bx_{n_1}
%+ \Bb
%\lr{v_{\textrm{clk}}(t_{n-1})
%+
%v_{\textrm{clk}}(t_{n}) }.
%\end{dmath}
%
As with BE a \( \BL \BU = \BP \BA \) factorization can be computed once outside of the loop.  In \cref{fig:ps3aTR:ps3aTRFig4} are plots of such TR solutions with a range of timesteps, calculated using \matlabFuncPath{calculateSolutionForTimestep(..., 0)}{ps3a:calculateSolutionForTimestep.m}.
%
\mathImageFigure{../figures/ece1254-multiphysics/ps3aTRFig4.pdf}{TR solutions.}{fig:ps3aTR:ps3aTRFig4}{0.5}{ps3a:driver.m}
%
To facilitate calculation of the error at different timesteps a linear interpolation function \matlabFuncPath{interpolate()}{ps3a:interpolate.m} was used to construct a vector for each of the measurements with the timescale of the reference 1 ps TR solution.
%
Plots of the errors for each of these, relative to the reference solution are given in \cref{fig:ps3aErrorBE:ps3aErrorBEFig5}, and \cref{fig:ps3aErrorTR:ps3aErrorTRFig6}, for BE and TR respectively.
%
\imageFigure{../figures/ece1254-multiphysics/ps3aErrorBEFig5.pdf}{BE error relative to 1 ps TR reference solution.}{fig:ps3aErrorBE:ps3aErrorBEFig5}{0.3}
\imageFigure{../figures/ece1254-multiphysics/ps3aErrorTRFig6.pdf}{TR error relative to 1 ps TR reference solution.}{fig:ps3aErrorTR:ps3aErrorTRFig6}{0.3}
%
At a very course granularity timestep the largest TR error is larger than with BE at the same timestep, however, the TR error is smaller at any given timestep for finer granularity timesteps.  This variation of maximum error with timestep for these methods is tabulated in \cref{tab:multiphysicsProblemSet3a:271}, where the error values are in volts and the cpu times in seconds.  As expected, the cpu cost measurements at fixed timestep for BE and TR do not have statistically significant differences.
%
% raw data:
%    BE, 50        0.2772       1.96 \pm 0.02
%    TR, 50        0.4567       1.95 \pm 0.01
%    BE, 10        0.1114       9.77 \pm 0.07
%    TR, 10        0.0846       9.74 \pm 0.03
%    BE, 5         0.0688      19.45 \pm 0.06
%    TR, 5         0.0409      19.59 \pm 0.07
%    BE, 1         0.0261      96.9  \pm 0.2
\captionedTable{Error and CPU usage for BE and TR methods at various timesteps.}{tab:multiphysicsProblemSet3a:271}{
\begin{tabular}{|l||l|l|l|l|}
\hline
timestep (ps) & error (BE) & error (TR) & cpu (BE) & cpu (TR) \\
\hline
\hline
50            & 0.2772 & 0.4567 & \( 1.96 \pm 0.02 \) & \( 1.95 \pm 0.01 \) \\ \hline
10            & 0.1114 & 0.0846 & \( 9.77 \pm 0.07 \) & \( 9.74 \pm 0.03 \) \\ \hline
5             & 0.0688 & 0.0409 & \( 19.45 \pm 0.06 \) & \( 19.59 \pm 0.07 \) \\ \hline
1             & 0.0261 &        & \( 96.9  \pm 0.2 \) & \\ \hline
\end{tabular}
}
%
The plots and tabulated data above was produced with a call to \matlabFuncPath{driver(1, 0)}{ps3a:driver.m}.
%
\makeSubAnswer{}{multiphysics:problemSet3a:1f}
%
In \cref{fig:ps3aLogLogError:ps3aLogLogErrorFig7} is a log-log plot of timestep vs error.
%
\imageFigure{../figures/ece1254-multiphysics/ps3aLogLogErrorFig7.pdf}{LogLog plot of timestep vs error for BE and TR.}{fig:ps3aLogLogError:ps3aLogLogErrorFig7}{0.3}
%
Close to linear relationships in the log-log plots indicate that the BE and TR 	 timestep have a power relationship of differing degree, where the slope is related to the degree.
%
Those lines are found to be
%
\begin{subequations}
\begin{dmath}\label{eqn:multiphysicsProblemSet3a:471}
\ln(\textrm{err}_{\textrm{BE}}) = -3.6366 + 0.6067 \ln( {\Delta t}_{\textrm{BE}} )
\end{dmath}
\begin{dmath}\label{eqn:multiphysicsProblemSet3a:491}
\ln(\textrm{err}_{\textrm{TR}}) = -4.8835 + 1.0480 \ln( {\Delta t}_{\textrm{TR}} )
\end{dmath}
\end{subequations}
%
With \( \Delta t = 5 \text{ns}/(N -1) \sim 1/N \) this indicates that
%
\begin{subequations}
\begin{dmath}\label{eqn:multiphysicsProblemSet3a:671}
\textrm{err}_{\textrm{BE}} \propto N^{-0.6}
\end{dmath}
\begin{dmath}\label{eqn:multiphysicsProblemSet3a:691}
\textrm{err}_{\textrm{TR}} \propto N^{-1.05}
\end{dmath}
\end{subequations}
%
This set of proportionality relationships is plotted in \cref{fig:ps3aLogLogExponential:ps3aLogLogExponentialFig7}.  While this ignores the exact proportionality constants and scale factors, this plot illustrates the order of the experimentally determined relationship of the number of timestep intervals to the error for BE and TR.
%
\mathImageFigure{../figures/ece1254-multiphysics/ps3aLogLogExponentialFig7.pdf}{Error vs \( N^m \).}{fig:ps3aLogLogExponential:ps3aLogLogExponentialFig7}{0.3}{ps3a:loglogExponential.m}
}
}
