%
% Copyright © 2014 Peeter Joot.  All Rights Reserved.
% Licenced as described in the file LICENSE under the root directory of this GIT repository.
%

\section{Damped Newton's method}
\index{Newton's method!damped}

\Cref{fig:lecture12:lecture12Fig3} illustrates a system that may have troublesome oscillation, depending on the initial guess selection.

\imageFigure{../figures/ece1254-multiphysics/lecture12Fig3}{Oscillatory Newton's iteration.}{fig:lecture12:lecture12Fig3}{0.2}


Large steps can be dangerous, and can be avoided by modifying Newton's method as follows

The algorithm is
\makealgorithm{Damped Newton's method.}{alg:multiphysicsL12DampedNewtons:402}{
\STATE Guess \( \Bx^0, k = 0 \).
\REPEAT
\STATE Compute \( F \) and \( J_F \) at \( \Bx^k \)
\STATE Solve linear system  \( J_F(\Bx^k) \Delta \Bx^k = - F(\Bx^k) \)
\STATE \( \Bx^{k+1} = \Bx^k + \alpha^k \Delta \Bx^k \)
\STATE \( k = k + 1 \)
\UNTIL{converged}
}

This is the standard Newton's method when \( \alpha^k = 1 \).
%ZWe want to pick \( \alpha^k \) so that Zwe minimize

\section{Continuation parameters}
\index{continuation parameter}

Newton's method converges given a close initial guess.  A sequence of problems can be generated where the previous problem generates a good initial guess for the next problem.

An example is a heat conducting bar, with a final heat distribution.  The numeric iteration can be started with \( T = 0 \).  The temperature can be gradually increased until the final desired heat distribution is increased.

To solve a general system of the form

\begin{equation}\label{eqn:multiphysicsL12:340}
F(\Bx) = 0.
\end{equation}

modify this problem by introducing a parameter

\begin{equation}\label{eqn:multiphysicsL12:360}
\tilde{F}(\Bx(\lambda), \lambda) = 0,
\end{equation}

where

\begin{itemize}
\item \( \tilde{F}(\Bx(0), 0) = 0 \) is easy to solve
\item \( \tilde{F}(\Bx(1), 1) = 0 \) is equivalent to \( F(\Bx) = 0 \).
\item (more on slides)
\end{itemize}

The source load stepping algorithm is

\begin{itemize}
\item Solve \(\tilde{F}(\Bx(0), 0) = 0 \), with \( \Bx(\lambda_{\text{prev}} = \Bx(0) \)
\item (more on slides)
\end{itemize}

This can still have problems, for example, when the parameterization is multivalued as in \cref{fig:lecture12:lecture12Fig4}.

\imageFigure{../figures/ece1254-multiphysics/lecture12Fig4}{Multivalued parameterization.}{fig:lecture12:lecture12Fig4}{0.2}

It is possible to adjust \( \lambda \) so that the motion is along the parameterization curve.

%\EndArticle
%\EndNoBibArticle
