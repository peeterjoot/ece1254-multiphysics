%
% Copyright � 2014 Peeter Joot.  All Rights Reserved.
% Licenced as described in the file LICENSE under the root directory of this GIT repository.
%
\index{modified nodal analysis}
\makeproblem{Modified Nodal Analysis.}{multiphysics:problemSet1:1}{
\makesubproblem{
Write a Matlab routine \matlabFunc{[G,b]=NodalAnalysis(filename)} that generates the modified nodal analysis (MNA) equations
\begin{equation}\label{eqn:multiphysicsProblemSet1Problem1:10}
\BG \Bx = \Bb,
\end{equation}
from a text file (netlist) that describes an electrical circuit made of resistors, independent current sources, independent voltage sources, voltage-controlled voltage sources. For the netlist, we use the widely-adopted SPICE syntax. For each resistor, the file will contain a line in the form
\begin{center}
\textbf{Rlabel node1 node2 value}
\end{center}
where ``value'' is the resistance value. Current sources are specified with the line
\begin{center}
\textbf{Ilabel node1 node2 DC value}
\end{center}
and current flows from node1 to node2. Note that DC is just a keyword.
A voltage source connected between the nodes node+ and node- is specified by the line
\begin{center}
\textbf{Vlabel node+ node- DC value}
\end{center}
where node+ and node- identify, respectively, the node where the ``positive'' and ``negative'' terminal is connected to. A voltage-controlled voltage source, connected between the nodes node+ and node-, is specified by the line
\begin{center}
\textbf{Elabel node+ node- nodectrl+ nodectrl- gain}
\end{center}
The controlling voltage is between the nodes nodectrl+ and nodectrl-, and the last argument is the source gain.
}{multiphysics:problemSet1:1a}
\makesubproblem{Explain how did you include the controlled source into the modified nodal analysis formulation. Which general rule can be given to
``stamp'' a voltage-controlled voltage source into MNA?}{multiphysics:problemSet1:1b}
\makesubproblem{Consider the circuit shown in the figure \cref{fig:ps1Orig:ps1OrigFig1}. Write an input file for the netlist parser developed in the previous point, and use it to generate the matrices G and b for the circuit. The operational amplifiers have an input resistance of \(1 M \Omega\)? and a gain of \(10^6\) . Model them with a resistor and a voltage-controlled voltage source. Use the Matlab command \matlabText{\(\backslash\)} to solve the linear system \cref{eqn:multiphysicsProblemSet1Problem1:10} and determine the voltage \(V_\circ\) shown in the figure.}{multiphysics:problemSet1:1c}
\imageFigure{../figures/ece1254-multiphysics/ps1OrigFig1}{Circuit to solve.}{fig:ps1Orig:ps1OrigFig1}{0.4}
\makesubproblem{Implement your own LU factorization routine. Repeat the previous point using your own LU factorization and forward/backward substitution routines to solve the circuit equations. Report the computed \(V_\circ\).}{multiphysics:problemSet1:1d}
} % makeproblem
%
\makeanswer{multiphysics:problemSet1:1}{
\withproblemsetsParagraph{
\makeSubAnswer{}{multiphysics:problemSet1:1a}
%
A netlist parser can be found in
%\par
\nbcite{ps1:NodalAnalysis.m}{ps1/NodalAnalysis.m}
%
Assumptions made in that implementation include
%
\begin{itemize}
\item The ``label'' following the R, I, V, E is numeric.  This appears to be the case in all the example spice circuits of \citep{allaboutcircuitsExampleCircuitsAndNetlists}.
%\href{http://www.allaboutcircuits.com/vol_5/chpt_7/8.html}{http://www.allaboutcircuits.com/vol_5/chpt_7/8.html}
\item .end terminates the netlist parsing.
\item The first line of netlist is a (title) comment unless it starts with R, I, V, E.
\item The netlist file will always include a 0 (ground) node.  No error checking was added to verify that is the case.
\item No gaps in the node numbers are allowed.
\item I vaguely recall that spice files allowed the constants to be specified with k, m, M modifiers.  No support for that was added.
\end{itemize}
%
\makeSubAnswer{}{multiphysics:problemSet1:1b}
%
%
To understand how to incorporate a voltage controlled voltage gain element into the MNA structure, consider the circuit of \cref{fig:circuitWithVoltageGain:circuitWithVoltageGainFig4}.
%
%\imageFigure{../figures/ece1254-multiphysics/circuitWithVoltageGainFig4}{circuit with voltage controlled voltage gain}{fig:circuitWithVoltageGain:circuitWithVoltageGainFig4}{0.3}
\imageFigure{../figures/ece1254-multiphysics/circuit-with-voltage-controlled-gain.pdf}{Circuit with voltage controlled voltage gain.}{fig:circuitWithVoltageGain:circuitWithVoltageGainFig4}{0.3}
%
%
The voltage sources introduce relationships between the node voltages that involve no current terms.
%
\begin{equation}\label{eqn:multiphysicsProblemSet1Problem1:20}
\begin{aligned}
V_2 - V_1 &= g \lr{ V_4 - V_3 } \\
V_3 &= 5 V \\
\end{aligned},
\end{equation}
%
However, since currents from these sources flow through other portions of the circuit, introduction of current unknowns through these elements is required.  Using \( i_{3,0} \), and \( i_{2,1} \) for these respective currents, directed from positive to negative, the KCL equations at each node are
%
\begin{enumerate}
\item
\(
  \lr{ V_1 - 0 } Z_2
- i_{2,1}
+ \lr{ V_1 - V_4 } Z_3  = 0 \)
\item
\(
  \lr{ V_2 - 0 } Z_1
+ \lr{ V_2 - V_4 } Z_4
+ i_{2,1}
+ \lr{ V_2 - V_3 } Z_6
= 0 \)
\item
\(
  i_{3,0}
+ \lr{ V_3 - V_4 } Z_5
+ \lr{ V_3 - V_2 } Z_6
= 0 \).
\item
\(
  \lr{ V_4 - V_1 } Z_3
+ \lr{ V_4 - V_2 } Z_4
+ \lr{ V_4 - V_3 } Z_5
= 0 \)
\end{enumerate}
%
\textbf{NOTE}: This is not the sign convention that we used in class for the currents through the voltage sources, but does match \citep{najm2010circuit}.  This way results in submatrices that are symmetric instead of having inverted signs.  Symmetric submatrices for the terms associated with the voltage sources seems more desirable?
%
This is four equations and six unknowns, but can be supplemented with \cref{eqn:multiphysicsProblemSet1Problem1:20} to form the invertible matrix equation
%
\begin{equation}\label{eqn:multiphysicsProblemSet1Problem1:40}
\begin{bmatrix}
%  \lr{ V_1 - 0 } Z_2 - i_{2,1} + \lr{ V_1 - V_4 } Z_3  = 0
Z_2 + Z_3 & 0 & 0 & - Z_3 & -1 & 0 \\
%  \lr{ V_2 - 0 } Z_1 + \lr{ V_2 - V_4 } Z_4  + i_{2,1} + \lr{ V_2 - V_3 } Z_6  = 0
0 & Z_1 + Z_4 + Z_6 & - Z_6 & -Z_4 & 1 & 0 \\
%  i_{3,0} + \lr{ V_3 - V_4 } Z_5  + \lr{ V_3 - V_2 } Z_6  = 0
0 & -Z_6 & Z_5 + Z_6 & - Z_5 & 0 & 1 \\
%  \lr{ V_4 - V_1 } Z_3 + \lr{ V_4 - V_2 } Z_4  + \lr{ V_4 - V_3 } Z_5  = 0
-Z_3 & -Z_4 & -Z_5 & Z_3 + Z_4 + Z_5 & 0 & 0 \\
%-V_2 + V_1 + g \lr{ V_4 - V_3 } = 0
1 & -1 & -g & g & 0 & 0 \\
%V_3 = 5 V
0 &  0 & 1 &  0 & 0 & 0  \\
\end{bmatrix}
\begin{bmatrix}
V_1 \\
V_2 \\
V_3 \\
V_4 \\
i_{2,1} \\
i_{3,0} \\
\end{bmatrix}
=
\begin{bmatrix}
0 \\
0 \\
0 \\
0 \\
0 \\
5 \\
\end{bmatrix}
\end{equation}
%
If the specification of a voltage controlled voltage gain has the form
%
\begin{center}
\textbf{Ennnnnnn \(n_{+}\) \(n_{-}\) \(nc_{+}\) \(nc_{-}\) \(g\)}
\end{center}
%
then the corresponding stamp has the form
%
\begin{equation}\label{eqn:multiphysicsProblemSet1Problem1:80}
\kbordermatrix{
        & n_{+} & n_{-} & nc_{+} & nc_{-} &  i_{n_{+},n_{-}} \\
n_{+}   & 0     & 0 & 0 & 0    & 1 \\
n_{-}   & 0     & 0 & 0 & 0    & -1 \\
nc_{+}  & 0     & 0 & 0 & 0    &  0 \\
nc_{-}  & 0     & 0 & 0 & 0    & 0 \\
        &    -1 & 1 & g & -g   & 0 \\
},
\end{equation}
%
where a current variable \( i_{n_{+},n_{-}} \) has been introduced for the current between nodes \( n_{+} \) and \( n_{-} \).  Observe that this stamp also incorporates this current variable in the KCL equations for the appropriate nodes.
%
\makeSubAnswer{}{multiphysics:problemSet1:1c}
%
An illustration of an ideal op amp model from \citep{sedra1982microelectronic} is copied in \cref{fig:idealOpAmp:idealOpAmpFig1}.  The model for this problem differs only by having a current and resistance between the input terminals.  Incorporating that model into the circuit gives the equivalent circuit of \cref{fig:ps1circuit:ps1circuitFig3}.
%
\imageFigure{../figures/ece1254-multiphysics/idealOpAmpFig1}{Ideal op amp.}{fig:idealOpAmp:idealOpAmpFig1}{0.3}
%
%\imageFigure{../figures/ece1254-multiphysics/ps1circuitFig3}{circuit with op model}{fig:ps1circuit:ps1circuitFig3}{0.5}
\imageFigure{../figures/ece1254-multiphysics/op-amp-equivalent-circuit.pdf}{Circuit with op model.}{fig:ps1circuit:ps1circuitFig3}{0.5}
%
This corresponding netlist representation can be found in
\par
\href{https://raw.github.com/peeterjoot/matlab/master/ece1254/ps1/testdata/ps1.circuit.netlist}{testdata/ps1.circuit.netlist},
%
%\begin{verbatim}
%R1 1 2 10000
%R2 2 0 1000000
%R3 2 3 40000
%R4 2 4 80000
%R5 3 5 1000000
%R6 5 0 40000
%R7 4 5 20000
%V1 1 0 DC 5
%E1 3 0 0 2 1000000
%E2 4 0 5 3 1000000
%\end{verbatim}
%
The resulting system \( G \Bx = \Bb \) looks somewhat ill conditioned with both very large and very small numbers
%
\begin{equation}\label{eqn:multiphysicsProblemSet1Problem1:100}
\begin{bmatrix}
10^{-4} &-10^{-4}&0&0&0&-1&0&0 \\
-10^{-4}&138.5 \times 10^{-6}&-25 \times 10^{-6}&-12.5 \times 10^{-6}&0&0&0&0 \\
0       &-25 \times 10^{-6}&26 \times 10^{-6}&0&- 10^{-6}&0&1&0 \\
0       &-12.5 \times 10^{-6}&0&62.5 \times 10^{-6}&-50 \times 10^{-6}&0&0&1 \\
0       &0&- 10^{-6}&-50 \times 10^{-6}&76 \times 10^{-6}&0&0&0 \\
1       &0&0&0&0&0&0&0 \\
0       &- 10^6&-1&0&0&0&0&0 \\
0       &0&- 10^6&-1& 10^6&0&0&0 \\
\end{bmatrix}
\begin{bmatrix}
V_1 \\
V_2 \\
V_3 \\
V_4 \\
V_5 \\
i_{1,0} \\
i_{3,0} \\
i_{5,0}
\end{bmatrix}
=
\begin{bmatrix}
0 \\
0 \\
0 \\
0 \\
0 \\
5 \\
0 \\
0
\end{bmatrix}
\end{equation}
%
However, this system yields to both the \matlabText{G \(\backslash\) b} solution and \( L U \) decomposition (even without pivots).  This is likely because the difference in min and max non-zero values is still within the range of the default Matlab double precision arithmetic.
%
%Which can be solved with
%
%\begin{verbatim}
%clear all ; [G, b] = NodalAnalysis( 'ps1.circuit.netlist' ) ; G\(\backslash\)b
%\end{verbatim}
%which has solution
%
The system of \cref{eqn:multiphysicsProblemSet1Problem1:100} yields the solution
%
\begin{equation}\label{eqn:multiphysicsProblemSet1Problem1:120}
\Bx =
\begin{bmatrix}
     5 \\
    11.4285 \times 10^{-6} \\
   -11.4285 \\
   -17.1428 \\
   -11.4285 \\
   499.9989 \times 10^{-6} \\
   285.7135 \times 10^{-6} \\
   499.9990 \times 10^{-6} \\
\end{bmatrix}.
\end{equation}
%
So the voltage difference of interest in this problem is \( V_\circ = V_5 - V_0 = -11.4 V \).
%
\makeSubAnswer{}{multiphysics:problemSet1:1d}
%
The routines for the LU solution and substitutions are found in:
%
\begin{itemize}
	\item \nbcite{ps1:noPivotLU.m}{ps1/noPivotLU.m}
	\item \nbcite{ps1:forwardSubst.m}{ps1/forwardSubst.m}
	\item \nbcite{ps1:backSubst.m}{ps1/backSubst.m}
	\item \nbcite{ps1:findMaxIndexOfColumnMatrix.m}{ps1/findMaxIndexOfColumnMatrix.m}
	\item \nbcite{ps1:verifyUpperTriangular.m}{ps1/verifyUpperTriangular.m}
	\item \nbcite{ps1:ps1d.m}{ps1/ps1d.m}
\end{itemize}
%
That last contains is a driver for these functions, callable with \matlabFuncPath{ps1d()}{ps1:ps1d.m}.  That will produce the netlist, and its solution, both directly and with the \matlabText{\(\backslash\)} operator.
%
An \( L U \) factorization that properly treated pivots was not numerically required for this part of the problem, but was implemented for \ref{multiphysics:problemSet1:2}.
%
%An inefficient pivot \( L U \) factorization routine was implemented in \nbcite{ps1:withPivotLU.m}{ps1/withPivotLU.m}.
%
%\par
%\href{https://raw.github.com/peeterjoot/matlab/master/ece1254/ps1/}{https://raw.github.com/peeterjoot/matlab/master/ece1254/ps1/}
}
} % makeanswer
