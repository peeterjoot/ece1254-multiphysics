%
% Copyright � 2014 Peeter Joot.  All Rights Reserved.
% Licenced as described in the file LICENSE under the root directory of this GIT repository.
%
%\input{../blogpost.tex}
%\renewcommand{\basename}{multiphysicsL3}
%\renewcommand{\dirname}{notes/ece1254/}
%\newcommand{\keywords}{ECE1254H}
%\input{../peeter_prologue_print2.tex}
%
%\beginArtNoToc
%\generatetitle{ECE1254H Modeling of Multiphysics Systems.  Lecture 3: Nodal Analysis.  Taught by Prof.\ Piero Triverio}
%\section{Nodal Analysis}
\index{nodal analysis}
\label{chap:multiphysicsL3}
%
%\section{Disclaimer}
%
%Peeter's lecture notes from class.  These may be incoherent and rough.
%
\section{Nodal Analysis.}
%
Avoiding branch currents can reduce the scope of the computational problem.  Consider the same circuit \cref{fig:lecture3:lecture3Fig1}, this time introducing only node voltages as unknowns
%
%\imageFigure{../figures/ece1254-multiphysics/lecture3Fig1}{Resistive circuit with current sources}{fig:lecture3:lecture3Fig1}{0.3}
\imageFigure{../figures/ece1254-multiphysics/1254-sample-resistive-circuit-node-voltage-unknowns.pdf}{Resistive circuit with current sources.}{fig:lecture3:lecture3Fig1}{0.3}
%
Unknowns: node voltages: \(V_1, V_2, \cdots V_4\)
%
Equations are \textAndIndex{KCL} at each node except \(0\).
%
\begin{enumerate}
\item
      \(
      \frac{V_1 - 0}{R_{\textrm{A}}} +
      \frac{V_1 - V_2}{R_{\textrm{B}}} + i_{\textrm{S},\textrm{A}} = 0
      \)
\item
      \(
      \frac{V_2 - 0}{R_{\textrm{E}}} +
      \frac{V_2 - V_1}{R_{\textrm{B}}} + i_{\textrm{S},\textrm{B}} + i_{\textrm{S},\textrm{C}} = 0
      \)
\item
      \(
      \frac{V_3 - V_4}{R_{\textrm{C}}} - i_{\textrm{S},\textrm{C}} = 0
      \)
\item
      \(
      \frac{V_4 - 0}{R_{\textrm{D}}}
      +\frac{V_4 - V_3}{R_{\textrm{C}}}
      - i_{\textrm{S},\textrm{A}} - i_{\textrm{S},\textrm{B}} = 0
      \)
\end{enumerate}
%
In matrix form this is
%
\begin{equation}\label{eqn:multiphysicsL3:20}
\begin{bmatrix}
   \inv{R_{\textrm{A}}} + \inv{R_{\textrm{B}}} & - \inv{R_{\textrm{B}}} & 0 & 0 \\
   -\inv{R_{\textrm{B}}} & \inv{R_{\textrm{B}}} + \inv{R_{\textrm{E}}} & 0 & 0 \\
   0 & 0 & \inv{R_{\textrm{C}}} & -\inv{R_{\textrm{C}}} \\
   0 & 0 & -\inv{R_{\textrm{C}}} & \inv{R_{\textrm{C}}} + \inv{R_{\textrm{D}}}
\end{bmatrix}
\begin{bmatrix}
   V_1 \\
   V_2 \\
   V_3 \\
   V_4 \\
\end{bmatrix}
=
\begin{bmatrix}
   -i_{\textrm{S},\textrm{A}} \\
-i_{\textrm{S},\textrm{B}} - i_{\textrm{S},\textrm{C}} \\
   i_{\textrm{S},\textrm{C}} \\
   i_{\textrm{S},\textrm{A}} + i_{\textrm{S},\textrm{B}}
\end{bmatrix}
\end{equation}
%
Introducing the \textAndIndex{nodal matrix} \(G\), this is written as
%
\begin{equation}\label{eqn:multiphysicsL3:40}
   G \BV_N = \BI_{\textrm{S}}
\end{equation}
%
There is a recurring pattern in the nodal matrix, designated the
\textAndIndex{stamp} for the resistor of value \(R\) between nodes \(n_1\) and \(n_2\)
%
% FIXME: could later point back to simple resistor node figure.
%\cref{fig:lecture3:lecture3Fig2}
%\imageFigure{../figures/ece1254-multiphysics/lecture3Fig2}{Resistor stamp matrix.}{fig:lecture3:lecture3Fig2}{0.3}
%
% http://stackoverflow.com/questions/664798/matrices-in-latex-row-and-column-labelling
\begin{equation}\label{eqn:multiphysicsL3:60}
\kbordermatrix{
    & n_1      & n_2 \\
n_1 & \inv{R}  & -\inv{R} \\
n_2 & -\inv{R} & \inv{R}
},
\end{equation}
%
containing a set of rows and columns for each of the node voltages \(n_1, n_2\).
%
Note that some care is required to use this nodal analysis method since the invertible relationship \(i = V/R\) is required.  Short circuits \(V = 0\), and voltage sources such as \(V = 5\) also cannot be handled directly.  The mechanisms to deal with differential terms like inductors will be discussed later.
%
\paragraph{Recap of node branch equations}
%
The node branch equations were
%
\begin{itemize}
   \item KCL: \( A \BI_{\textrm{B}} = \BI_{\textrm{S}}\)
   \item Constitutive: \( \BI_{\textrm{B}} = \alpha A^\T \BV_N\),
   \item Nodal equations: \(
\mathLabelBox
{
      A \alpha A^\T
}
{\(G\)}
      \BV_N
      = \BI_{\textrm{S}} \)
\end{itemize}
%
where \(\BI_{\textrm{B}}\) was the branch currents, \(A\) was the \textAndIndex{incidence matrix}, and \(\alpha = \begin{bmatrix}\inv{R_1} & & \\ & \inv{R_2} & \\ & & \ddots \end{bmatrix} \).
%
The \textAndIndex{stamp} can be observed in the multiplication of the contribution for a single resistor.  The incidence matrix has the form \( G = A \alpha A^\T \)
%as illustrated in \cref{fig:lecture3:lecture3Fig3}.
%\imageFigure{../figures/ece1254-multiphysics/lecture3Fig3}{Factoring of the stamp matrix.}{fig:lecture3:lecture3Fig3}{0.3}
\begin{dmath}\label{eqn:multiphysicsL3:100}
G =
\kbordermatrix{
    & \downarrow   & \\
n_1 & +1             & \\
n_2 & -1           & \\
}
\begin{bmatrix}
   & &
   & \inv{R} &
   & &
\end{bmatrix}
\kbordermatrix{
   & n_1 & n_2 \\
   & +1  & -1 \\
   &     &
}
=
\kbordermatrix{
    & n_1      & n_2 \\
n_1 & \inv{R}  & -\inv{R} \\
n_2 & -\inv{R} & \inv{R}
}
\end{dmath}
%
\paragraph{Theoretical facts}
%
Noting that \(\lr{ A B }^\T = B^\T A^\T \), it is clear that the nodal matrix \(G = A \alpha A^\T \) is symmetric \index{symmetric matrix}
%
\begin{dmath}\label{eqn:multiphysicsL3:80}
G^\T
=
\lr{ A \alpha A^\T }^\T
=
\lr{ A^\T }^\T \alpha^\T A^\T
=
A \alpha A^\T
= G
\end{dmath}
%
\section{Modified nodal analysis (MNA).}
\index{modified nodal analysis}
\index{MNA|see {modified nodal analysis}}
%
Modified nodal analysis (MNA), eliminates the branch currents for the resistive circuit elements, and is the method used to implement software such as \textAndIndex{spice}.  To illustrate the method, consider the same circuit, augmented with an additional voltage sources as in \cref{fig:lecture3:lecture3Fig4}.  This method can also accomodate voltage sources, provided an unknown current
source is also introduced for each voltage source circuit element.  The unknowns are
%
%\imageFigure{../figures/ece1254-multiphysics/lecture3Fig4}{Resistive circuit with current and voltage sources}{fig:lecture3:lecture3Fig4}{0.3}
\imageFigure{../figures/ece1254-multiphysics/1254-resistive-circuit-with-current-and-voltage-sources.pdf}{Resistive circuit with current and voltage sources.}{fig:lecture3:lecture3Fig4}{0.3}
%
\begin{itemize}
   \item node voltages (\(N-1\)): \( V_1, V_2, \cdots V_5 \)
   \item branch currents for selected components (\(K\)): \( i_{\textrm{S},\textrm{C}}, i_{\textrm{S},\textrm{D}} \)
\end{itemize}
%
Compared to standard nodal analysis, two less unknowns for this system are required.  The equations are
%
\begin{enumerate}
\item
   \(
   -\frac{V_5-V_1}{R_{\textrm{A}}}
   +\frac{V_1-V_2}{R_{\textrm{B}}}
   + i_{\textrm{S},\textrm{A}} = 0
   \)
\item
   \(
    \frac{V_2-V_5}{R_{\textrm{E}}}
   +\frac{V_2-V_1}{R_{\textrm{B}}}
   + i_{\textrm{S},\textrm{B}}
   - i_{\textrm{S},\textrm{C}}
   = 0
   \)
\item
   \(
   i_{\textrm{S},\textrm{C}} +
   \frac{V_3-V_4}{R_{\textrm{C}}} = 0
   \)
\item
   \(
   \frac{V_4-0}{R_{\textrm{D}}}
   +\frac{V_4-V_3}{R_{\textrm{C}}}
   - i_{\textrm{S},\textrm{A}}
   - i_{\textrm{S},\textrm{B}}
   = 0
   \)
\item
   \(
   \frac{V_5-V_2}{R_{\textrm{E}}}
   +\frac{V_5-V_1}{R_{\textrm{A}}}
   - i_{\textrm{S},\textrm{D}} = 0
   \)
\end{enumerate}
%
Put into giant matrix form, this is
%
\begin{equation}\label{eqn:multiphysicsL3:120}
\begin{aligned}
&
   \kbordermatrix{
      & G         &           &        &           &              &        & A_{\textrm{V}}      &    \\
      & Z_{\textrm{A}} + Z_{\textrm{B}} & -Z_{\textrm{B}}      & .      & .         & -Z_{\textrm{A}}         & \vrule &          &    \\
      & -Z_{\textrm{B}}      & Z_{\textrm{B}} - Z_{\textrm{E}} & .      & .         & -Z_{\textrm{E}}         & \vrule & -1       &    \\
      & .         & .         & Z_{\textrm{C}}    & -Z_{\textrm{C}}      & .            & \vrule & +1       &    \\
      & .         & .         & -Z_{\textrm{C}}   & Z_{\textrm{C}} + Z_{\textrm{D}} & .            & \vrule &          &    \\
      & -Z_{\textrm{A}}      & -Z_{\textrm{E}}      &        &           & Z_{\textrm{A}} + Z_{\textrm{E}}    & \vrule &          & -1 \\
      \cline{2-9}
        -A_{\textrm{V}}^\T &           & +1        & -1     &           &              & \vrule &  \\
                &           &           &        &           & 1            & \vrule &
   }
\begin{bmatrix}
   V_1 \\
   V_2 \\
   V_3 \\
   V_4 \\
   V_5 \\
%   \cline{1} \\
   i_{\textrm{S},\textrm{C}} \\
   i_{\textrm{S},\textrm{D}}
\end{bmatrix} \\
&\quad =
\begin{bmatrix}
   -i_{\textrm{S},\textrm{A}} \\
   -i_{\textrm{S},\textrm{B}} \\
   0 \\
   i_{\textrm{S},\textrm{A}} + i_{\textrm{S},\textrm{B}} \\
   0 \\
%   \cline{1}
   V_{\textrm{S},\textrm{C}} \\
   V_{\textrm{S},\textrm{D}}
\end{bmatrix}
\end{aligned}
\end{equation}
%
%\cref{fig:lecture3:lecture3Fig5}
%\imageFigure{../figures/ece1254-multiphysics/lecture3Fig5}{Nodal and voltage incidence matrices.}{fig:lecture3:lecture3Fig5}{0.3}
%
Call the extension to the \textAndIndex{nodal matrix} \(G\), the \textAndIndex{voltage incidence matrix} \(A_{\textrm{V}}\).
%
% from L4:
\paragraph{Review}
%
Additional unknowns are added for
%
\begin{itemize}
\item branch currents for voltage sources
\item all elements for which it is impossible or inconvenient to write \( i = f(v_1, v_2) \).
%
Imagine, for example, a component as illustrated in \cref{fig:lecture4:lecture4Fig1}.
%
\imageFigure{../figures/ece1254-multiphysics/lecture4Fig1}{Variable voltage device.}{fig:lecture4:lecture4Fig1}{0.15}
%
\begin{equation}\label{eqn:multiphysicsL4:20}
v_1 - v_2 = 3 i^2
\end{equation}
\item any current which is controlling dependent sources, as in \cref{fig:lecture4:lecture4Fig2}.
%
\imageFigure{../figures/ece1254-multiphysics/lecture4Fig2}{Current controlled device.}{fig:lecture4:lecture4Fig2}{0.15}
%
\item Inductors
\begin{equation}\label{eqn:multiphysicsL4:40}
v_1 - v_2 = L \frac{di}{dt}.
\end{equation}
\end{itemize}
%
%\EndArticle
%\EndNoBibArticle
