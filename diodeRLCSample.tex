%
% Copyright � 2014 Peeter Joot.  All Rights Reserved.
% Licenced as described in the file LICENSE under the root directory of this GIT repository.
%
%\input{../blogpost.tex}
%\renewcommand{\basename}{diodeRLCSample}
%\renewcommand{\dirname}{notes/ece1254/}
%\newcommand{\dateintitle}{}
%\newcommand{\keywords}{}
%
%\input{../peeter_prologue_print2.tex}
%
%\beginArtNoToc
%
%\generatetitle{A sample diode RLC circuit}
%\chapter{A sample diode RLC circuit}
\label{chap:diodeRLCSample}
%
To get a feel for how to generate the MLN equations for a circuit that has both RLC and nonlinear components, consider the circuit of \cref{fig:diodeRLC:diodeRLCFig1}.
%
%\imageFigure{../figures/ece1254-multiphysics/diodeRLCFig1}{An RLC circuit with a diode.}{fig:diodeRLC:diodeRLCFig1}{0.3}
%\imageFigure{../figures/ece1254-multiphysics/diodeRLCFig3}{An RLC circuit with a diode.}{fig:diodeRLC:diodeRLCFig1}{0.1}
%
%diodeRLCFig4
\imageFigure{../figures/ece1254-multiphysics/An-RLC-circuit-with-a-diode.pdf}{An RLC circuit with a diode.}{fig:diodeRLC:diodeRLCFig1}{0.2}
%
An-RLC-circuit-with-a-diode.pdf
%
The KCL equations for this circuit are
%
\begin{enumerate}
\item \( 0 = i_{\textrm{s}} - i_{\textrm{d}} \)
\item \( i_{\textrm{L}} + \frac{v_2 - v_3}{R} = i_{\textrm{d}} \)
\item \( \frac{v_3 - v_2}{R} + C \frac{dv_3}{dt} = 0 \)
\item \( -v_2 + L \frac{d i_{\textrm{L}}}{dt} = 0 \)
\item \( i_{\textrm{d}} = I_0 \lr{ e^{(v_1 - v_2)/v_{\textrm{T}}} - 1} \)
\end{enumerate}
%
%FIXME: for the diode, is that the right voltage sign with respect to the current direction?
%
With \( Z = 1/R \), these can be put into the standard MLN matrix form as
%
\begin{equation}\label{eqn:diodeRLCSample:20}
\begin{aligned}
&\begin{bmatrix}
0 & 0 & 0 & 0 \\
0 & Z & -Z & 1 \\
0 & -Z & Z & 0 \\
0 & -1 & 0 & 0 \\
\end{bmatrix}
\begin{bmatrix}
v_1 \\
v_2 \\
v_3 \\
i_{\textrm{L}} \\
\end{bmatrix} \\
&\qquad +
\begin{bmatrix}
0 & 0 & 0 & 0 \\
0 & 0 & 0 & 0 \\
0 & 0 & C & 0 \\
0 & 0 & 0 & L \\
\end{bmatrix}
{\begin{bmatrix}
v_1 \\
v_2 \\
v_3 \\
i_{\textrm{L}} \\
\end{bmatrix}}'
=
\begin{bmatrix}
 I_0 & 1 \\
 -I_0 & 0 \\
 0  & 0 \\
 0  & 0 \\
\end{bmatrix}
\begin{bmatrix}
1 \\
i_{\textrm{s}}(t) \\
\end{bmatrix}
+
\begin{bmatrix}
-I_0 \\
 I_0 \\
 0  \\
 0  \\
\end{bmatrix}
\begin{bmatrix}
e^{(v_2 - v_3)/v_{\textrm{T}}}
\end{bmatrix}
\end{aligned}
\end{equation}
%
Let's write this as
%
\begin{equation}\label{eqn:diodeRLCSample:40}
\BG \BX(t) + \BC \dot{\BX}(t) = \BB \Bu(t) + \BD \Bw(t).
\end{equation}
%
Here \( \Bu(t) \) collects up all the unique signature sources (for example sources with each different frequency in the system), and \( \Bw(t) \) is a vector of all the unique nonlinear (time dependent) terms.
%
Assuming a bandwidth limited periodic source we know how to express any of the time dependent variables \( v_1, ... \) in terms of their (discrete) Fourier transforms.  Suppose that
%\( V_n^{(a)} \), \( U_n^{(b)} \), and \( W_n^{(c)} \) are
the Fourier coefficients for \( v_a(t), u_b(t), w_c(t) \) are given by
%
\begin{equation}\label{eqn:diodeRLCSample:60}
\begin{aligned}
v_a(t) &= \sum_{n = -N}^N V_n^{(a)} e^{j \omega_0 n t} \\
u_b(t) &= \sum_{n = -N}^N U_n^{(b)} e^{j \omega_0 n t} \\
w_c(t) &= \sum_{n = -N}^N W_n^{(c)} e^{j \omega_0 n t}.
\end{aligned}
\end{equation}
%
For example, in this circuit, if the source was zero phase signal at the fundamental frequency of our Fourier basis (\( i_{\textrm{s}}(t) = e^{j \omega_0 t} \)), the only non-zero Fourier components \( U_n^{(a)} \) would be \( U_0^{(1)} = 1, U_1^{(2)} = 1 \).
%
First evaluating the derivatives, and then evaluating the result at each of the Nykvist times \( t_k \), yields a \( (2 N + 1) \times 1 \) equations
%
\begin{equation}\label{eqn:diodeRLCSample:80}
0 = \sum_{n=-N}^N
e^{j n \omega_0 t_k}
\lr{
\lr{
\BG + j \omega_0 n \BC
}
\begin{bmatrix}
V_n^{(1)} \\
V_n^{(2)} \\
\vdots
\end{bmatrix}
-
\BB
\begin{bmatrix}
U_n^{(1)} \\
U_n^{(2)} \\
\end{bmatrix}
-\BD
\begin{bmatrix}
W_n^{(1)} \\
\end{bmatrix}
}
\end{equation}
%
With the assumption of periodicity, each of these equations must separately equal zero for each \( ( n, k ) \) pair so that
%
\begin{equation}\label{eqn:diodeRLCSample:240}
\lr{
\BG + j \omega_0 n \BC
}
\begin{bmatrix}
V_n^{(1)} \\
V_n^{(2)} \\
\vdots
\end{bmatrix}
=
\BB
\begin{bmatrix}
U_n^{(1)} \\
U_n^{(2)} \\
\end{bmatrix}
+\BD
\begin{bmatrix}
W_n^{(1)} \\
\end{bmatrix}
\end{equation}
%
The next goal is to put this in block matrix form.  First introducing discrete time sampling vectors
%
\begin{equation}\label{eqn:diodeRLCSample:100}
\Bv_a =
\begin{bmatrix}
v_a(t_{-N}) \\
v_a(t_{1-N}) \\
\vdots \\
v_a(t_{N-1}) \\
v_a(t_{N}) \\
\end{bmatrix}, \qquad
\Bu_a =
\begin{bmatrix}
u_b(t_{-N}) \\
u_b(t_{1-N}) \\
\vdots \\
u_b(t_{N-1}) \\
u_b(t_{N}) \\
\end{bmatrix}, \qquad
\Bw_a =
\begin{bmatrix}
w_c(t_{-N}) \\
w_c(t_{1-N}) \\
\vdots \\
w_c(t_{N-1}) \\
w_c(t_{N}) \\
\end{bmatrix},
\end{equation}
%
and Fourier component vectors
%
\begin{equation}\label{eqn:diodeRLCSample:120}
\BV_a =
\begin{bmatrix}
V^{(a)}_{-N} \\
V^{(a)}_{1-N} \\
\vdots \\
V^{(a)}_{N-1} \\
V^{(a)}_{N} \\
\end{bmatrix}, \qquad
\BU_b =
\begin{bmatrix}
U^{(b)}_{-N} \\
U^{(b)}_{1-N} \\
\vdots \\
U^{(b)}_{N-1} \\
U^{(b)}_{N} \\
\end{bmatrix}, \qquad
\BW_c =
\begin{bmatrix}
W^{(c)}_{-N} \\
W^{(c)}_{1-N} \\
\vdots \\
W^{(c)}_{N-1} \\
W^{(c)}_{N} \\
\end{bmatrix}.
\end{equation}
%
With \( W = e^{ 2 \pi j /(2 N + 1) } \), and \( \BF \) equals
%
\begin{equation}\label{eqn:diodeRLCSample:140}
%\BF =
\begin{bmatrix}
 W^{ N N } &  W^{ \lr{N-1} N }  & \cdots & 1 & \cdots &  W^{ -\lr{N-1} N }  &  W^{ -N N } \\
 W^{ N \lr{N-1} } &  W^{ \lr{N-1} \lr{N-1} }  & \cdots & 1 & \cdots &  W^{ -\lr{N-1} \lr{N-1} }  &  W^{ -N \lr{N-1} } \\
 \vdots              &  \vdots                      & \vdots      & \vdots & \vdots      &  \vdots                    &  \vdots               \\
 1              &  1                      & 1      & 1 & 1      &  1                    &  1               \\
 \vdots              &  \vdots                      & \vdots      & \vdots & \vdots      &  \vdots                    &  \vdots               \\
 W^{ -N \lr{N-1} } &  W^{ -\lr{N-1} \lr{N-1} }  & \cdots & 1 & \cdots &  W^{ {N-1} \lr{N-1} }  &  W^{ N \lr{N-1} } \\
 W^{ -N N } &  W^{ -N N }  & \cdots & 1 & \cdots &  W^{ \lr{N-1} N }  &  W^{ N N }
\end{bmatrix}.
\end{equation}
%
In each case the time domain sampling vectors are related to the Fourier components by relations of the form
%
\begin{equation}\label{eqn:diodeRLCSample:160}
\Bx = \BF \BX.
\end{equation}
%
\section{Block matrix form, physical parameter ordering.}
%
To understand how to put \cref{eqn:diodeRLCSample:240} in block matrix form, it is helpful to consider a specific example.  Consider again the specific example of the RLC circuit above, which has the form
%
\begin{equation}\label{eqn:diodeRLCSample:260}
\begin{bmatrix}
0 & 0 & 0 & 0 \\
0 & Z & -Z & 1 \\
0 & -Z & Z + j \omega_0 n C & 0 \\
0 & -1 & 0 & + j \omega_0 n L
\end{bmatrix}
\begin{bmatrix}
V_n^{(1)} \\
V_n^{(2)} \\
V_n^{(3)} \\
I_n^{(L)}
\end{bmatrix}
=
\begin{bmatrix}
I_n^{(1)} \\
I_n^{(2)} \\
I_n^{(3)} \\
I_n^{(4)}
\end{bmatrix}
\end{equation}
%
Here the \( I^{(i)} \) terms are the DFT representations of both the linear and nonlinear sources.
%
Suppose for example that \( N = 1 \).  One way to write \cref{eqn:diodeRLCSample:260} would be
%
\begin{equation}\label{eqn:diodeRLCSample:320a}
\BI = \BG \BV + \BC \BV',
\end{equation}
where
\begin{equation}\label{eqn:diodeRLCSample:320b}
\BI =
{\begin{bmatrix}
I_{-1}^{(1)} &
I_0^{(1)} &
I_{1}^{(1)} &
I_{-1}^{(2)} &
I_0^{(2)} &
I_{1}^{(2)} &
I_{-1}^{(3)} &
I_0^{(3)} &
I_{1}^{(3)} &
I_{-1}^{(4)} &
I_0^{(4)} &
I_{1}^{(4)}
\end{bmatrix}}^\T,
\end{equation}
%where
%\( \BV^\T \) equals
\begin{equation}\label{eqn:diodeRLCSample:320c}
\BV^\T =
\begin{bmatrix}
V_{-1}^{(1)} &
V_{0}^{(1)} &
V_{1}^{(1)} &
V_{-1}^{(2)} &
%V_{0}^{(2)} &
\cdots &
%V_{1}^{(2)} &
%V_{-1}^{(3)} &
%V_{0}^{(3)} &
V_{1}^{(3)} &
I_{-1}^{(L)} &
I_{0}^{(L)} &
I_{1}^{(L)}
\end{bmatrix},
\end{equation}
%and
%\( \BG \) equals
\begin{equation}\label{eqn:diodeRLCSample:320d}
\BG =
\left[
\begin{array}{c|c|c|c}
% row 1:
%\begin{matrix}
%0 & 0 & 0 \\
%0 & 0 & 0 \\
%0 & 0 & 0 \\
%\end{matrix}
\Bzero
&
\Bzero
%\begin{matrix}
%0 & 0 & 0 \\
%0 & 0 & 0 \\
%0 & 0 & 0 \\
%\end{matrix}
&
\Bzero
%\begin{matrix}
%0 & 0 & 0 \\
%0 & 0 & 0 \\
%0 & 0 & 0 \\
%\end{matrix}
&
\Bzero
%\begin{matrix}
%0 & 0 & 0 \\
%0 & 0 & 0 \\
%0 & 0 & 0 \\
%\end{matrix}
\\ \hline
% row 2:
\Bzero
%\begin{matrix}
%0 & 0 & 0 \\
%0 & 0 & 0 \\
%0 & 0 & 0 \\
%\end{matrix}
&
Z \BOne
%\begin{matrix}
%Z & 0 & 0 \\
%0 & Z & 0 \\
%0 & 0 & Z \\
%\end{matrix}
&
-Z \BOne
%\begin{matrix}
%-Z & 0 & 0 \\
%0 & -Z & 0 \\
%0 & 0 & -Z \\
%\end{matrix}
&
\BOne
%\begin{matrix}
%1 & 0 & 0 \\
%0 & 1 & 0 \\
%0 & 0 & 1 \\
%\end{matrix}
\\ \hline
% row 3:
\Bzero
%\begin{matrix}
%0 & 0 & 0 \\
%0 & 0 & 0 \\
%0 & 0 & 0 \\
%\end{matrix}
&
-Z \BOne
%\begin{matrix}
%-Z & 0 & 0 \\
%0 & -Z & 0 \\
%0 & 0 & -Z \\
%\end{matrix}
&
Z \BOne
%\begin{matrix}
%Z & 0 & 0 \\
%0 & Z & 0 \\
%0 & 0 & Z \\
%\end{matrix}
&
\Bzero
%\begin{matrix}
%0 & 0 & 0 \\
%0 & 0 & 0 \\
%0 & 0 & 0 \\
%\end{matrix}
\\ \hline
% row 4:
\Bzero
%\begin{matrix}
%0 & 0 & 0 \\
%0 & 0 & 0 \\
%0 & 0 & 0 \\
%\end{matrix}
&
-\BOne
%\begin{matrix}
%-1 & 0 & 0 \\
%0 & -1 & 0 \\
%0 & 0 & -1 \\
%\end{matrix}
&
\Bzero
%\begin{matrix}
%0 & 0 & 0 \\
%0 & 0 & 0 \\
%0 & 0 & 0 \\
%\end{matrix}
&
\Bzero
%\begin{matrix}
%0 & 0 & 0 \\
%0 & 0 & 0 \\
%0 & 0 & 0 \\
%\end{matrix}
\end{array}
\right],
\end{equation}
and
\begin{equation}\label{eqn:diodeRLCSample:320e}
\BC =
j \omega_0
\left[
\begin{array}{c|c|c|c}
% row 1:
\Bzero
%\begin{matrix}
%0 & 0 & 0 \\
%0 & 0 & 0 \\
%0 & 0 & 0 \\
%\end{matrix}
&
\Bzero
%\begin{matrix}
%0 & 0 & 0 \\
%0 & 0 & 0 \\
%0 & 0 & 0 \\
%\end{matrix}
&
\Bzero
%\begin{matrix}
%0 & 0 & 0 \\
%0 & 0 & 0 \\
%0 & 0 & 0 \\
%\end{matrix}
&
\Bzero
%\begin{matrix}
%0 & 0 & 0 \\
%0 & 0 & 0 \\
%0 & 0 & 0 \\
%\end{matrix}
\\ \hline
% row 2:
\Bzero
%\begin{matrix}
%0 & 0 & 0 \\
%0 & 0 & 0 \\
%0 & 0 & 0 \\
%\end{matrix}
&
\Bzero
%\begin{matrix}
%0 & 0 & 0 \\
%0 & 0 & 0 \\
%0 & 0 & 0 \\
%\end{matrix}
&
\Bzero
%\begin{matrix}
%0 & 0 & 0 \\
%0 & 0 & 0 \\
%0 & 0 & 0 \\
%\end{matrix}
&
\Bzero
%\begin{matrix}
%0 & 0 & 0 \\
%0 & 0 & 0 \\
%0 & 0 & 0 \\
%\end{matrix}
\\ \hline
% row 3:
\Bzero
%\begin{matrix}
%0 & 0 & 0 \\
%0 & 0 & 0 \\
%0 & 0 & 0 \\
%\end{matrix}
&
\Bzero
%\begin{matrix}
%0 & 0 & 0 \\
%0 & 0 & 0 \\
%0 & 0 & 0 \\
%\end{matrix}
&
\begin{matrix}
 (-1) C & 0 & 0 \\
0 &  (0) C & 0 \\
0 & 0 &  (1) C \\
\end{matrix} &
\Bzero
%\begin{matrix}
%0 & 0 & 0 \\
%0 & 0 & 0 \\
%0 & 0 & 0 \\
%\end{matrix}
\\ \hline
% row 4:
\Bzero
%\begin{matrix}
%0 & 0 & 0 \\
%0 & 0 & 0 \\
%0 & 0 & 0 \\
%\end{matrix}
&
\Bzero
%\begin{matrix}
%0 & 0 & 0 \\
%0 & 0 & 0 \\
%0 & 0 & 0 \\
%\end{matrix}
&
\Bzero
%\begin{matrix}
%0 & 0 & 0 \\
%0 & 0 & 0 \\
%0 & 0 & 0 \\
%\end{matrix}
&
\begin{matrix}
 (-1) L & 0 & 0 \\
0 &  (0) L & 0 \\
0 & 0 &  (1) L \\
\end{matrix} \\
\end{array}
\right].
\end{equation}
%
With a vector of Fourier coefficient vectors
%
\begin{equation}\label{eqn:diodeRLCSample:280}
\BV =
\begin{bmatrix}
\BV^{(1)} \\
\BV^{(2)} \\
\BV^{(3)} \\
\BI^{(L)}
\end{bmatrix}, \qquad
\BI =
\begin{bmatrix}
\BI^{(1)} \\
\BI^{(2)} \\
\BI^{(3)} \\
\BI^{(4)}
\end{bmatrix}.
\end{equation}
%
and a \( (2 N + 1) \times (2 N + 1) \) matrix of indexes
%
\begin{equation}\label{eqn:diodeRLCSample:220}
\BN =
\begin{bmatrix}
-N &     &        &     &    \\
  & 1-N &        &     &    \\
  &     & \ddots &     &    \\
  &     &        & N-1 &    \\
  &     &        &     & N \\
\end{bmatrix},
\end{equation}
%
the complete block diagonalization is
%
%\begin{equation}\label{eqn:diodeRLCSample:300}
\boxedEquation{eqn:diodeRLCSample:300}{
{\begin{bmatrix}
g_{rs} \BI_{2 N + 1} +
j \omega_0 c_{rs} \BN
\end{bmatrix}
}_{rs}
\BV
=
\BI.
}
%\end{equation}
%
\section{Block matrix form, with frequency ordering.}
%
It turns out that a better way of ordering the vector of Fourier components is using a frequency ordering that interleaves the physical parameters.  With such an ordering the DFT MNA equations are
%
\begin{equation}\label{eqn:diodeRLCSample:11}
\BI
=
\begin{bmatrix}
I_{-1}^{(1)}  \\
I_{-1}^{(2)}  \\
I_{-1}^{(3)}  \\
I_{-1}^{(4)}  \\
I_0^{(1)}  \\
I_0^{(2)}  \\
I_0^{(3)}  \\
I_0^{(4)}  \\
I_{1}^{(1)}  \\
I_{1}^{(2)}  \\
I_{1}^{(3)}  \\
I_{1}^{(4)}  \\
\end{bmatrix}, \qquad
\BV
=
\begin{bmatrix}
V_{-1}^{(1)}  \\
V_{-1}^{(2)}  \\
V_{-1}^{(3)}  \\
I_{-1}^{(L)}  \\
V_{0}^{(1)}  \\
V_{0}^{(2)}  \\
V_{0}^{(3)}  \\
I_{0}^{(L)}  \\
V_{1}^{(1)}  \\
V_{1}^{(2)}  \\
V_{1}^{(3)}  \\
I_{1}^{(L)}  \\
\end{bmatrix},
\end{equation}
\begin{equation}\label{eqn:diodeRLCSample:340}
\begin{aligned}
&\BI
=
\left[
\begin{array}{c|c|c}
\begin{matrix}
0 & 0  & 0  & 0  \\
0 & Z  & -Z & 1 \\
0 & -Z & Z & 0 \\
0 & -1 & 0 & 0 \\
\end{matrix}
& \Bzero & \Bzero \\ \hline
\Bzero &
\begin{matrix}
0 & 0  & 0  & 0 \\
0 & Z  & -Z & 1 \\
0 & -Z & Z & 0 \\
0 & -1 & 0 & 0 \\
\end{matrix} & \Bzero \\ \hline
\Bzero & \Bzero &
\begin{matrix}
0 & 0  & 0  & 0 \\
0 & Z  & -Z & 1 \\
0 & -Z & Z & 0 \\
0 & -1 & 0 & 0 \\
\end{matrix} \\
\end{array}
\right]
\BV
\\
&+
j \omega_0
\left[
\begin{array}{c|c|c}
(-1)
\begin{bmatrix}
0 & 0 & 0 & 0 \\
0 & 0 & 0 & 0 \\
0 & 0 & C & 0 \\
0 & 0 & 0 & L \\
\end{bmatrix} & 0 & 0 \\ \hline
0 &
(0)
\begin{bmatrix}
0 & 0 & 0 & 0 \\
0 & 0 & 0 & 0 \\
0 & 0 & C & 0 \\
0 & 0 & 0 & L \\
\end{bmatrix} & 0 \\ \hline
0 & 0 &
(1)
\begin{bmatrix}
0 & 0 & 0 & 0 \\
0 & 0 & 0 & 0 \\
0 & 0 & C & 0 \\
0 & 0 & 0 & L \\
\end{bmatrix} \\
\end{array}
\right]
\BV
\end{aligned}
\end{equation}
%
This ordering matches that of \citep{giannini2004NonlinearMicrowaveCircuitDesign}.
%
\section{Representing the linear sources.}
%
Assuming real sources with frequencies that are only multiples of the fundamental harmonic, a reasonable way to represent them in storage is with a pair of matrices
%
\begin{equation}\label{eqn:diodeRLCSample:360}
\begin{bmatrix}
\BI \sim \BB \Bomega
\end{bmatrix}.
\end{equation}
%
If \( R \) is the dimension of \(\BG\) and \( \BC \), then \( \BB \) is a \( R \times S \) dimension matrix, where \( S \) is the sum of
%
\begin{itemize}
\item 1, if there are any DC sources, plus
\item 2 times the number of unique frequency sources.
\end{itemize}
%
For example, if there is a DC source and one AC source with frequency \( \nu \), then for column vectors \( \Bb_i \) this pair is of the form
%
\begin{equation}\label{eqn:diodeRLCSample:380}
\BU \Bomega =
\begin{bmatrix}
\Bb_{-1} & \Bb_0 & \Bb_1
\end{bmatrix}
\begin{bmatrix}
- 2 \pi \nu \\
0 \\
2 \pi \nu
\end{bmatrix}.
\end{equation}
%
This representation produces the time domain representation exactly when there are only DC sources, and can be used to construct the Fourier coefficients by inspection when there are AC sources.  For example, for \( N = 1 \) in the example above, the Fourier coefficient vector is
%
\begin{equation}\label{eqn:diodeRLCSample:400}
\BI
=
\begin{bmatrix}
\Bb_{-1} \\
\Bb_{0} \\
\Bb_{1} \\
\end{bmatrix}.
\end{equation}
%
If \( N = 2 \) was used, then we would have instead
%
\begin{equation}\label{eqn:diodeRLCSample:420}
\BI
=
\begin{bmatrix}
\Bzero \\
\Bb_{-1} \\
\Bb_{0} \\
\Bb_{1} \\
\Bzero \\
\end{bmatrix}.
\end{equation}
%
\section{Representing nonlinear sources.}
%
The time domain MNA \cref{eqn:diodeRLCSample:40} include a wide range of matrix dimensions.  It is now clear how to handle the transition to the frequency domain for all the linear terms.  Working a simple diode example to understand how to handle the nonlinear terms is useful.  Consider the circuit of \cref{fig:diodeWithResistor:diodeWithResistorFig1}.
%
%diodeWithResistorFig1
\imageFigure{../figures/ece1254-multiphysics/simple-diode-circuit.pdf}{Simple diode circuit.}{fig:diodeWithResistor:diodeWithResistorFig1}{0.3}
%
With \( Z = 1/R, Z_{\textrm{g}} = 1/R_{\textrm{g}} \), the KCL equations are
%
\begin{enumerate}
\item \( \lr{ v_1 - v_2 } Z_{\textrm{s}} = i_{\textrm{s}} - i_{\textrm{d}} \)
\item \( \lr{ v_2 - v_1 } Z_{\textrm{s}} + v_2 Z_{\textrm{g}} = -i_{\textrm{s}} + i_{\textrm{d}} \)
\end{enumerate}
%
Using the model \( i_{\textrm{d}} = I^{(0)} \lr{ e^{ (v_1 - v_2)/V_{\textrm{T}} } - 1 } \), with source \( i_{\textrm{s}} = I^{(s)} \cos( \omega_0 t ) \),
this has the block matrix form
%
\begin{subequations}
\begin{equation}\label{eqn:diodeRLCSample:580}
\BG =
\begin{bmatrix}
Z_{\textrm{s}} & -Z_{\textrm{s}} \\
-Z_{\textrm{s}} & Z_{\textrm{s}} + Z_{\textrm{g}} \\
\end{bmatrix}, \qquad
\Bx =
\begin{bmatrix}
v_1(t) \\
v_2(t) \\
\end{bmatrix}
\end{equation}
\begin{equation}\label{eqn:diodeRLCSample:600}
\BB =
\begin{bmatrix}
I^{(s)}/2 & -I^{(0)} & I^{(s)}/2 \\
-I^{(s)}/2 & I^{(0)} & -I^{(s)}/2
\end{bmatrix}, \qquad
\Bu(t) =
\begin{bmatrix}
e^{-j \omega_0 t} \\
1  \\
e^{j \omega_0 t}
\end{bmatrix}
\end{equation}
\begin{equation}\label{eqn:diodeRLCSample:620}
\BD
=
\begin{bmatrix}
I^{(0)} \\
-I^{(0)}
\end{bmatrix}, \qquad
\Bw(t) = e^{(v_1(t) - v_2(t))/V_{\textrm{T}}}.
\end{equation}
\end{subequations}
%
If \( E_n \) is the nth DFT coefficient for \( e(t) = e^{(v_1(t) - v_2(t))/V_{\textrm{T}}} \), then the DFT equations for the \( N = 1 \) DFT is
%
\begin{equation}\label{eqn:diodeRLCSample:640}
\begin{aligned}
\lr{ V_{-1}^{(1)} - V_{-1}^{(2)} } Z_{\textrm{s}} &= I^{(s)}/2 - I^{(0)} E_{-1}  \\
\lr{ V_{-1}^{(2)} - V_{-1}^{(1)} } Z_{\textrm{s}} + V_{-1}^{(2)} Z_{\textrm{g}} &= -I^{(s)}/2 + I^{(0)} E_{-1}  \\
\lr{ V_{0}^{(1)} - V_{0}^{(2)} } Z_{\textrm{s}} &= I^{(0)} - I^{(0)} E_{0}  \\
\lr{ V_{0}^{(2)} - V_{0}^{(1)} } Z_{\textrm{s}} + V_{0}^{(2)} Z_{\textrm{g}} &= -I^{(0)} + I^{(0)} E_{0}  \\
\lr{ V_{1}^{(1)} - V_{1}^{(2)} } Z_{\textrm{s}} &= I^{(s)}/2 - I^{(0)} E_{1}  \\
\lr{ V_{1}^{(2)} - V_{1}^{(1)} } Z_{\textrm{s}} + V_{1}^{(2)} Z_{\textrm{g}} &= -I^{(s)}/2 + I^{(0)} E_{1}
\end{aligned}
\end{equation}
%
Let \( \Bb = [ \Bb_{-1}\, \Bb_0\, \Bb_1 ] \), and \( \BD = [ \Bd_1 ] \).  The block matrix equivalent form, by inspection, is
%
\begin{equation}\label{eqn:diodeRLCSample:660}
\begin{bmatrix}
\BG & 0 & 0 \\
0 & \BG & 0  \\
0 & 0 & \BG
\end{bmatrix}
\begin{bmatrix}
V_{-1}^{(1)} \\
V_{-1}^{(2)} \\
V_{0}^{(1)} \\
V_{0}^{(2)} \\
V_{1}^{(1)} \\
V_{1}^{(2)}
\end{bmatrix}
=
\begin{bmatrix}
\Bb_{-1} \\
\Bb_{0} \\
\Bb_{1} \\
\end{bmatrix}
+
\begin{bmatrix}
\Bd_1 E_{-1} \\
\Bd_1 E_{0} \\
\Bd_1 E_{1}
\end{bmatrix}.
\end{equation}
%
This shows the stamping pattern required to form the nonlinear portion of the Harmonic balance equations.  The general pattern can be written as
%
%\begin{equation}\label{eqn:diodeRLCSample:820}
\boxedEquation{eqn:diodeRLCSample:820}{
\BY \BV = \BI + \calI(\BV),
}
%\end{equation}
%
Here \( \BY \) is block diagonal, and in general has blocks of \( \BG + j \omega_0 n \BC \).  The matrix \( \BI \) was generated from the Fourier coefficients of all the linear sources, and \( \calI(\BV) \) encodes all the nonlinear contributions to the system.
%
\paragraph{More general nonlinear structure, for multiple diodes}
\index{nonlinear!diode}
%
For the diode exponentials, these nonlinear term will be of the form
%
\begin{equation}\label{eqn:diodeRLCSample:680}
\BD \Bw(t)
=
\begin{bmatrix}
\Bd_1 & \Bd_2 & \cdots & \Bd_S
\end{bmatrix}
\begin{bmatrix}
w_1(t) \\
w_2(t) \\
\vdots \\
w_S(t) \\
\end{bmatrix},
\end{equation}
%
where \( w_i(t) = \exp( (v_{i,1}(t) - v_{i,2}(t))/V_{T,i} ) \).  If the DFT coordinates of these functions are \( E_n^{(i)} \), then the frequency domain representation is
%
\begin{equation}\label{eqn:diodeRLCSample:700}
\calI(\BV)
=
\sum_{i = 1}^S
\begin{bmatrix}
\Bd_i E_{-N}^{(i)} \\
\Bd_i E_{1-N}^{(i)} \\
\vdots \\
\Bd_i E_{N-1}^{(i)} \\
\Bd_i E_{N}^{(i)} \\
\end{bmatrix}.
\end{equation}
%
This is a \( R (2 N + 1) \times 1 \) matrix, as expected.
%
The computation of these DFT coordinates is a bit messy since they are time dependent, and thus dependent on the (unknown) values of \( V_n^{(1)} \).  Consider the above circuit as an example where we have
%
\begin{equation}\label{eqn:diodeRLCSample:720}
w(t) = \exp\lr{ (v_1(t) - v_2(t))/V_{\textrm{T}} }.
\end{equation}
%
The DFT inverse is
%
\begin{equation}\label{eqn:diodeRLCSample:740}
E_n = \sum_{k=-N}^N \exp\lr{ (v_1(t_k) - v_2(t_k))/V_{\textrm{T}} } e^{-j \omega_0 n t_k }.
\end{equation}
%
With \( \BE = ( E_{-N}, E_{1-N}, \cdots, E_{N-1}, E_N ) \), this is
%
\begin{dmath}\label{eqn:diodeRLCSample:760}
\BE =
\inv{2 N + 1} \overbar{\BF}
\begin{bmatrix}
\exp\lr{ (v_{-N}^{(1)} - v_{-N}^{(2)})/v_{\textrm{T}} } \\
\exp\lr{ (v_{1-N}^{(1)} - v_{1-N}^{(2)})/v_{\textrm{T}} } \\
\vdots \\
\exp\lr{ (v_{N-1}^{(1)} - v_{N-1}^{(2)})/v_{\textrm{T}} } \\
\exp\lr{ (v_{N}^{(1)} - v_{N}^{(2)})/v_{\textrm{T}} } \\
\end{bmatrix}
=
\inv{2 N + 1} \overbar{\BF}
\begin{bmatrix}
\exp\lr{ {[\BF (\BV^{(1)} - \BV^{(2)})/v_{\textrm{T}}]}_{-N} } \\
\exp\lr{ {[\BF (\BV^{(1)} - \BV^{(2)})/v_{\textrm{T}}]}_{1-N} } \\
\vdots \\
\exp\lr{ {[\BF (\BV^{(1)} - \BV^{(2)})/v_{\textrm{T}}]}_{N-1} } \\
\exp\lr{ {[\BF (\BV^{(1)} - \BV^{(2)})/v_{\textrm{T}}]}_{N} }
\end{bmatrix}.
\end{dmath}
%
With the introduction a term by term exponentiation operator
%
\begin{equation}\label{eqn:diodeRLCSample:780}
\exp[ \Bx ] =
\begin{bmatrix}
\exp(x_1) \\
\exp(x_2) \\
\vdots \\
\end{bmatrix},
\end{equation}
%
this one Fourier coefficient vector is
%
\begin{equation}\label{eqn:diodeRLCSample:800}
\BE =
\inv{2 N + 1} \overbar{\BF}
\exp[ \BF (\BV^{(1)} - \BV^{(2)})/v_{\textrm{T}} ].
\end{equation}
%
This is now completely expressed in terms of the unknown Fourier component vectors, each a subset of the aggregated ``voltage'', range selectable with the Matlab operation \( \BV^{(i)} = \BV(i:R:end) \).
%
\section{Newton's method.}
%
In order to solve the system, Newton's method on the Fourier coefficients is required.  Solutions to \( \calF(\BV) = 0 \) are sought, where
%
\begin{equation}\label{eqn:diodeRLCSample:840}
\calF(\BV) = \BY \BV - \BI - \calI(\BV).
\end{equation}
%
Here the sources current vector DFT coordinates have been split into the linear contributions \( \BI \) and nonlinear contributions \( \calI \) defined by \cref{eqn:diodeRLCSample:700}.
%
Working with ones-indexed coordinates of \( \BV = [V_k]_k \), where \( k \in [1, R(2 N + 1) ] \), the Jacobian is
%
\begin{equation}\label{eqn:diodeRLCSample:860}
\BJ = \BY - {[ \PDi{V_s}{\calI_r} ]}_{rs}.
\end{equation}
%
To calculate these partials we need the partials of the coordinates of \( \BE \) of
\cref{eqn:diodeRLCSample:800}.  The kth coordinate of \( \BV^{(1)}, \BV^{(2)} \) in terms of the coordinates of the \( R(2 N + 1) \) vector of unknowns \( \BV \) are
%
\begin{equation}\label{eqn:diodeRLCSample:880}
\begin{aligned}
[\BV^{(1)}]_k &= V_{1 + (k-1) R} \\
[\BV^{(2)}]_k &= V_{2 + (k-1) R}
\end{aligned}
\end{equation}
%
Using summation convention, with sums over any repeated indexes implied, those coordinates are
%
\begin{equation}\label{eqn:diodeRLCSample:900}
E_r =
\inv{2 N + 1} \overbar{F}_{r a}
\exp\lr{ F_{ab}
(
V_{1 + (b-1) R} -
V_{2 + (b-1) R}
)/V_{\textrm{T}} }.
\end{equation}
%
The partials now follow immediately
%
\begin{equation}\label{eqn:diodeRLCSample:920}
\begin{aligned}
\PD{V_s}{E_r}
&=
\inv{2 N + 1} \inv{V_{\textrm{T}}} \overbar{F}_{r a} F_{ab}
\exp\lr{ F_{ab}
(
V_{1 + (b-1) R} -
V_{2 + (b-1) R}
)/V_{\textrm{T}} }
\times \\
&\qquad
\lr{
\delta_{s,1 + (b-1) R} -
\delta_{s,2 + (b-1) R}
}.
\end{aligned}
\end{equation}
%
\paragraph{Generalization}
%
To generalize this, suppose that the diode exponential was associated with voltages spanning nodes \( m, n\) so that
%
\begin{equation}\label{eqn:diodeRLCSample:980}
\BE =
\inv{2 N + 1} \overbar{\BF}
\exp[ \BF (\BV^{(m)} - \BV^{(n)})/V_{\textrm{T}} ].
\end{equation}
%
In this case, the coordinates of the physical ``voltages'' are
%
\begin{equation}\label{eqn:diodeRLCSample:1020}
\begin{aligned}
[\BV^{(m)}]_k &= V_{m + (k-1) R} \\
[\BV^{(n)}]_k &= V_{n + (k-1) R}
\end{aligned},
\end{equation}
%
so
%
\begin{equation}\label{eqn:diodeRLCSample:1040}
E_r =
\inv{2 N + 1} \overbar{F}_{r a}
\exp\lr{ F_{ab}
(
V_{m + (b-1) R} -
V_{n + (b-1) R}
)/V_{\textrm{T}} }.
\end{equation}
%
The Jacobian partials are
%
\begin{equation}\label{eqn:diodeRLCSample:1060}
\begin{aligned}
\PD{V_s}{E_r} &=
\inv{2 N + 1} \inv{V_{\textrm{T}}} \overbar{F}_{r a} F_{ab}
\exp\lr{ F_{ab}
(
V_{m + (b-1) R} -
V_{n + (b-1) R}
)/V_{\textrm{T}} }
\times \\
&\qquad
\lr{
\delta_{s,m + (b-1) R} -
\delta_{s,n + (b-1) R}
}.
\end{aligned}
\end{equation}
%
Note that this Jacobian
%
\begin{equation}\label{eqn:diodeRLCSample:1080}
\BJ_\BE =
{
\begin{bmatrix}
\PD{V_s}{E_r}
\end{bmatrix}}_{rs}
\end{equation}
%
is a \( (2 N + 1) \times R(2N + 1) \) matrix.
%
The full Jacobian of \( \calI(\BV) \) is
%
\begin{equation}\label{eqn:diodeRLCSample:1120}
\BJ_{\calI}(\BV)
=
\sum_{i = 1}^S
\begin{bmatrix}
\Bd_i \PD{\BV}{E_{-N}^{(i)}} \\
\Bd_i \PD{\BV}{E_{1-N}^{(i)}} \\
\vdots \\
\Bd_i \PD{\BV}{E_{N-1}^{(i)}} \\
\Bd_i \PD{\BV}{E_{N}^{(i)}} \\
\end{bmatrix}.
\end{equation}
%
Where \( \PDi{\BV}{E_{k}^{(i)}} \) is the kth row of \( \BJ_{\BE^{(i)}} \).  The complete Jacobian is
%
\begin{equation}\label{eqn:diodeRLCSample:1100}
\BJ = \BY - \BJ_{\calI}(\BV).
\end{equation}
%
%%%\paragraph{Optimization attempt}
%%%
%%%Note that \cref{eqn:diodeRLCSample:920} is zero unless one of \( s - 1 \Mod R = 0 \), or \( s - 2 \Mod R = 0 \).  For example, this is zero for \( s = 3 \),
%%%but if \( s = R + 1 \) (where \( b = 2 \)) is
%%%
%%%\begin{equation}\label{eqn:diodeRLCSample:940}
%%%\PD{V_{R+1}}{E_r} =
%%%\inv{2 N + 1} \inv{V_{\textrm{T}}} \overbar{F}_{r a} F_{a2}
%%%\exp\lr{ F_{a2} V_{R+1} /V_{\textrm{T}} },
%%%\end{equation}
%%%
%%%and for \( s = R + 2 \) is
%%%
%%%\begin{equation}\label{eqn:diodeRLCSample:960}
%%%\PD{V_{R+2}}{E_r} =
%%%-\inv{2 N + 1} \inv{V_{\textrm{T}}} \overbar{F}_{r a} F_{a2}
%%%\exp\lr{ -F_{a2} V_{R+2} /V_{\textrm{T}} }.
%%%\end{equation}
%%%
%%%In the more general case, with explicit summation restored, the Jacobian coordinates can be seen to have the following banded structure
%%%
%%%\begin{equation}\label{eqn:diodeRLCSample:1000}
%%%\begin{aligned}
%%%\PD{V_{q R+m}}{E_r} &=
%%%\inv{2 N + 1} \inv{V_{\textrm{T}}} \sum_{a = 1}^{2 N + 1} \overbar{F}_{r a} F_{a, q + 1}
%%%\exp\lr{ F_{a,q+1} V_{q R+m} /V_{\textrm{T}} } \\
%%%\PD{V_{q R+n}}{E_r} &=
%%%-\inv{2 N + 1} \inv{V_{\textrm{T}}} \sum_{a = 1}^{2 N + 1} \overbar{F}_{r a} F_{a, q + 1}
%%%\exp\lr{ -F_{a, q + 1} V_{q R+n} /V_{\textrm{T}} } \\
%%%\PD{V_{s}}{E_r} &= 0, \qquad \mbox{ \( s \notin \setlr{q R + m, q R + n} \) }.
%%%\end{aligned}
%%%\end{equation}
%%%
%%%Note that the \( q \) values are all integers in the range \( [0, 2 N ] \).
%
\section{Harmonic Balance nonlinear currents (matrix).}
%
Because it was simple, the coordinate expansion of the Jacobian of the nonlinear currents above seemed like a good approach.  The end result of that Jacobian calculation was something that turned out to be impossibly slow to compute for larger \( N \).
It seems plausible that eliminating the coordinate expansion,
expressing both the current and the Jacobian directly in terms of the Harmonic Balance unknowns vector \( \BV \), would lead to a simpler set of equations that could be implemented in a computationally more effective way.
%
To aid in this discovery, consider the simple RC load diode circuit of \cref{fig:diodeWithResistor:diodeWithResistorAndCapacitorFig1}.  It's not too hard to start from scratch with the time domain nodal equations for this circuit, which are
%
%diodeWithResistorAndCapacitorFig1
\imageFigure{../figures/ece1254-multiphysics/Simple-diode-and-resistor-circuit.pdf}{Simple diode and resistor circuit.}{fig:diodeWithResistor:diodeWithResistorAndCapacitorFig1}{0.3}
%
%
\begin{enumerate}
\item \( 0 = i_{\textrm{s}} - i_{\textrm{d}} \)
\item \( Z v^{(2)} + C dv^{(2)}/dt = i_{\textrm{d}}  \)
\item \( i_{\textrm{d}} = I_0 \lr{ e^{(v^{(1)} - v^{(2)})/V_{\textrm{T}}} - 1} \)
\end{enumerate}
%
To setup for matrix form, let
%
\begin{subequations}
\begin{equation}\label{eqn:diodeRLCSample:1240}
\Bv(t) =
\begin{bmatrix}
v^{(1)}(t) \\
v^{(2)}(t) \\
\end{bmatrix}
\end{equation}
\begin{equation}\label{eqn:diodeRLCSample:1140}
\BG =
\begin{bmatrix}
0 & 0 \\
0 & Z \\
\end{bmatrix}
\end{equation}
\begin{equation}\label{eqn:diodeRLCSample:1160}
\BC =
\begin{bmatrix}
0 & 0 \\
0 & C \\
\end{bmatrix}
\end{equation}
\begin{equation}\label{eqn:diodeRLCSample:1180}
\Bd =
\begin{bmatrix}
1 \\
-1
\end{bmatrix}
\end{equation}
\begin{equation}\label{eqn:diodeRLCSample:1200}
\Bb =
\begin{bmatrix}
1 \\
0
\end{bmatrix},
\end{equation}
\end{subequations}
%
so that the time domain equations can be written as
%
\begin{dmath}\label{eqn:diodeRLCSample:1220}
\BG \Bv(t)
+ \BC \Bv'(t)
=
\Bb i_{\textrm{s}}(t)
+
I_0
\Bd
\lr{
e^{ (v^{(1)}(t) - v^{(2)}(t))/V_{\textrm{T}}} - 1
}
=
\begin{bmatrix}
\Bb & -I_0 \Bd
\end{bmatrix}
\begin{bmatrix}
i_{\textrm{s}}(t) \\
1
\end{bmatrix}
+
I_0 \Bd
e^{ (v^{(1)}(t) - v^{(2)}(t))/V_{\textrm{T}}}.
\end{dmath}
%
Harmonic Balance is essentially the
assumption that the input and outputs are assumed to be a bandwidth limited periodic signal, and the nonlinear components can be approximated by the same
%
\begin{subequations}
\begin{equation}\label{eqn:diodeRLCSample:1260}
i_{\textrm{s}}(t) = \sum_{n=-N}^N I^{(s)}_n e^{ j \omega_0 n t },
\end{equation}
\begin{equation}\label{eqn:diodeRLCSample:1280}
v^{(k)}(t) =
\sum_{n=-N}^N V^{(k)}_n e^{ j \omega_0 n t },
\end{equation}
\begin{equation}\label{eqn:diodeRLCSample:1300}
\epsilon(t) =
e^{ (v^{(1)}(t) - v^{(2)}(t))/V_{\textrm{T}}}
\simeq
\sum_{n=-N}^N E_n e^{ j \omega_0 n t },
\end{equation}
\end{subequations}
%
The approximation in \cref{eqn:diodeRLCSample:1300} is an equality only at the Nykvist sampling times \( t_k = T k/(2 N + 1) \).  The Fourier series provides a periodic extension to other times that will approximate the underlying periodic nonlinear relation.
%
With all the time dependence locked into the exponentials, the derivatives are really easy to calculate
%
\begin{equation}\label{eqn:diodeRLCSample:1281}
\frac{d}{dt} v^{(k)}(t) =
\sum_{n=-N}^N j \omega_0 n V^{(k)}_n e^{ j \omega_0 n t }.
\end{equation}
%
Inserting all of these into \cref{eqn:diodeRLCSample:1220} gives
%
\begin{equation}\label{eqn:diodeRLCSample:1320}
\sum_{n=-N}^N e^{ j \omega_0 n t} \lr{ \BG + j \omega_0 n \BC }
\begin{bmatrix}
V^{(1)}_n \\
V^{(2)}_n \\
\end{bmatrix}
=
\sum_{n=-N}^N e^{ j \omega_0 n t}
\lr{
-I_0 \Bd \delta_{n 0}
+
\Bb I^{(s)}_n
+ I_0 \Bd E_n
}.
\end{equation}
%
The periodic assumption requires equality for each \( e^{j \omega_0 n t} \), or
%
\begin{equation}\label{eqn:diodeRLCSample:1340}
\lr{ \BG + j \omega_0 n \BC }
\begin{bmatrix}
V^{(1)}_n \\
V^{(2)}_n \\
\end{bmatrix}
=
-I_0 \Bd \delta_{n 0}
+
\Bb I^{(s)}_n
+ I_0 \Bd E_n.
\end{equation}
%
For illustration, consider the \( N = 1 \) case, where the block matrix form is
%
\begin{equation}\label{eqn:diodeRLCSample:1360}
\begin{aligned}
&\begin{bmatrix}
\BG + j \omega_0 (-1) \BC & 0 & 0 \\
0 & \BG + j \omega_0 (0) \BC & 0 \\
0 & 0 & \BG + j \omega_0 (1) \BC
\end{bmatrix}
\begin{matrix}
\begin{bmatrix}
V^{(1)}_{-1} \\
V^{(2)}_{-1} \\
\end{bmatrix} \\
\begin{bmatrix}
V^{(1)}_{0} \\
V^{(2)}_{0} \\
\end{bmatrix} \\
\begin{bmatrix}
V^{(1)}_{1} \\
V^{(2)}_{1} \\
\end{bmatrix}
\end{matrix} \\
&\qquad =
\begin{bmatrix}
\Bb I^{(s)}_{-1} \\
\Bb I^{(s)}_{0} - I_0 \Bd \\
\Bb I^{(s)}_{1} \\
\end{bmatrix}
+
I_0
\begin{bmatrix}
\Bd E_{-1} \\
\Bd E_{0} \\
\Bd E_{1} \\
\end{bmatrix}.
\end{aligned}
\end{equation}
%
The structure of this equation is
%
\begin{equation}\label{eqn:diodeRLCSample:1380}
\BY \BV = \BI + \calI(\BV),
\end{equation}
%
The nonlinear current \( \calI(\BV) \) needs to be examined further.  How much of this can be precomputed, and what is the simplest way to compute the Jacobian?
%
With
%
\begin{equation}\label{eqn:diodeRLCSample:1400}
\BE =
\begin{bmatrix}
E_{-1} \\
E_{0} \\
E_{1} \\
\end{bmatrix}, \qquad
\Bepsilon =
\begin{bmatrix}
\epsilon_{-1} \\
\epsilon_{0} \\
\epsilon_{1} \\
\end{bmatrix},
\end{equation}
%
the nonlinear current is
%
\begin{dmath}\label{eqn:diodeRLCSample:1420}
\calI =
I_0
\begin{bmatrix}
\Bd E_{-1} \\
\Bd E_{0} \\
\Bd E_{1} \\
\end{bmatrix}
=
I_0
\begin{bmatrix}
\Bd \begin{bmatrix} 1 & 0 & 0 \end{bmatrix} \BE \\
\Bd \begin{bmatrix} 0 & 1 & 0 \end{bmatrix} \BE \\
\Bd \begin{bmatrix} 0 & 0 & 1 \end{bmatrix} \BE
\end{bmatrix}
=
I_0
\begin{bmatrix}
\Bd & 0 & 0 \\
0 & \Bd & 0 \\
0 & 0 & \Bd
\end{bmatrix}
\BF^{-1} \Bepsilon
\end{dmath}
%
In the last step \( \BE = \BF^{-1} \Bepsilon \) has been factored out (in its inverse Fourier form).  With
%
\begin{dmath}\label{eqn:diodeRLCSample:1480}
\BD =
\begin{bmatrix}
\Bd & 0 & 0 \\
0 & \Bd & 0 \\
0 & 0   & \Bd \\
\end{bmatrix},
\end{dmath}
%
the current is
%
%\begin{dmath}\label{eqn:diodeRLCSample:1540}
\boxedEquation{eqn:diodeRLCSample:1540}{
\calI(\BV) =
I_0 \BD \BF^{-1} \Bepsilon(\BV).
}
%\end{dmath}
%
The next step is finding an appropriate form for \( \Bepsilon \)
%
\begin{dmath}\label{eqn:diodeRLCSample:1440}
\Bepsilon =
\begin{bmatrix}
\epsilon(t_{-1}) \\
\epsilon(t_{0}) \\
\epsilon(t_{1}) \\
\end{bmatrix}
=
\begin{bmatrix}
\exp\lr{ \lr{ v^{(1)}_{-1} - v^{(2)}_{-1} }/V_{\textrm{T}} } \\
\exp\lr{ \lr{ v^{(1)}_{0} - v^{(2)}_{0} }/V_{\textrm{T}} } \\
\exp\lr{ \lr{ v^{(1)}_{1} - v^{(2)}_{1} }/V_{\textrm{T}} }
\end{bmatrix}
=
\begin{bmatrix}
\exp\lr{
\begin{bmatrix}
1 & 0 & 0
\end{bmatrix}
\lr{ \Bv^{(1)} - \Bv^{(2)} }/V_{\textrm{T}} } \\
\exp\lr{
\begin{bmatrix}
0 & 1 & 0
\end{bmatrix}
\lr{ \Bv^{(1)} - \Bv^{(2)} }/V_{\textrm{T}} } \\
\exp\lr{
\begin{bmatrix}
0 & 0 & 1
\end{bmatrix}
\lr{ \Bv^{(1)} - \Bv^{(2)} }/V_{\textrm{T}} } \\
\end{bmatrix}
=
\begin{bmatrix}
\exp\lr{
\begin{bmatrix}
1 & 0 & 0
\end{bmatrix}
\BF
\lr{ \BV^{(1)} - \BV^{(2)} }/V_{\textrm{T}} } \\
\exp\lr{
\begin{bmatrix}
0 & 1 & 0
\end{bmatrix}
\BF
\lr{ \BV^{(1)} - \BV^{(2)} }/V_{\textrm{T}} } \\
\exp\lr{
\begin{bmatrix}
0 & 0 & 1
\end{bmatrix}
\BF
\lr{ \BV^{(1)} - \BV^{(2)} }/V_{\textrm{T}} } \\
\end{bmatrix}.
\end{dmath}
%
It would be nice to have the difference of frequency domain vectors expressed in terms of \( \BV \), which can be done with a bit of rearrangement
%
\begin{dmath}\label{eqn:diodeRLCSample:1460}
\BV^{(1)} - \BV^{(2)}
=
\begin{bmatrix}
V^{(1)}_{-1} - V^{(2)}_{-1} \\
V^{(1)}_{0} - V^{(2)}_{0} \\
V^{(1)}_{1} - V^{(2)}_{1} \\
\end{bmatrix}
=
\begin{bmatrix}
1 & -1 & 0 & 0 & 0 & 0 \\
0 &  0 & 1 & -1 & 0 & 0 \\
0 &  0 & 0 & 0 & 1 & -1 \\
\end{bmatrix}
\begin{bmatrix}
V_{-1}^{(1)} \\
V_{-1}^{(2)} \\
V_{0}^{(1)} \\
V_{0}^{(2)} \\
V_{1}^{(1)} \\
V_{1}^{(2)} \\
\end{bmatrix}
=
\begin{bmatrix}
\Bd^\T & 0 & 0 \\
0 & \Bd^\T & 0 \\
0 & 0 & \Bd^\T \\
\end{bmatrix}
\BV
= \BD^\T \BV,
\end{dmath}
%
%To summarize, let
%
%\begin{equation}\label{eqn:diodeRLCSample:1500}
%\begin{aligned}
%\BA^{(1)} &=
%I_0 \begin{bmatrix} \Bd & 0 & 0 \end{bmatrix} \BF^{-1} \\
%\BA^{(2)} &=
%I_0 \begin{bmatrix} 0 & \Bd & 0 \end{bmatrix} \BF^{-1} \\
%\BA^{(3)} &=
%I_0 \begin{bmatrix} 0 & 0 & \Bd \end{bmatrix} \BF^{-1} \\
%\end{aligned},
%\end{equation}
%
%and
\begin{equation}\label{eqn:diodeRLCSample:1520}
\BH
=
\BF \BD^\T /V_{\textrm{T}}
=
\begin{bmatrix}
\Bh_1^\T \\
\Bh_2^\T \\
\Bh_3^\T
\end{bmatrix},
\end{equation}
%
which allows the nonlinear current to can now be completely expressed in terms of \( \BV \).
%
%\begin{dmath}\label{eqn:diodeRLCSample:1560}
\boxedEquation{eqn:diodeRLCSample:1560}{
\Bepsilon(\BV)
=
\begin{bmatrix}
e^{\Bh_1^\T \BV} \\
e^{\Bh_2^\T \BV} \\
e^{\Bh_3^\T \BV} \\
\end{bmatrix}.
}
%\end{dmath}
%
\paragraph{Jacobian}
\index{Jacobian}
%
With a compact matrix representation of the nonlinear current, attention can now be turned to the Jacobian of the nonlinear current.  Let \( \BA = I_0 \BD \BF^{-1} = [ a_{ij} ]_{ij} \), the current (with summation implied) is
%
\begin{dmath}\label{eqn:diodeRLCSample:1580}
\calI =
\begin{bmatrix}
a_{ik} \epsilon_k,
\end{bmatrix}
\end{dmath}
%
with coordinates
%
\begin{dmath}\label{eqn:diodeRLCSample:1600}
\calI_i = a_{ik} \epsilon_k = a_{ik} \exp\lr{ \Bh_k^\T \BV }.
\end{dmath}
%
so the Jacobian components are
%
\begin{dmath}\label{eqn:diodeRLCSample:1620}
[\BJ^{\calI}]_{ij}
=
a_{ik} \epsilon_k = a_{ik}
\PD{V_j}{}
\exp\lr{ \Bh_k^\T \BV }
=
a_{ik}
h_{kj}
\exp\lr{ \Bh_k^\T \BV }.
\end{dmath}
%
Factoring out \( \BU = [h_{ij} \exp\lr{ \Bh_i^\T \BV }]_{ij} \),
%
\begin{dmath}\label{eqn:diodeRLCSample:1640}
\BJ^{\calI}
= \BA \BU
=
\BA
\begin{bmatrix}
\begin{bmatrix} h_{11} & h_{12} & \cdots h_{1, R(2 N + 1)}\end{bmatrix} \exp\lr{ \Bh_1^\T \BV } \\
\begin{bmatrix} h_{21} & h_{22} & \cdots h_{2, R(2 N + 1)}\end{bmatrix} \exp\lr{ \Bh_2^\T \BV } \\
\begin{bmatrix} h_{31} & h_{32} & \cdots h_{3, R(2 N + 1)}\end{bmatrix} \exp\lr{ \Bh_3^\T \BV } \\
\end{bmatrix}
=
\BA
\begin{bmatrix}
\Bh_1^\T \exp\lr{ \Bh_1^\T \BV } \\
\Bh_2^\T \exp\lr{ \Bh_2^\T \BV } \\
\Bh_3^\T \exp\lr{ \Bh_3^\T \BV } \\
\end{bmatrix}.
\end{dmath}
%
A quick sanity check of dimensions seems worthwhile, and shows that all is well
%
\begin{itemize}
\item \( \BA \) : \( R(2 N + 1) \times 2 N + 1 \)
\item \( \BU \) : \( 2 N + 1 \times R(2 N + 1) \)
\item \( \BJ^{\calI} \) : \( R(2 N + 1) \times R(2 N + 1) \)
\end{itemize}
%
The Jacobian of the nonlinear current is now completely determined
%
%\begin{dmath}\label{eqn:diodeRLCSample:1660}
\boxedEquation{eqn:diodeRLCSample:1660}{
\BJ^{\calI}( \BV ) =
I_0 \BD \BF^{-1}
\begin{bmatrix}
\Bh_1^\T \exp\lr{ \Bh_1^\T \BV } \\
\Bh_2^\T \exp\lr{ \Bh_2^\T \BV } \\
\Bh_3^\T \exp\lr{ \Bh_3^\T \BV } \\
\end{bmatrix}.
%\end{dmath}
}
%
\paragraph{Newton's method solution}
\index{Newton's method!diode circuit}
%
All the pieces required for a Newton's method solution are now in place.  The goal is to find a value of \( \BV \) that provides the zero
%
\begin{dmath}\label{eqn:diodeRLCSample:1680}
f(\BV) = \BY \BV - \BI - \calI(\BV).
\end{dmath}
%
Expansion to first order around an initial guess \( \BV^0 \), gives
%
\begin{dmath}\label{eqn:diodeRLCSample:1700}
f( \BV^0 + \Delta \BV ) = f(\BV^0) + \BJ(\BV^0) \Delta \BV \approx 0,
\end{dmath}
%
where the full Jacobian of \( f(\BV) \) is
%
\begin{dmath}\label{eqn:diodeRLCSample:1720}
\BJ(\BV) = \BY - \BJ^{\calI}(\BV).
\end{dmath}
%
The Newton's method refinement of the initial guess follows by inversion
%
\begin{dmath}\label{eqn:diodeRLCSample:1740}
\Delta \BV = -\lr{ \BY - \BJ^{\calI}(\BV^0) }^{-1} f(\BV^0).
\end{dmath}
%
%\EndArticle
