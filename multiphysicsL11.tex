%
% Copyright � 2014 Peeter Joot.  All Rights Reserved.
% Licenced as described in the file LICENSE under the root directory of this GIT repository.
%
%\input{../blogpost.tex}
%\renewcommand{\basename}{multiphysicsL11}
%\renewcommand{\dirname}{notes/ece1254/}
%\newcommand{\keywords}{ECE1254H}
%\input{../peeter_prologue_print2.tex}
%
%\usepackage{algorithmic}
%\usepackage{kbordermatrix}
%
%\beginArtNoToc
%\generatetitle{ECE1254H Modeling of Multiphysics Systems.  Lecture 11: Nonlinear equations.  Taught by Prof.\ Piero Triverio}
%\chapter{Nonlinear equations}
\label{chap:multiphysicsL11}

%\section{Disclaimer}
%
%Peeter's lecture notes from class.  These may be incoherent and rough.
%
\section{Solution of N nonlinear equations in N unknowns}
\index{nonlinear equations!multivariable}

It is now time to
move from solutions of nonlinear functions in one variable:

\begin{equation}\label{eqn:multiphysicsL11:200}
f(x^\conj) = 0,
\end{equation}

to multivariable systems of the form

\begin{equation}\label{eqn:multiphysicsL11:20}
\begin{aligned}
f_1(x_1, x_2, \cdots, x_N) &= 0 \\
\vdots & \\
f_N(x_1, x_2, \cdots, x_N) &= 0 \\
\end{aligned},
\end{equation}

where the unknowns are
\begin{equation}\label{eqn:multiphysicsL11:40}
\Bx =
\begin{bmatrix}
x_1 \\
x_2 \\
\vdots \\
x_N \\
\end{bmatrix}.
\end{equation}

%FIXME
Form the vector \( F \)
%: \text{\R{N}} \rightarrow \text{\R{N}} \)

\begin{equation}\label{eqn:multiphysicsL11:60}
F(\Bx) =
\begin{bmatrix}
f_1(x_1, x_2, \cdots, x_N) \\
\vdots \\
f_N(x_1, x_2, \cdots, x_N) \\
\end{bmatrix},
\end{equation}

so that the equation to solve is

\boxedEquation{eqn:multiphyiscsL11:80}{
F(\Bx) = 0.
}
%\end{equation}

The Taylor expansion of \( F \)
%FIXME
%: \text{\R}^N \rightarrow \text{\R}^N \)
around point \( \Bx_0 \) is

\begin{equation}\label{eqn:multiphysicsL11:100}
F(\Bx) = F(\Bx_0) +
\mathLabelBox
{
J_F(\Bx_0)
}
{Jacobian}
\lr{ \Bx - \Bx_0},
\end{equation}

where the Jacobian is

\begin{equation}\label{eqn:multiphysicsL11:120}
J_F(\Bx_0)
=
\begin{bmatrix}
\PD{x_1}{f_1} & \hdots & \PD{x_N}{f_1} \\
              & \ddots &  \\
\PD{x_1}{f_N} & \hdots & \PD{x_N}{f_N}
\end{bmatrix}
\end{equation}

\section{Multivariable Newton's iteration}
\index{Newton's method!multivariable}

Given \( \Bx^k \), expand \( F(\Bx) \) around \( \Bx^k \)

\begin{equation}\label{eqn:multiphysicsL11:140}
F(\Bx) \approx F(\Bx^k) + J_F(\Bx^k) \lr{ \Bx - \Bx^k }
\end{equation}

Applying the approximation

\begin{equation}\label{eqn:multiphysicsL11:160}
0 = F(\Bx^k) + J_F(\Bx^k) \lr{ \Bx^{k + 1} - \Bx^k },
\end{equation}

multiplying by the inverse \textAndIndex{Jacobian}, and some rearranging gives

%\begin{equation}\label{eqn:multiphysicsL11:180}
\boxedEquation{eqn:multiphyiscsL11:220}{
\Bx^{k+1} = \Bx^k - J_F^{-1}(\Bx^k) F(\Bx^k).
}
%\end{equation}

The algorithm is
\makealgorithm{Newton's method.}{alg:multiphysicsL11:401}{
\STATE Guess \( \Bx^0, k = 0 \).
\REPEAT
\STATE Compute \( F \) and \( J_F \) at \( \Bx^k \)
\STATE Solve linear system  \( J_F(\Bx^k) \Delta \Bx^k = - F(\Bx^k) \)
\STATE \( \Bx^{k+1} = \Bx^k + \Delta \Bx^k \)
\STATE \( k = k + 1 \)
\UNTIL{converged}
}

As with one variable, there are a number of convergence conditions to check for

\begin{equation}\label{eqn:multiphysicsL11:240}
\begin{aligned}
\Norm{ \Delta \Bx^k } &< \epsilon_1 \\
\Norm{ F(\Bx^{k+1}) } &< \epsilon_2 \\
\frac{\Norm{ \Delta \Bx^k }}{\Norm{\Bx^{k+1}}} &< \epsilon_3 \\
\end{aligned}
\end{equation}

Typical termination is some multiple of eps, where eps is the machine precision.  This may be something like:

\begin{equation}\label{eqn:multiphysicsL11:260}
4 \times N \times \text{eps},
\end{equation}

where \( N \) is the ``size of the problem''.  Sometimes there may be physically meaningful values for the problem that define the desirable iteration cutoff.  For example, for a voltage problem, it might be that precisions greater than a millivolt are not of interest.

\section{Automatic assembly of equations for nonlinear system}

\paragraph{Nonlinear circuits}
\index{nonlinear resistor}

Start off by considering a nonlinear resistor, designated within a circuit as sketched in \cref{fig:lecture11:lecture11Fig2}.

\imageFigure{../../figures/ece1254/lecture11Fig2}{Non-linear resistor.}{fig:lecture11:lecture11Fig2}{0.1}

Example: diode, with \( i = g(v) \), such as

\begin{equation}\label{eqn:multiphysicsL11:280}
i = I_0 \lr{ e^{v/{\eta V_{\textrm{T}}}} - 1 }.
\end{equation}

Consider the example circuit of \cref{fig:lecture11:lecture11Fig3}.  KCL's at each of the nodes are

%\imageFigure{../../figures/ece1254/lecture11Fig3}{Example circuit}{fig:lecture11:lecture11Fig3}{0.3}
\imageFigure{../../figures/ece1254/example-circuit-variable-resisitors.pdf}{Example circuit.}{fig:lecture11:lecture11Fig3}{0.3}

\begin{enumerate}
\item \( I_{\textrm{A}} + I_{\textrm{B}} + I_{\textrm{D}} - I_{\textrm{s}} = 0 \)
\item \( - I_{\textrm{B}} + I_{\textrm{C}} - I_{\textrm{D}} = 0 \)
\end{enumerate}

Introducing the constitutive equations this is

\begin{enumerate}
\item \( g_{\textrm{A}}(V_1) + g_{\textrm{B}}(V_1 - V_2) + g_{\textrm{D}} (V_1 - V_2) - I_{\textrm{s}} = 0 \)
\item \( - g_{\textrm{B}}(V_1 - V_2) + g_{\textrm{C}}(V_2) - g_{\textrm{D}} (V_1 - V_2) = 0 \)
\end{enumerate}

In matrix form this is
\begin{equation}\label{eqn:multiphysicsL11:300}
\begin{bmatrix}
g_{\textrm{D}} & -g_{\textrm{D}} \\
-g_{\textrm{D}} & g_{\textrm{D}}
\end{bmatrix}
\begin{bmatrix}
V_1 \\
V_2
\end{bmatrix}
+
\begin{bmatrix}
g_{\textrm{A}}(V_1) &+ g_{\textrm{B}}(V_1 - V_2) &            & - I_{\textrm{s}} \\
         &- g_{\textrm{B}}(V_1 - V_2) & + g_{\textrm{C}}(V_2) & \\
\end{bmatrix}
=
0
%\begin{bmatrix}
%I_{\textrm{s}} \\
%0
%\end{bmatrix}
.
\end{equation}

The entire system can be written in matrix form as

%\begin{equation}\label{eqn:multiphysicsL11:320}
\boxedEquation{eqn:multiphyiscsL11:320}{
F(\Bx) = G \Bx + F'(\Bx) = 0.
}
%\end{equation}

The first term, a product of a nodal matrix \( G \) represents the linear subnetwork, and is filled with the stamps that are already familiar.

The second term encodes the relationships of the nonlinear subnetwork.   This nonlinear component has been marked with a prime to distinguish it from the complete network function that includes both linear and nonlinear elements.

Observe the similarity with the previous stamp analysis.  With \( g_{\textrm{A}}() \) connected on one end to ground it occurs only in the resulting vector, whereas the nonlinear elements connected to two non-zero nodes in the network occur once with each sign.

\paragraph{Stamp for nonlinear resistor}
\index{nonlinear resistor!stamp}

For the nonlinear circuit element of \cref{fig:lecture11:lecture11Fig4}.

\imageFigure{../../figures/ece1254/lecture11Fig4}{Non-linear resistor circuit element.}{fig:lecture11:lecture11Fig4}{0.1}

\begin{equation}\label{eqn:multiphysicsL11:340}
F'(\Bx) =
\kbordermatrix{
	& \\
	n_1 \rightarrow & + g( V_{n_1} - V_{n_2} ) \\
	n_2 \rightarrow & - g( V_{n_1} - V_{n_2} ) \\
}
\end{equation}

\paragraph{Stamp for Jacobian}
\index{stamp!Jacobian}

\begin{equation}\label{eqn:multiphysicsL11:360}
J_F(\Bx^k) = G + J_{F'}(\Bx^k).
\end{equation}

Here the stamp for the Jacobian, an \( N \times N \) matrix, is

\begin{equation}\label{eqn:multiphysicsL11:n}
J_{F'}(\Bx^k)
=
\kbordermatrix{
       & V_1 & \cdots & V_{n_1} & V_{n_2} & \cdots & V_N \\
1      &     &        &         &         &        &     \\
\vdots &     &        &         &         &        &     \\
n_1    &     &        & \PD{V_{n_1}}{g(V_{n_1} - V_{n_2})}   & \PD{V_{n_2}}{g(V_{n_1} - V_{n_2})}        &        &     \\
n_2    &     &        & -\PD{V_{n_1}}{g(V_{n_1} - V_{n_2})}  & -\PD{V_{n_2}}{g(V_{n_1} - V_{n_2})}        &        &     \\
\vdots &     &        &         &         &        &     \\
N      &     &        &         &         &        &     \\
}.
\end{equation}

%\EndArticle
%\EndNoBibArticle
