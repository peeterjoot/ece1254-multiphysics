%
% Copyright � 2014 Peeter Joot.  All Rights Reserved.
% Licenced as described in the file LICENSE under the root directory of this GIT repository.
%
\index{model order reduction}
\makeproblem{Model order reduction methods.}{multiphysics:problemSet3b:1}{

In this problem you will apply different model order reduction methods to
the heat conducting bar. Use Backward Euler for time integration.

\makesubproblem{}{multiphysics:problemSet3b:1a}
In this problem we shall introduce a heat source \( u(t) \) and its
spatial distribution in the form of \( h(x) \).

\begin{equation}\label{eqn:multiphysicsProblemSet3b:20}
\PD{t}{T(x,t)}
=
\PDSq{x}{T(x,t)}
- \alpha T(x,t) + h(x)u(t) \qquad x \in [0,1]
\end{equation}

where \( T(x,t) \) is the temperature at location \( x \) at time \( t \). The boundary condition is that no heat flows away from the two ends of the bar, i.e.

\begin{equation}\label{eqn:multiphysicsProblemSet3b:40}
\evalbar{\PD{x}{T}}{x = 0}
=
\evalbar{\PD{x}{T}}{x = 1}
= 0.
\end{equation}

The term with \( \alpha \) models the heat dissipation from the bar to the surrounding
environment. Let \( \alpha = 0.01 \). You can assume zero initial conditions. Explain
how this heat equation can be formulated into an equivalent dynamical system in the form

\begin{equation}\label{eqn:multiphysicsProblemSet3b:60}
\BG \Bx(t) + \BC \dot{\Bx}(t) = \BB \Bu(t)
\end{equation}

Explain the choice of your matrices \( \BG,\BC,\BB \). What do the states \( \Bx(t) \) of
your dynamical system correspond to, physically? We define the output \( \By(t) \)
as \( \By(t) = \BL^\T \Bx(t) \), where \( {}^\T \) denotes the matrix transpose.

\makesubproblem{}{multiphysics:problemSet3b:1b}
We are interested in the following scenarios:

\begin{itemize}
   \item [\textbf{Case 1}]
\par
Input: heat flow at the left end of the bar
\par
Output: temperature at the right end
   \item [\textbf{Case 2}]
\par
Input: heat flow at the left end of the bar
\par
Output: average temperature along the bar
   \item [\textbf{Case 3}]
\par
Input: uniform heating
\par
Output: temperature at the right end
   \item [\textbf{Case 4}]
\par
Input: uniform heating
\par
Output: average temperature along the bar
\end{itemize}

Explain how you will pick the matrices \( \BB, \BL \) in each case.
\textbf{In all remaining questions, consider only case 1.}

\makesubproblem{}{multiphysics:problemSet3b:1c}

Write a Matlab routine \matlabFunc{PlotFreqResp(\(\omega\),G,C,B,L)} which takes
in \( \omega,G,C,B,L \) as input and plots the system frequency response. Here \( \omega \)
is a vector of frequencies in rad/s. For \( N = 500 \) (500 nodes) plot the real
and imaginary part of the frequency response for case 1.
\paragraph{Hints:} To plot the frequency response, use the command \matlabFunc{semilogx(x,y)}.
You can generate the frequency vector with the command: \matlabFunc{\( \omega \) = logspace(-8,4,500)}.

\makesubproblem{}{multiphysics:problemSet3b:1d}

Reduce the dynamical system for \( N = 500 \) using the modal
truncation method. Retain the modes that are associated with the ``slowest'' eigenvalues. Note that you will have to convert your system from the
modified nodal analysis representation to the state-space representation. Use
\( q = 1, 2, 4, 10, 50 \) (\( q = \text{number of states in the reduced system} \)). Plot the
frequency response of the original system and frequency response of the reduced system in the same plot. Select a ``reasonably small'' order \( q \) for the
reduced model, that ensures a reasonable approximation of the original system. Clearly explain how did you pick that \( q \), which factors did you look at.


\makesubproblem{}{multiphysics:problemSet3b:1e}

Using your time domain solver, compute the output of the
original system (\( N=500 \)) and of the reduced model (with the order \( q \) you
selected) for each of the following two inputs:

\begin{enumerate}
\item Take the first input \( u(t) = u_1(t) \) as

\begin{equation}\label{eqn:multiphysicsProblemSet3b:80}
u_1(t) =
\left\{
\begin{array}{l l}
1 & \quad \mbox{if \( t \ge 0 \)} \\
0 & \quad \mbox{if \( t < 0 \)}
\end{array}
\right.
\end{equation}

For \( u_1(t) \) pick \( t_{\textrm{stop}} \) such that the system reaches steady state.

\item Take the second input \( u(t) = u_2(t) \) as

\begin{equation}\label{eqn:multiphysicsProblemSet3b:100}
u_2(t) = \sin(0.01 t) \quad \mbox{for \( t_{\textrm{stop}} = 10000 \)}
\end{equation}
\end{enumerate}

Be sure to pick an appropriate time step and explain your choice. How much
speed-up did you get for your reduced models for time domain simulations?
Explain your results.

\makesubproblem{}{multiphysics:problemSet3b:1f}
Repeat
\partref{multiphysics:problemSet3b:1d}
using the PRIMA model order reduction algorithm (see provided routine \textbf{prima.m}). In order to maximize the efficiency of
the reduced model, after you generate it with \matlabText{prima()}, bring it to state-space
form (by multiplying by \(\BC^{-1}\) ). Then, make the \( \BA \) matrix of the state space
model diagonal. In this way, the reduced model becomes sparse rather than
full, and can be solved more quickly.

\makesubproblem{}{multiphysics:problemSet3b:1g}

Repeat
\partref{multiphysics:problemSet3b:1e}
using the PRIMA model order reduction algorithm.

\makesubproblem{}{multiphysics:problemSet3b:1h}

Compare the results obtained with the two methods. A table
with a few columns will suffice.

} % makeproblem

\makeanswer{multiphysics:problemSet3b:1}{
\withproblemsetsParagraph{
\makeSubAnswer{}{multiphysics:problemSet3b:1a}

A spatial discretization with width \( \Delta x = 1 / N \)
is illustrated in \cref{fig:ps3bDiscretization:ps3bDiscretizationFig1}

\imageFigure{../figures/ece1254-multiphysics/ps3bDiscretizationFig1}{Discretization interval.}{fig:ps3bDiscretization:ps3bDiscretizationFig1}{0.2}

Let

\begin{equation}\label{eqn:multiphysicsProblemSet3b:120}
\begin{aligned}
x^k &= (k -1)\Delta x , \quad k \in \{ 1, \cdots, N + 1 \} \\
\tau^{k}(t) &= T( x^k, t ) \\
h^k &= h( x^k ) \\
i^k(t) &= \frac{ \tau^k - \tau^{k-1}}{\Delta x}
\end{aligned}
\end{equation}

The temperature has an electric circuit equivalent to voltage, and the heat flow \( i^k \) has the structure of an electric circuit current \( I = \Delta V/R \), through a resistance \( \Delta x \).

In the interior the discretization of the heat equation takes the form of a KCL equation

\begin{equation}\label{eqn:multiphysicsProblemSet3b:640}
0 =
\lr{\Delta x}^2 \ddt{\tau^k}
-\lr{ i^k - i^{k-1} }
+ \lr{\Delta x}^2 \alpha \tau^k - \lr{\Delta x}^2 h^k u(t).
\end{equation}

This is illustrated in \cref{fig:ps3bInteriorNode:ps3bInteriorNodeFig1}.

\imageFigure{../figures/ece1254-multiphysics/ps3bInteriorNodeFig1}{Equivalent circuit for heat equation.}{fig:ps3bInteriorNode:ps3bInteriorNodeFig1}{0.3}

The initial node where \( i^1 = 0 \), or \( \tau^2 = \tau^1 \), can be modeled with a zero voltage source, replacing the resistance.  This is illustrated in \cref{fig:ps3bInitialNode:ps3bInitialNodeFig2}.

\imageFigure{../figures/ece1254-multiphysics/ps3bInitialNodeFig2}{Initial node.}{fig:ps3bInitialNode:ps3bInitialNodeFig2}{0.3}

Similarly, the terminal node boundary value constraint \( i^{N+1} = 0 \), or \( \tau^N = \tau^{N+1} \) can also be modeled by a zero voltage source, as illustrated in \cref{fig:ps3bTerminalNode:ps3bTerminalNodeFig3}.

\imageFigure{../figures/ece1254-multiphysics/ps3bTerminalNodeFig3}{Terminal node.}{fig:ps3bTerminalNode:ps3bTerminalNodeFig3}{0.3}

The pattern of \( \BG \) is illustrated well by the \( N = 6 \) case

\begin{subequations}
\begin{equation}\label{eqn:multiphysicsProblemSet3b:660}
\BG =
\begin{bmatrix}
    \alpha \lr{ \Delta x}^2 &       0 &       0 &       0 &       0 &       0 &       0 & -1 &       0 \\
         0 &  N + \alpha \lr{ \Delta x}^2 & -N &       0 &       0 &       0 &       0 &  1 &       0 \\
         0 & -N & 2 N + \alpha \lr{ \Delta x}^2 & -N &       0 &       0 &       0 &       0 &       0 \\
         0 &       0 & -N & 2 N + \alpha \lr{ \Delta x}^2 & -N &       0 &       0 &       0 &       0 \\
         0 &       0 &       0 & -N & 2 N + \alpha \lr{ \Delta x}^2 & -N &       0 &       0 &       0 \\
         0 &       0 &       0 &       0 & -N &  N + \alpha \lr{ \Delta x}^2 &       0 &       0 & -1 \\
         0 &       0 &       0 &       0 &       0 &       0 &  \alpha \lr{ \Delta x}^2 &       0 &  1 \\
    1 & -1 &       0 &       0 &       0 &       0 &       0 &       0 &       0 \\
         0 &       0 &       0 &       0 &       0 &  1 & -1 &       0 &       0 \\
\end{bmatrix}
\end{equation}
\begin{equation}\label{eqn:multiphysicsProblemSet3b:680}
\Bx =
\begin{bmatrix}
\tau^1 \\
\tau^2 \\
\vdots \\
\tau^{N+1} \\
i^1 \\
i^N
\end{bmatrix}
\end{equation}
\begin{equation}\label{eqn:multiphysicsProblemSet3b:700}
\BC = \lr{\Delta x}^2
\begin{bmatrix}
\BI_{N+1} & \Bzero_2 \\
\Bzero_2 & \Bzero_2
\end{bmatrix}
\end{equation}
\begin{equation}\label{eqn:multiphysicsProblemSet3b:720}
\BB \Bu
=
\lr{\Delta x}^2 u(t)
\begin{bmatrix}
h^1 \\
h^2 \\
\vdots \\
h^{N+1} \\
0 \\
0 \\
\end{bmatrix}
\end{equation}
\end{subequations}

\makeSubAnswer{}{multiphysics:problemSet3b:1b}

\begin{itemize}
   \item [\textbf{Case 1}]
\begin{equation}\label{eqn:multiphysicsProblemSet3b:380}
\begin{aligned}
\BB &= h(0)
%\begin{bmatrix}
%1 & 0  & \hdots & 0 \\
%0 & 0  & \hdots & 0 \\
%\vdots  & 0 & \ddots & 0 \\
%0 & 0  & \hdots & 0 \\
%\end{bmatrix} \\
\begin{bmatrix}
1 & 0 \\
0        & \Bzero_{N+2}
\end{bmatrix} \\
\BL^T &=
\kbordermatrix{
 & 1 & 2 & \cdots & N & N+1 & & \\
 & 0 & 0 & \cdots & 0 & 1 & 0 & 0
}
\end{aligned}
\end{equation}
   \item [\textbf{Case 2}]
\begin{equation}\label{eqn:multiphysicsProblemSet3b:420}
\begin{aligned}
\BB &= h(0)
%\begin{bmatrix}
%1 & 0  & \hdots & 0 \\
%0 & 0  & \hdots & 0 \\
%\vdots  & 0 & \ddots & 0 \\
%0 & 0  & \hdots & 0 \\
%\end{bmatrix} \\
\begin{bmatrix}
1 & 0 \\
0        & \Bzero_{N+2}
\end{bmatrix} \\
\BL^T &=
\inv{N+1}
%\begin{bmatrix}
%1 & 1 & \cdots & 1 & 1 & 0 & 0
%\end{bmatrix}
\kbordermatrix{
 & 1 & 2 & \cdots & N & N+1 & & \\
 & 1 & 1 & \cdots & 1 & 1 & 0 & 0
}
\end{aligned}
\end{equation}
   \item [\textbf{Case 3}]
\begin{equation}\label{eqn:multiphysicsProblemSet3b:440}
\begin{aligned}
\BB &= h(1)
\begin{bmatrix}
\BI_{N+1} & \Bzero_2 \\
\Bzero_2 & \Bzero_2
\end{bmatrix} \\
\BL^T &=
\kbordermatrix{
  & 1 & 2 & \cdots & N & N+1 &   & \\
  & 0 & 0 & \cdots & 0 & 1   & 0 & 0
}
\end{aligned}
\end{equation}
   \item [\textbf{Case 4}]
\begin{equation}\label{eqn:multiphysicsProblemSet3b:460}
\begin{aligned}
\BB &= h(1)
\begin{bmatrix}
\BI_{N+1} & \Bzero_2 \\
\Bzero_2 & \Bzero_2
\end{bmatrix}
\\
\BL^T &=
\inv{N+1}
%\begin{bmatrix}
%1 & 1 & \cdots & 1 & 1 & 0 & 0
%\end{bmatrix}
\kbordermatrix{
 & 1 & 2 & \cdots & N & N+1 & & \\
 & 1 & 1 & \cdots & 1 & 1 & 0 & 0
}
\end{aligned}
\end{equation}
\end{itemize}


\makeSubAnswer{}{multiphysics:problemSet3b:1c}

The MLN equations for the \( N = 500 \) case were generated indirectly using
\matlabFuncPath{generateNetlist()}{ps3b:generateNetlist.m}, to create a netlist file, with the previous implementation of
\matlabFuncPath{NodalAnalysis()}{ps2a:NodalAnalysis.m} doing the grunt work.
A call to \matlabFuncPath{plotFreqPartC(500, 501, 0)}{ps3b:plotFreqPartC.m} was used to
%calls these and \matlabFuncPath{PlotFreqResp()}{ps3b:PlotFreqResp.m} to
produce the plot of \cref{fig:ps3bFreqResponsePartC:ps3bFreqResponsePartCFig1}.

\mathImageFigure{../figures/ece1254-multiphysics/ps3bFreqResponsePartCFig1.pdf}{\( N = 500 \) frequency response for case 1.}{fig:ps3bFreqResponsePartC:ps3bFreqResponsePartCFig1}{0.3}{ps3b:PlotFreqResp.m}

\makeSubAnswer{}{multiphysics:problemSet3b:1d}

\paragraph{Background}

\index{modal truncation}
It is helpful to review the modal truncation method before attempting to implement it.  The starting point is the system equations

\begin{subequations}
\begin{equation}\label{eqn:multiphysicsProblemSet3b:740}
\BG \Bx + \BC \dot{\Bx} = \BB \Bu
\end{equation}
\begin{equation}\label{eqn:multiphysicsProblemSet3b:760}
\By = \BL^\T \Bx.
\end{equation}
\end{subequations}

Assuming that \( \BC \) is invertible, \cref{eqn:multiphysicsProblemSet3b:740} can be rewritten in \textAndIndex{state space form}

\begin{equation}\label{eqn:multiphysicsProblemSet3b:780}
\dot{\Bx} = -\BC^{-1} \BG \Bx + \BC^{-1} \BB \Bu.
\end{equation}

Introduce a diagonalization of \( \BA = - \BC^{-1} \BG \), and associated change of variables

\begin{equation}\label{eqn:multiphysicsProblemSet3b:800}
\begin{aligned}
\BA &= \BV \Bwedge \BV^{-1} \\
\Bw &= \BV^{-1} \Bx \\
\Bb(t) &= \BV^{-1} \BC^{-1} \BB \Bu(t),
\end{aligned}
\end{equation}

so that

%\begin{equation}\label{eqn:multiphysicsProblemSet3b:820}
\boxedEquation{eqn:multiphysicsProblemSet3b:820}{
\begin{aligned}
\dot{\Bw} &= \Bwedge \Bw + \Bb \\
\By &= \BL^\T \BV \Bw.
\end{aligned}
}
%\end{equation}

Also observe that the frequency response can be calculated directly from this alternate set of system equations, since

\begin{equation}\label{eqn:multiphysicsProblemSet3b:840}
\BY(s) = \BL^\T \BV \lr{ s \BI - \Bwedge }^{-1} \Bb(s).
\end{equation}

\Cref{eqn:multiphysicsProblemSet3b:820} is the starting point for the modal truncation method, a strategic selection of which of the eigenvalues of \( \Bwedge \) should be either retained or ignored.  Suppose, for example, after some sort criteria, the first \( q \) eigenvalues of \( \Bwedge \) are deemed of interest.
%If the columns of \( \BV \), in eigenvalue order, are \( \setlr{\Bv_1, \Bv_2, \cdots } \), then
If \( \mytilde{\Bwedge} \) is the \( q \times q \) submatrix of \( \Bwedge \),
\( \mytilde{\Bw}, \mytilde{\Bb} \) are formed from the first \( q \) rows of \( \Bw \) and \( \Bb \) respectively,
and \( \mytilde{\BL} \) has the first \( q \) rows of \( \BV^\T \BL \), then

\begin{equation}\label{eqn:multiphysicsProblemSet3b:860}
\begin{aligned}
\dot{\mytilde{\Bw}} &= \mytilde{\Bwedge} \mytilde{\Bw} + \mytilde{\Bb}  \\
\By &\approx \mytilde{\BL}^\T \mytilde{\Bw}.
\end{aligned}
\end{equation}

\paragraph{Model modification}
\index{modal truncation}

The modeling of the zero heat flow condition using a zero voltage source is problematic for the modal truncation method, since the matrix \( \BC \) is not invertible.  Restoring the resistor \( \Delta x \) between nodes \( 1,2 \) and nodes \( N, N+1 \), and adding a (very large) parallel resistor and (very small) capacitor from the end points into ``space'' also works to model this boundary value constraint.  For the final node, this is illustrated in \cref{fig:ps3bTerminalNode:ps3bTerminalNodeFig4}.

\paragraph{Grading remark:}

On this it was noted ``Since the Re and Ce subcircuit is floating, no current is flowing. This dangling subcircuit can be removed with no effect on the solution of the rest of the system.''

\imageFigure{../figures/ece1254-multiphysics/ps3bTerminalNodeFig4}{Alternate modeling of the boundary value constraints.}{fig:ps3bTerminalNode:ps3bTerminalNodeFig4}{0.3}

Using resistor and capacitor values (somewhat arbitrarily chosen) of \( 10^8 \) and \( 10^{-8} \) respectively, results in the same frequency response, and allows for inversion of \( \BC \).  Using very large and very small values (like \( 10^{100}, 10^{-100} \) ) which are representable in double precision, but cause ill conditioning effects.  This floating resistor need not be included in the model, since \( \BG \) can be allowed to be non-invertible, but all the plots and calculations below were done with such a resistor included.

\paragraph{Model reduction results}

No differences for the frequency response of the original vs model truncation models are visible in the plots, even for \( q = 1 \), as shown in \cref{fig:ps3bFreqResponsePartCq1:ps3bFreqResponsePartCq1Fig1}.
This is likely because most of the eigenvalues of \( - \BC^{-1} \BG \) are very fast, so the smallest magnitude eigenvalue dominates.  The first few of these, sorted smallest first, are

\mathImageFigure{../figures/ece1254-multiphysics/ps3bFreqResponsePartCq1Fig1.pdf}{Overlaid plots with modal reduced and original models.}{fig:ps3bFreqResponsePartCq1:ps3bFreqResponsePartCq1Fig1}{0.3}{ps3b:plotFreqPartC.m}

\begin{equation}\label{eqn:multiphysicsProblemSet3b:880}
\lambda_i =
\begin{bmatrix}
   -10.0000 \times 10^{-3} \\
    -1.0000  \\
    -1.0000  \\
    -4.9151 \times 10^3 \\
   -19.6602 \times 10^3 \\
   -44.2348 \times 10^3 \\
   -78.6378 \times 10^3 \\
  -122.8680 \times 10^3 \\
  -176.9235 \times 10^3 \\
  -240.8023 \times 10^3 \\
  -314.5019 \times 10^3 \\
  -398.0192 \times 10^3 \\
  -491.3512 \times 10^3 \\
  -594.4940 \times 10^3 \\
  -707.4437 \times 10^3 \\
  -830.1957 \times 10^3 \\
  -962.7454 \times 10^3 \\
    -1.1051 \times 10^6 \\
    -1.2572 \times 10^6 \\
    \vdots
\end{bmatrix}
\end{equation}

There are six orders of magnitude difference between the first and fourth eigenvalues.  The complete range of absolute values of these eigenvalues are plotted in \cref{fig:ps3bEigenvalues:ps3bEigenvaluesFig1}, showing the 10 orders of magnitude difference in the eigenvalues (all real and negative).

\mathImageFigure{../figures/ece1254-multiphysics/ps3bEigenvaluesFig1.pdf}{Log-log plot of \( -\BC^{-1} \BG \) eigenvalue magnitude.}{fig:ps3bEigenvalues:ps3bEigenvaluesFig1}{0.3}{ps3b:plotFreqPartC.m}

\makeSubAnswer{}{multiphysics:problemSet3b:1e}

\begin{enumerate}
   \item The end of bar temperature for the unit step input \( u_1(t) \) is plotted in
\cref{fig:ps3bUnitDriverOutput:ps3bUnitDriverOutputFig1}
.  There was no visible difference between any of the \( q \) reduced models and the full model, with errors
of order of \( 10^{-6}, 10^{-7} \)
for \( q = 1, 50 \) respectively as plotted in
\cref{fig:ps3bUnitDriverError:ps3bUnitDriverErrorFig1}
.

\mathImageFigure{../figures/ece1254-multiphysics/ps3bUnitDriverOutputmodalFig1.pdf}{Unit step response at the end of the bar.}{fig:ps3bUnitDriverOutput:ps3bUnitDriverOutputFig1}{0.3}{ps3b:displayOutputErrorAndCpuTimes.m}
\mathImageFigure{../figures/ece1254-multiphysics/ps3bUnitDriverErrormodalFig1.pdf}{Error for unit step for \( q = 1,50 \).}{fig:ps3bUnitDriverError:ps3bUnitDriverErrorFig1}{0.3}{ps3b:displayOutputErrorAndCpuTimes.m}

The CPU times for all the reduced order models were small and comparable.  These are plotted in \cref{fig:ps3bUnitDriverCpuTimes:ps3bUnitDriverCpuTimesFig1} for \( 2000 \) time step intervals.

\mathImageFigure{../figures/ece1254-multiphysics/ps3bUnitDriverCpuTimesmodalFig1.pdf}{CPU times for unit step input.}{fig:ps3bUnitDriverCpuTimes:ps3bUnitDriverCpuTimesFig1}{0.3}{ps3b:displayOutputErrorAndCpuTimes.m}

   \item The end of bar temperature for the sinusoidal input \( u_2(t) \) is plotted in
\cref{fig:ps3bSineDriverOutput:ps3bSineDriverOutputFig1}
.  Again, there was no visible difference between any of the \( q \) reduced models and the full model.  As with the unit step
the errors were also
of order of \( 10^{-6}, 10^{-7} \)
for \( q = 1, 50 \) respectively as plotted in
\cref{fig:ps3bSineDriverError:ps3bSineDriverErrorFig1}.

\mathImageFigure{../figures/ece1254-multiphysics/ps3bSineDriverOutputmodalFig1.pdf}{Response for sine input at the end of the bar.}{fig:ps3bSineDriverOutput:ps3bSineDriverOutputFig1}{0.3}{ps3b:displayOutputErrorAndCpuTimes.m}
\mathImageFigure{../figures/ece1254-multiphysics/ps3bSineDriverErrormodalFig1.pdf}{Error for sine input for \( q = 1, 50 \).}{fig:ps3bSineDriverError:ps3bSineDriverErrorFig1}{0.3}{ps3b:displayOutputErrorAndCpuTimes.m}

The CPU times for all the reduced order models were also small and comparable, and are plotted in \cref{fig:ps3bSineDriverCpuTimes:ps3bSineDriverCpuTimesFig1} also for \( 2000 \) time step intervals.

\mathImageFigure{../figures/ece1254-multiphysics/ps3bSineDriverCpuTimesmodalFig1.pdf}{CPU times for sinusoidal input.}{fig:ps3bSineDriverCpuTimes:ps3bSineDriverCpuTimesFig1}{0.3}{ps3b:displayOutputErrorAndCpuTimes.m}
\end{enumerate}

\paragraph{On the \( q \) and time step selection}
\index{time step}

Part \ref{multiphysics:problemSet3b:1d} asked for the selection of a ``reasonably small'' order \( q \) for the reduced model.  Based on the frequency response, and the fact that the slowest eigenvalue was two orders of magnitude smaller than the next two, it appeared that \( q = 1 \) was sufficient.  This choice is further justified by looking at the error, which was \( O(10^{-6}) \) even for \( q = 1 \).

With the time required for any of the reduced model systems being so small, a fairly arbitrary, but large, number of time steps (2000) was picked.

\makeSubAnswer{}{multiphysics:problemSet3b:1f}

With the prima reduction \( q = 2 \) offers an error less than the \( q = 50 \) error of the modal reduction.  This is illustrated in the plots of
\cref{fig:ps3bUnitDriverErrorprima:ps3bUnitDriverErrorprimaFig1},
\cref{fig:ps3bSineDriverErrorprima:ps3bSineDriverErrorprimaFig1}.

\mathImageFigure{../figures/ece1254-multiphysics/ps3bSineDriverErrorprimaFig1.pdf}{Prima reduction errors for \( q=1,2\), unit step source.}{fig:ps3bSineDriverErrorprima:ps3bSineDriverErrorprimaFig1}{0.3}{ps3b:displayOutputErrorAndCpuTimes.m}

\mathImageFigure{../figures/ece1254-multiphysics/ps3bUnitDriverErrorprimaFig1.pdf}{Prima reduction error for \( q = 1,2\), sine source.}{fig:ps3bUnitDriverErrorprima:ps3bUnitDriverErrorprimaFig1}{0.3}{ps3b:displayOutputErrorAndCpuTimes.m}

\makeSubAnswer{}{multiphysics:problemSet3b:1g}

CPU times for the prima method look comparable at a glance to those of the modal reductions, which is not surprising.  The prima CPU times are plotted in \cref{fig:ps3bSineDriverCpuTimesprima:ps3bSineDriverCpuTimesprimaFig1}, and \cref{fig:ps3bUnitDriverCpuTimesprima:ps3bUnitDriverCpuTimesprimaFig1} respectively.

\mathImageFigure{../figures/ece1254-multiphysics/ps3bSineDriverCpuTimesprimaFig1.pdf}{CPU times for sine input, prima method.}{fig:ps3bSineDriverCpuTimesprima:ps3bSineDriverCpuTimesprimaFig1}{0.3}{ps3b:displayOutputErrorAndCpuTimes.m}
\mathImageFigure{../figures/ece1254-multiphysics/ps3bUnitDriverCpuTimesprimaFig1.pdf}{CPU times for unit step input, prima method.}{fig:ps3bUnitDriverCpuTimesprima:ps3bUnitDriverCpuTimesprimaFig1}{0.3}{ps3b:displayOutputErrorAndCpuTimes.m}

\makeSubAnswer{}{multiphysics:problemSet3b:1h}

%\captionedTable{Prima vs Modal}{tab:multiphysicsProblemSet3b:900}{
%\begin{tabular}{|l||l|l|l|l|l|l|l|l|}
%\hline
%  & \multicolumn{4}{ c| }{Modal reduction} & \multicolumn{4}{ c| }{Prima reduction} \\ \hline
%  & \multicolumn{2}{ c| }{Sine} & \multicolumn{2}{ c| }{Unit step } & \multicolumn{2}{ c| }{Sine} & \multicolumn{2}{ c| }{Unit step } \\ \hline
%  & error & cpu & error & cpu & error & cpu & error & cpu \\ \hline \hline
%1 & \( 6.71 \times 10^{-7} \) & \( 1.73 \times 10^{-2} \pm 2.54 \times 10^{-4} \) & \( 6.69 \times 10^{-7} \) & \( 3.65 \times 10^{-2} \pm 1.96 \times 10^{-4} \) & \( 1.10 \times 10^{-6} \) & \( 3.71 \times 10^{-2} \pm 8.76 \times 10^{-5} \) & \( 8.10 \times 10^{-7} \) & \( 3.82 \times 10^{-2} \pm 3.70 \times 10^{-4} \) \\ \hline
%2 & \( 6.71 \times 10^{-7} \) & \( 2.38 \times 10^{-2} \pm 3.05 \times 10^{-4} \) & \( 6.69 \times 10^{-7} \) & \( 2.36 \times 10^{-2} \pm 2.19 \times 10^{-4} \) & \( 3.62 \times 10^{-9} \) & \( 2.45 \times 10^{-2} \pm 1.37 \times 10^{-4} \) & \( 1.37 \times 10^{-9} \) & \( 2.48 \times 10^{-2} \pm 3.82 \times 10^{-4} \) \\ \hline
%4 & \( 1.61 \times 10^{-7} \) & \( 2.38 \times 10^{-2} \pm 1.59 \times 10^{-4} \) & \( 2.02 \times 10^{-7} \) & \( 2.37 \times 10^{-2} \pm 1.67 \times 10^{-4} \) & \( 3.62 \times 10^{-9} \) & \( 2.46 \times 10^{-2} \pm 5.80 \times 10^{-5} \) & \( 1.33 \times 10^{-9} \) & \( 2.50 \times 10^{-2} \pm 5.62 \times 10^{-4} \) \\ \hline
%10 & \( 3.79 \times 10^{-8} \) & \( 2.47 \times 10^{-2} \pm 5.88 \times 10^{-5} \) & \( 6.52 \times 10^{-8} \) & \( 2.51 \times 10^{-2} \pm 9.79 \times 10^{-4} \) & \( 3.78 \times 10^{-9} \) & \( 2.56 \times 10^{-2} \pm 4.43 \times 10^{-4} \) & \( 1.58 \times 10^{-9} \) & \( 2.65 \times 10^{-2} \pm 5.06 \times 10^{-4} \) \\ \hline
%50 & \( 3.59 \times 10^{-8} \) & \( 3.87 \times 10^{-2} \pm 4.91 \times 10^{-5} \) & \( 5.82 \times 10^{-8} \) & \( 3.85 \times 10^{-2} \pm 1.82 \times 10^{-4} \) & \( 3.06 \times 10^{-8} \) & \( 6.10 \times 10^{-2} \pm 3.08 \times 10^{-4} \) & \( 4.93 \times 10^{-8} \) & \( 6.17 \times 10^{-2} \pm 9.47 \times 10^{-4} \) \\ \hline
%\end{tabular}
%}

\Cref{tab:multiphysicsProblemSet3b:920} and \cref{tab:multiphysicsProblemSet3b:940} summarize the maximum absolute errors compared to the full model and CPU costs for the modal and prima reductions of various orders of \( q \).

\captionedTable{Modal reduction summary.}{tab:multiphysicsProblemSet3b:920}{
\begin{tabular}{|l||l|l|l|l|}
\hline
   & \multicolumn{2}{ c| }{Sine}                                                           & \multicolumn{2}{ c| }{Unit step }                                                      \\ \hline
   & \( \text{error} \times 10^{-7} \) & \( \text{CPU} \times 10^{-2} \)                   & \( \text{error} \times 10^{-7} \) & \( \text{CPU} \times 10^{-2} \)                    \\ \hline \hline
1  & \( 6.71 \)                        & \( 1.73                \pm 0.03                \) & \( 6.69                \)         & \( 3.65                \pm 0.02                \)  \\ \hline
2  & \( 6.71 \)                        & \( 2.38                \pm 0.03                \) & \( 6.69                \)         & \( 2.36                \pm 0.02                \)  \\ \hline
4  & \( 1.61 \)                        & \( 2.38                \pm 0.01                \) & \( 2.02                \)         & \( 2.37                \pm 0.02                \)  \\ \hline
10 & \( 0.379 \)                       & \( 2.47                \pm 0.006               \) & \( 0.652               \)         & \( 2.51               \pm 0.10                \)  \\ \hline
50 & \( 0.359 \)                       & \( 3.87                \pm 0.005               \) & \( 0.582               \)         & \( 3.85               \pm 0.02                \)  \\ \hline
\end{tabular}
}

\captionedTable{Prima reduction summary.}{tab:multiphysicsProblemSet3b:940}{
\begin{tabular}{|l||l|l|l|l|}
\hline
   & \multicolumn{2}{ c| }{Sine}                                                           & \multicolumn{2}{ c| }{Unit step }                                                     \\ \hline
   & \( \text{error} \times 10^{-7} \) & \( \text{CPU} \times 10^{-2} \)                   & \( \text{error} \times 10^{-7} \) & \( \text{CPU} \times 10^{-2} \)                   \\ \hline \hline
1  &  \( 11.0 \)                       & \( 3.71                \pm 0.009               \) & \( 8.10                \)         & \( 3.82                \pm 0.04                \) \\ \hline
2  &  \( 0.0362 \)                     & \( 2.45                \pm 0.01                \) & \( 0.0137                \)       & \( 2.48                \pm 0.04                \) \\ \hline
4  &  \( 0.0362 \)                     & \( 2.46                \pm 0.006               \) & \( 0.0133                \)       & \( 2.50                \pm 0.06                \) \\ \hline
10 &  \( 0.0378 \)                     & \( 2.56                \pm 0.04                \) & \( 0.0158                \)       & \( 2.65                \pm 0.05                \) \\ \hline
50 &  \( 0.306 \)                      & \( 6.10                \pm 0.03                \) & \( 0.493                \)        & \( 6.17                \pm 0.09                \) \\ \hline
\end{tabular}
}

Some observations

\begin{itemize}
\item
As seen in the error plots above, the prima model of order \( q = 1 \) has significantly more error than that of \( q = 2 \), whereas the modal reduction error was similar for \( q = 1,2 \).
\item
There is not much benefit past \( q = 2 \) for the prima reduction, and in fact as \( q \) is increased it appears that numerical errors start degrading the correctness of the prima reduced models.
\item At \( q = 50 \) the prima reduction results in approximately the same error as modal reduction of the same order.
\item Within an order of magnitude the modal and prima reductions of any order \( q \le 50 \) have the same CPU cost.
\end{itemize}

\paragraph{Grading remark:}

The \( q = 50 \) prima error result noted above was flagged in a grading comment as ``This is a bit strange. It may be due to the non-standard way you implemented the boundary conditions.''  This is probably related to the other grading remark, but I'm not sure what the ``standard'' way to model the boundary condition would be.
}
}
