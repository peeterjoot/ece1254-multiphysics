%
% Copyright © 2014 Peeter Joot.  All Rights Reserved.
% Licenced as described in the file LICENSE under the root directory of this GIT repository.
%

\makedefinition{Positive (negative) definite.}{dfn:multiphysicsL7:40}{
   A matrix \( M \) is positive (negative) definite, denoted \( M > 0 (<0) \) if \( \By^\T M \By > 0 (<0), \quad \forall \By \).

   If a matrix is neither positive, nor negative definite, it is called indefinite.

   When zero equality is possible \( \By^\T M \By \ge 0 (\le 0) \), the matrix is positive (negative) semi-definite.
}
\index{indefinite matrix}
\index{positive definite}
\index{negative definite}
\index{positive semi-definite}
\index{negative semi-definite}

\maketheorem{Positive (negative) definite.}{thm:multiphysicsL7:60}{
   A symmetric matrix \( M > 0 (<0)\) iff \( \lambda_i > 0 (<0)\) for all eigenvalues \( \lambda_i \), or is indefinite iff its eigenvalues \( \lambda_i \) are of mixed sign.
}

\index{mean value theorem}
\maketheorem{Mean value theorem.}{thm:multiphysicsL10:140}{

For a continuous and differentiable function \( f(x) \), the difference can be expressed in terms of the derivative at an intermediate point

\begin{equation*}
f(x_2) - f(x_1)
= \evalbar{ \PD{x}{f} }{\tilde{x}} \lr{ x_2 - x_1 }
\end{equation*}

where \( \tilde{x} \in [x_1, x_2] \).

This is illustrated (roughly) in \cref{fig:lecture10:lecture10Fig2}.

\imageFigure{../figures/ece1254-multiphysics/lecture10Fig2}{Mean value theorem illustrated.}{fig:lecture10:lecture10Fig2}{0.2}
}


