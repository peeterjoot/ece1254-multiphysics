%
% Copyright © 2014 Peeter Joot.  All Rights Reserved.
% Licenced as described in the file LICENSE under the root directory of this GIT repository.
%
To find \( X_n \) evaluate the sum
%
\begin{equation}\label{eqn:discreteFourier:80}
\begin{aligned}
\sum_{k = -N}^N &x(t_k) e^{-j m \omega_0 t_k} \\
&=
\sum_{k = -N}^N
\lr{
\sum_{n = -N}^N X_n e^{ j n \omega_0 t_k}
}
e^{-j m \omega_0 t_k} \\
&=
\sum_{n = -N}^N X_n
\sum_{k = -N}^N
e^{ j (n -m )\omega_0 t_k}
\end{aligned}
\end{equation}
%
This interior sum has the value \( 2 N + 1 \) when \( n = m \).  For \( n \ne m \), and
\( a = e^{j (n -m ) \frac{2 \pi}{2 N + 1}} \), this is
%
\begin{equation}\label{eqn:discreteFourier:100}
\begin{aligned}
\sum_{k &= -N}^N
e^{ j (n -m )\omega_0 t_k}
\\ &=
\sum_{k = -N}^N
e^{ j (n -m )\omega_0 \frac{T k}{2 N + 1}}
\\ &=
\sum_{k= -N}^N a^k
\\ &=
a^{-N} \sum_{k= -N}^N a^{k+ N}
\\ &=
a^{-N} \sum_{r= 0}^{2 N} a^{r}
\\ &=
a^{-N} \frac{a^{2 N + 1} - 1}{a - 1}.
\end{aligned}
\end{equation}
%
Since \( a^{2 N + 1} = e^{2 \pi j (n - m)} = 1 \), this sum is zero when \( n \ne m \).  This means that
%
\begin{equation}\label{eqn:discreteFourier:120}
\sum_{k = -N}^N
e^{ j (n -m )\omega_0 t_k} = (2 N + 1) \delta_{n,m}.
\end{equation}
%
Substitution back into \cref{eqn:discreteFourier:80} proves the Fourier inversion relation \cref{eqn:discreteFourier:60}.
%
%which provides the desired Fourier inversion relation
%which provides the desired Fourier inversion relation
%
%%\begin{equation}\label{eqn:discreteFourier:140}
%\boxedEquation{eqn:discreteFourier:140}{
%X_m = \inv{2 N + 1} \sum_{k = -N}^N x(t_k) e^{-j m \omega_0 t_k}.
%}
%%\end{equation}
