%
% Copyright � 2014 Peeter Joot.  All Rights Reserved.
% Licenced as described in the file LICENSE under the root directory of this GIT repository.
%
%\input{../blogpost.tex}
%
%\renewcommand{\basename}{multiphysicsL2}
%\renewcommand{\dirname}{notes/ece1254/}
%\newcommand{\keywords}{ECE1254H}
%\input{../peeter_prologue_print2.tex}
%
%\usepackage{listing}
%\usepackage{algorithm}
%\usepackage{algorithmic}
%
%\beginArtNoToc
%\generatetitle{ECE1254H Modeling of Multiphysics Systems.  Lecture 2: Assembling system equations automatically.  Taught by Prof.\ Piero Triverio}
%\section{Assembling system equations automatically}
\label{chap:multiphysicsL2}

%\section{Disclaimer}
%
%Peeter's lecture notes from class.  These may be incoherent and rough.

\section{Assembling system equations automatically.  Node/branch method}
\index{node branch method}

Consider the sample circuit of \cref{fig:lecture2:lecture2Fig1}.

%\imageFigure{../../figures/ece1254/lecture2Fig1}{Sample resistive circuit}{fig:lecture2:lecture2Fig1}{0.3}
\imageFigure{../../figures/ece1254/1254-sample-resistive-circuit.pdf}{Sample resistive circuit.}{fig:lecture2:lecture2Fig1}{0.3}


\paragraph{Step 1.  Choose unknowns:}  For this problem, take

\begin{itemize}
\item
node voltages: \(V_1, V_2, V_3, V_4\)
\item
branch currents: \(i_{\textrm{A}}, i_{\textrm{B}}, i_{\textrm{C}}, i_{\textrm{D}}, i_{\textrm{E}}\)
\end{itemize}

No additional labels are required for the source current sources.
A reference node is always introduced, given the node number zero.

For a circuit with \(N\)
nodes, and
\(B\) resistors, there will be
\(N-1\)
unknown node voltages \index{node voltage} and \(B\) unknown branch currents \index{branch current}, for a total number of
\(N - 1 + B\) unknowns.

\paragraph{Step 2.  Conservation equations:}  KCL

\begin{itemize}
   \item 0: \(i_{\textrm{A}} + i_{\textrm{E}} - i_{\textrm{D}} = 0\)
   \item 1: \(-i_{\textrm{A}} + i_{\textrm{B}} + i_{\textrm{S},\textrm{A}} = 0\)
   \item 2: \(-i_{\textrm{B}} + i_{\textrm{S},\textrm{B}} - i_{\textrm{E}} + i_{\textrm{S},\textrm{C}} = 0\)
   \item 3: \(i_{\textrm{C}} - i_{\textrm{S},\textrm{C}} = 0\)
   \item 4: \(-i_{\textrm{S},\textrm{A}} - i_{\textrm{S},\textrm{B}} + i_{\textrm{D}} - i_{\textrm{C}} = 0\)
\end{itemize}

Grouping unknown currents, this is

\begin{itemize}
   \item 0: \(i_{\textrm{A}} + i_{\textrm{E}} - i_{\textrm{D}} = 0\)
   \item 1: \(-i_{\textrm{A}} + i_{\textrm{B}} = -i_{\textrm{S},\textrm{A}}\)
   \item 2: \(-i_{\textrm{B}} -i_{\textrm{E}} = -i_{\textrm{S},\textrm{B}} -i_{\textrm{S},\textrm{C}}\)
   \item 3: \(i_{\textrm{C}} = i_{\textrm{S},\textrm{C}}\)
   \item 4: \(i_{\textrm{D}} - i_{\textrm{C}} = i_{\textrm{S},\textrm{A}} + i_{\textrm{S},\textrm{B}}\)
\end{itemize}

Note that one of these equations is redundant (sum 1-4).  In a circuit with
\(N\)
nodes, are are at most \(N-1\) independent KCLs.  In matrix form

\begin{equation}\label{eqn:multiphysicsL2:20}
\begin{bmatrix}
   -1 & 1 & 0 & 0 & 0 \\
   0 & -1 & 0 & 0 & -1 \\
   0 & 0 & 1 & 0 & 0 \\
   0 & 0 & -1 & 1 & 0
\end{bmatrix}
\begin{bmatrix}
   i_{\textrm{A}} \\
   i_{\textrm{B}} \\
   i_{\textrm{C}} \\
   i_{\textrm{D}} \\
   i_{\textrm{E}} \\
\end{bmatrix}
=
\begin{bmatrix}
   -i_{\textrm{S},\textrm{A}} \\
   -i_{\textrm{S},\textrm{B}} -i_{\textrm{S},\textrm{C}} \\
   i_{\textrm{S},\textrm{C}} \\
   i_{\textrm{S},\textrm{A}} + i_{\textrm{S},\textrm{B}}
\end{bmatrix}
\end{equation}

This first matrix of ones and minus ones is called the \textAndIndex{incidence matrix}
\(A\).  This matrix has
\(B\) columns and
\(N-1\) rows.  The matrix of known currents is called
\(\BI_{\textrm{S}}\), and the branch currents called
\(\BI_{\textrm{B}}\).  That is

\begin{equation}\label{eqn:multiphysicsL2:40}
   A \BI_{\textrm{B}} = \BI_{\textrm{S}}.
\end{equation}

Observe that there is a plus and minus one in all columns except for those columns impacted by the neglect of the reference node current conservation equation.

\paragraph{Algorithm for filling \(A\)}

In the input file, to describe a resistor of \cref{fig:lecture2:lecture2Fig2}, you'll have a line of the form

\begin{center}
\textbf{Rlabel \(n_1\) \(n_2\) value}
\end{center}

\imageFigure{../../figures/ece1254/lecture2Fig2}{Resistor node convention.}{fig:lecture2:lecture2Fig2}{0.1}

The algorithm to process resistor lines is

\index{stamp!resistor}
\makealgorithm{Resistor line handling.}{alg:multiphysicsL2:1}{
\STATE \(A \leftarrow 0\)
\STATE \(ic \leftarrow 0\)
\FORALL{resistor lines}
\STATE \(ic \leftarrow ic + 1\), adding one column to \(A\)
\IF{\(n_1 != 0\)}
\STATE \(A(n_1, ic) \leftarrow +1\)
\ENDIF
\IF{\(n_2 != 0\)}
\STATE \(A(n_2, ic) \leftarrow -1\)
\ENDIF
\ENDFOR
}

\paragraph{Algorithm for filling \(\BI_{\textrm{S}}\)}

Current sources, as in \cref{fig:lecture2:lecture2Fig3}, a line will have the specification

\begin{center}
\textbf{Ilabel \(n_1\) \(n_2\) value}
\end{center}

\imageFigure{../../figures/ece1254/lecture2Fig3}{Current source conventions.}{fig:lecture2:lecture2Fig3}{0.1}

\index{stamp!current}
\makealgorithm{Current line handling.}{alg:multiphysicsL2:2}{
\STATE \(\BI_{\textrm{S}} = \text{zeros}(N-1, 1)\)
\FORALL{current lines}
\STATE \(\BI_{\textrm{S}}(n_1) \leftarrow \BI_{\textrm{S}}(n_1) - \text{value}\)
\STATE \(\BI_{\textrm{S}}(n_2) \leftarrow \BI_{\textrm{S}}(n_2) + \text{value}\)
\ENDFOR
}

\paragraph{Step 3.  Constitutive equations:}

\begin{equation}\label{eqn:multiphysicsL2:60}
\begin{bmatrix}
i_{\textrm{A}} \\
i_{\textrm{B}} \\
i_{\textrm{C}} \\
i_{\textrm{D}} \\
i_{\textrm{E}}
\end{bmatrix}
=
\begin{bmatrix}
1/R_{\textrm{A}} & & & & \\
      & 1/R_{\textrm{B}} & & & & \\
      & & 1/R_{\textrm{C}} & & & \\
      & & & 1/R_{\textrm{D}} & & \\
      & & & & & 1/R_{\textrm{E}}
\end{bmatrix}
\begin{bmatrix}
v_{\textrm{A}} \\
v_{\textrm{B}} \\
v_{\textrm{C}} \\
v_{\textrm{D}} \\
v_{\textrm{E}}
\end{bmatrix}
\end{equation}

Or

\begin{equation}\label{eqn:multiphysicsL2:80}
   \BI_{\textrm{B}} = \alpha \BV_{\textrm{B}},
\end{equation}

where \(\BV_{\textrm{B}}\) are the branch voltages, not unknowns of interest directly.  That can be written

\begin{equation}\label{eqn:multiphysicsL2:100}
\begin{bmatrix}
v_{\textrm{A}} \\
v_{\textrm{B}} \\
v_{\textrm{C}} \\
v_{\textrm{D}} \\
v_{\textrm{E}}
\end{bmatrix}
=
\begin{bmatrix}
   -1 & & &  \\
   1 & -1 & & \\
   & & 1 & -1 \\
   & &  & 1 \\
   & -1 & &
\end{bmatrix}
\begin{bmatrix}
v_1 \\
v_2 \\
v_3 \\
v_4
\end{bmatrix}
\end{equation}

Observe that this %\(\alpha\)  %FIXME: label
is the transpose of \(A\), so

\begin{equation}\label{eqn:multiphysicsL2:120}
\BV_{\textrm{B}} = A^\T \BV_{\textrm{N}}.
\end{equation}

\paragraph{Solving}

\begin{itemize}
\item
KCLs: \(A \BI_{\textrm{B}} = \BI_{\textrm{S}}\)
\item
constitutive: \(\BI_{\textrm{B}} = \alpha \BV_{\textrm{B}} \implies \BI_{\textrm{B}} = \alpha A^\T \BV_{\textrm{N}}\)
\item
branch and node voltages: \(\BV_{\textrm{B}} = A^\T \BV_{\textrm{N}}\)
\end{itemize}

In block matrix form, this is

\begin{equation}\label{eqn:multiphysicsL2:140}
\begin{bmatrix}
   A & 0 \\
   I & -\alpha A^\T
\end{bmatrix}
\begin{bmatrix}
   \BI_{\textrm{B}} \\
   \BV_{\textrm{N}}
\end{bmatrix}
=
\begin{bmatrix}
   \BI_{\textrm{S}} \\
   0
\end{bmatrix}.
\end{equation}

Is it square?  This can be observing to be the case after noting that there are

\begin{itemize}
\item \(N-1\)
rows in \(A\)
, and \(B\) columns.
\item \(B\)
rows in \(I\).
\item \(N-1\) columns.
\end{itemize}

%\EndArticle
%\EndNoBibArticle
