%
% Copyright © 2014 Peeter Joot.  All Rights Reserved.
% Licenced as described in the file LICENSE under the root directory of this GIT repository.
%
%
\section{Singular Jacobians}
\index{Jacobian!singular}
%
(mostly on slides)
%
There is the possibility of singular Jacobians to consider.  FIXME: not sure how this system represented that.  Look on slides.
%
%\cref{fig:lecture13:lecture13Fig1}.
%\imageFigure{../figures/ece1254-multiphysics/lecture13Fig1}{Diode system that results in singular Jacobian}{fig:lecture13:lecture13Fig1}{0.2}
\imageFigure{../figures/ece1254-multiphysics/diode-circuit-voltage-source-singular-jacobian.pdf}{Diode system that results in singular Jacobian.}{fig:lecture13:lecture13Fig1}{0.2}
%
\begin{equation}\label{eqn:multiphysicsL13:20}
\tilde{f}(v(\lambda), \lambda) = i(v) - \inv{R}( v - \lambda V_{\textrm{s}} ) = 0.
\end{equation}
%
An alternate continuation scheme uses
%
\begin{equation}\label{eqn:multiphysicsL13:40}
\tilde{F}(\Bx(\lambda), \lambda) = \lambda F(\Bx(\lambda)) + (1-\lambda) \Bx(\lambda).
\end{equation}
%
This scheme has
%
\begin{subequations}
\begin{equation}\label{eqn:multiphysicsL13:60}
\tilde{F}(\Bx(0), 0) = 0
\end{equation}
\begin{equation}\label{eqn:multiphysicsL13:80}
\tilde{F}(\Bx(1), 1) = F(\Bx(1)),
\end{equation}
\end{subequations}
%
and for one variable, easy to compute Jacobian at the origin, or the original Jacobian at \( \lambda = 1 \)
%
\begin{subequations}
\begin{equation}\label{eqn:multiphysicsL13:100}
\PD{x}{\tilde{F}}(x(0), 0) = I
\end{equation}
\begin{equation}\label{eqn:multiphysicsL13:120}
\PD{x}{\tilde{F}}(x(1), 1) = \PD{x}{F}(x(1))
\end{equation}
\end{subequations}
%
%
