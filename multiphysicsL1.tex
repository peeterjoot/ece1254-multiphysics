%
% Copyright � 2014 Peeter Joot.  All Rights Reserved.
% Licenced as described in the file LICENSE under the root directory of this GIT repository.
%
%\input{../blogpost.tex}
%\renewcommand{\basename}{multiphysicsL1}
%\renewcommand{\dirname}{notes/ece1254/}
%\newcommand{\keywords}{ECE1254H}
%\input{../peeter_prologue_print2.tex}
%
%\beginArtNoToc
%\generatetitle{ECE1254H Modeling of Multiphysics Systems.  Lecture 1: Analogies to circuit systems.  Taught by Prof.\ Piero Triverio}
%\section{Analogies to circuit systems}
\label{chap:multiphysicsL1}

%\section{Disclaimer}
%
%Peeter's lecture notes from class.  May not be entirely coherent.

\section{In slides}

A review of systematic nodal analysis for a basic resistive circuit was outlined in slides, with a subsequent attempt to show how many similar linear systems can be modeled as circuits so that the same toolbox can be applied.  This included blood flow through a body (and blood flow to the brain), a model of antenna interference in a portable phone, heat conduction in a one dimensional conductor under a heat lamp, and a few other systems.

This discussion reminded me of the joke where the farmer, the butcher and the physicist are all invited to talk at a beef convention.  After meaningful and appropriate talks by the farmer and the butcher, the physicist gets his chance, and proceeds with ``We begin by modeling the cow as a sphere, ...''.  The ECE equivalent of that appears to be a Kirchhoff circuit problem.

\section{Mechanical structures example}
\index{mechanical structures}

Continuing the application of circuits like linear systems to other systems, consider a truss system as illustrated in \cref{fig:lecture1:lecture1Fig1}, or in the simpler similar system of \cref{fig:lecture1:lecture1Fig2}.

\imageFigure{../figures/ece1254-multiphysics/lecture1Fig1}{A static loaded truss configuration.}{fig:lecture1:lecture1Fig1}{0.3}

\imageFigure{../figures/ece1254-multiphysics/lecture1Fig2}{Simple static load.}{fig:lecture1:lecture1Fig2}{0.2}

The unknowns are

\begin{itemize}
   \item positions of the joints after deformation \((x_i, y_i)\).
   \item force acting on each strut \(\BF_j = (F_{j,x}, F_{j,y})\).
\end{itemize}

The \textAndIndex{constitutive equations}, assuming static conditions (steady state, no transients)

\begin{itemize}
   \item Load force.  \(\BF_{\textrm{L}} = (F_{\textrm{L}, x}, F_{\textrm{L}, y}) = (0, -m g)\).
   \item Strut forces.  Under static conditions the total resulting force on the strut is zero, so \(\BF'_j = -\BF_j\).  For this problem it is redundant to label forces on both ends, so the labeled end of the object is marked with an asterisk as in \cref{fig:lecture1:lecture1Fig3}.
\end{itemize}

\imageFigure{../figures/ece1254-multiphysics/lecture1Fig3}{Strut model.}{fig:lecture1:lecture1Fig3}{0.2}

\paragraph{Consider a simple case}

One strut as in \cref{fig:lecture1:lecture1Fig4}.

\imageFigure{../figures/ece1254-multiphysics/lecture1Fig4}{Very simple static load.}{fig:lecture1:lecture1Fig4}{0.2}

\begin{equation}\label{eqn:multiphysicsL1:20}
\BF^\conj = - \Ba_x
\mathLabelBox
[
   labelstyle={yshift=1.2em},
   linestyle={}
]
{
\epsilon
}
{constant, describes the beam elasticity, given}
\biglr{
\mathLabelBox
[
   labelstyle={below of=m\themathLableNode, below of=m\themathLableNode}
]
{
L
}
{unloaded length \(L = \Abs{x^\conj - 0}\), given}
- L_0}
\end{equation}

The constitutive law for a general strut as in \cref{fig:lecture1:lecture1Fig5} is

\imageFigure{../figures/ece1254-multiphysics/lecture1Fig5}{Strut force diagram.}{fig:lecture1:lecture1Fig5}{0.2}

The force is directed along the unit vector

\begin{equation}\label{eqn:multiphysicsL1:40}
\Be = \frac{\Br^\conj - \Br}{\Abs{\Br^\conj - \Br}},
\end{equation}

and has the form
\begin{equation}\label{eqn:multiphysicsL1:60}
   \BF^\conj = - \Be \epsilon \lr{ L - L_0 }.
\end{equation}

The value \(\epsilon\) may be related to Hooks' constant, and \(L_0\) is given by

\begin{equation}\label{eqn:multiphysicsL1:80}
   L = \Abs{\Br^\conj - \Br} = \sqrt{(x^\conj - x)^2 + (y^\conj - y)^2}.
\end{equation}

Observe that the relation between \(\BF^\conj\) and position is {\em nonlinear}!

Treatment of this system will be used as the prototype for the handling of other \textAndIndex{nonlinear systems}.

Returning to the simple static system, and introducing force and joint labels as in \cref{fig:lecture1:lecture1Fig6}, the \textAndIndex{conservation law}, a balance of forces, can be examined.

\imageFigure{../figures/ece1254-multiphysics/lecture1Fig6}{Strut system.}{fig:lecture1:lecture1Fig6}{0.3}

\begin{itemize}
\item
At joint 1:

\begin{equation}\label{eqn:multiphysicsL1:100}
\Bf_{\textrm{A}} + \Bf_{\textrm{B}} + \Bf_{\textrm{C}} = 0
\end{equation}

or
\begin{equation}\label{eqn:multiphysicsL1:120}
\begin{aligned}
   \Bf_{\textrm{A},x} + \Bf_{\textrm{B},x} + \Bf_{\textrm{C},x} &= 0 \\
   \Bf_{\textrm{A},y} + \Bf_{\textrm{B},y} + \Bf_{\textrm{C},y} &= 0
\end{aligned}
\end{equation}
\item
At joint 2:

\begin{equation}\label{eqn:multiphysicsL1:140}
-\Bf_{\textrm{C}} + \Bf_{\textrm{D}} + \Bf_{\textrm{L}} = 0
\end{equation}

or
\begin{equation}\label{eqn:multiphysicsL1:160}
\begin{aligned}
   -\Bf_{\textrm{C},x} + \Bf_{\textrm{D},x} + \Bf_{\textrm{L},x} &= 0 \\
   -\Bf_{\textrm{C},y} + \Bf_{\textrm{D},y} + \Bf_{\textrm{L},y} &= 0
\end{aligned}
\end{equation}

\end{itemize}

There are two equivalences

\begin{itemize}
\item
Force \(\leftrightarrow\) Current.
\item
Force balance equation \(\leftrightarrow\) KCL
\end{itemize}

%\EndArticle
%\EndNoBibArticle
