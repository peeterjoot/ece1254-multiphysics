%
% Copyright � 2014 Peeter Joot.  All Rights Reserved.
% Licenced as described in the file LICENSE under the root directory of this GIT repository.
%
%\input{../blogpost.tex}
%\renewcommand{\basename}{laplaceTransformVec}
%\renewcommand{\dirname}{notes/ece1254/}
%\newcommand{\dateintitle}{}
%\newcommand{\keywords}{}
%
%\input{../peeter_prologue_print2.tex}
%
%\beginArtNoToc
%
%\generatetitle{Laplace transform refresher}
%
Laplace transforms \index{Laplace transform} were used to solve the MNA equations for time dependent systems, and to find the moments used to in MOR.
\index{model order reduction}
%
For the record, the Laplace transform is defined as:
%
%\begin{equation}\label{eqn:laplaceTransformVec:20}
\boxedEquation{eqn:laplaceTransformVec:20}{
\LL( f(t) ) =
\int_0^\infty e^{-s t} f(t) dt.
}
%\end{equation}
%
The only Laplace transform pair used in the lectures is that of the first derivative
%
\begin{dmath}\label{eqn:laplaceTransformVec:40}
\LL(f'(t)) =
\int_0^\infty e^{-s t} \ddt{f(t)} dt
=
\evalrange{e^{-s t} f(t)}{0}{\infty} - (-s) \int_0^\infty e^{-s t} f(t) dt
=
-f(0) + s \LL(f(t)).
\end{dmath}
%
Here it is loosely assumed that the real part of \( s \) is positive, and that \( f(t) \) is ``well defined'' enough that \( e^{-s \infty } f(\infty) \rightarrow 0 \).
%
Where used in the lectures, the Laplace transforms were of vectors such as the matrix vector product \( \LL(\BG \Bx(t)) \).  Because such a product is linear, observe that it can be expressed as the original matrix times a vector of Laplace transforms
%
\begin{dmath}\label{eqn:laplaceTransformVec:60}
\LL( \BG \Bx(t) )
=
\LL {\begin{bmatrix}
G_{i k} x_k(t)
\end{bmatrix}}_i
=
{\begin{bmatrix}
G_{i k} \LL x_k(t)
\end{bmatrix}}_i
=
\BG
{\begin{bmatrix}
\LL x_i(t)
\end{bmatrix}}_i.
\end{dmath}
%
The following notation was used in the lectures for such a vector of Laplace transforms
%
\begin{equation}\label{eqn:laplaceTransformVec:80}
\BX(s) = \LL \Bx(t) =
{\begin{bmatrix}
\LL x_i(t)
\end{bmatrix}}_i.
\end{equation}
%
%\EndNoBibArticle
