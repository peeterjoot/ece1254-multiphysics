%
% Copyright � 2014 Peeter Joot.  All Rights Reserved.
% Licenced as described in the file LICENSE under the root directory of this GIT repository.
%
% for template copy, run:
%
% ~/bin/ct multiphysicsL1  multiphysicsLectureN tl1
%
%\input{../blogpost.tex}
%\renewcommand{\basename}{multiphysicsL15}
%\renewcommand{\dirname}{notes/ece1254/}
%\newcommand{\keywords}{ECE1254H}
%\input{../peeter_prologue_print2.tex}
%
%\beginArtNoToc
%\generatetitle{ECE1254H Modeling of Multiphysics Systems.  Lecture 15: Nonlinear differential equations.  Taught by Prof.\ Piero Triverio}
%\chapter{Nonlinear differential equations}
%\label{chap:multiphysicsL15}
%
%\section{Disclaimer}
%
%Peeter's lecture notes from class.  These may be incoherent and rough.
%
\section{Nonlinear differential equations}
\index{nonlinear differential equations}
%
Assume that the relationships between the zeroth and first order derivatives has the form
%
\begin{subequations}
\begin{equation}\label{eqn:multiphysicsL15:20}
F\lr{ x(t), \dot{x}(t) } = 0
\end{equation}
\begin{equation}\label{eqn:multiphysicsL15:40}
x(0) = x_0
\end{equation}
\end{subequations}
%
The backward Euler method where the derivative approximation is
%
\begin{equation}\label{eqn:multiphysicsL15:60}
\dot{x}(t_n) \approx \frac{x_n - x_{n-1}}{\Delta t},
\end{equation}
%
can be used to solve this numerically, reducing the problem to
%
\begin{equation}\label{eqn:multiphysicsL15:80}
F\lr{ x_n, \frac{x_n - x_{n-1}}{\Delta t} } = 0.
\end{equation}
%
This can be solved with Newton's method.  How do we find the initial guess for Newton's?  Consider a possible system in \cref{fig:lecture15:lecture15Fig1}.
%
\imageFigure{../figures/ece1254-multiphysics/lecture15Fig1}{Possible solution points.}{fig:lecture15:lecture15Fig1}{0.2}
%
One strategy for starting each iteration of Newton's method is to base the initial guess for \( x_1 \) on the value \( x_0 \), and do so iteratively for each subsequent point.  One can imagine that this may work up to some sample point \( x_n \), but then break down (i.e. Newton's diverges when the previous value \( x_{n-1} \) is used to attempt to solve for \( x_n \)).  At that point other possible strategies may work.  One such strategy is to use an approximation of the derivative from the previous steps to attempt to get a better estimate of the next value.  Another possibility is to reduce the time step, so the difference between successive points is reduced.
%
\section{Analysis, accuracy and stability (\(\Delta t \rightarrow 0\))}
\index{accuracy}
\index{stability}
%
Consider a differential equation
%
\begin{subequations}
\begin{equation}\label{eqn:multiphysicsL15:100}
\dot{x}(t) = f(x(t), t)
\end{equation}
\begin{equation}\label{eqn:multiphysicsL15:120}
x(t_0) = x_0
\end{equation}
\end{subequations}
%
A few methods of solution have been considered
%
\begin{itemize}
\item [(FE)] \( x_{n+1} - x_n = \Delta t f(x_n, t_n) \)
\item [(BE)] \( x_{n+1} - x_n = \Delta t f(x_{n+1}, t_{n+1}) \)
\item [(TR)] \( x_{n+1} - x_n = \frac{\Delta t}{2} f(x_{n+1}, t_{n+1}) + \frac{\Delta t}{2} f(x_{n}, t_{n}) \)
\end{itemize}
%
A common pattern can be observed, the generalization of which are called
\textit{linear multistep methods}
\index{linear multistep methods}
(LMS) \index{LMS|see {linear multistep methods}}, which have the form
%
\begin{equation}\label{eqn:multiphysicsL15:140}
\sum_{j=-1}^{k-1} \alpha_j x_{n-j} = \Delta t \sum_{j=-1}^{k-1} \beta_j f( x_{n-j}, t_{n-j} )
\end{equation}
%
The FE (explicit), BE (implicit), and TR methods are now special cases with
%
\begin{itemize}
\item [(FE)] \( \alpha_{-1} = 1, \alpha_0 = -1, \beta_{-1} = 0, \beta_0 = 1 \)
\item [(BE)] \( \alpha_{-1} = 1, \alpha_0 = -1, \beta_{-1} = 1, \beta_0 = 0 \)
\item [(TR)] \( \alpha_{-1} = 1, \alpha_0 = -1, \beta_{-1} = 1/2, \beta_0 = 1/2 \)
\end{itemize}
%
Here \( k \) is the number of timesteps used.  The method is explicit if \( \beta_{-1} = 0 \).
%
\index{convergence}
\makedefinition{Convergence.}{dfn:multiphysicsL15:160}{
With
%
\begin{itemize}
\item [\(x(t)\)] : exact solution
\item [\(x_n\)] : computed solution
\item [\(e_n\)] : where \( e_n = x_n - x(t_n) \), is the \textAndIndex{global error}
\end{itemize}
%
The LMS method is convergent if
%
\begin{equation*}%\label{eqn:multiphysicsL15:180}
\max_{n, \Delta t \rightarrow 0} \Abs{ x_n - t(t_n) } \rightarrow 0 %\xrightarrow[t \rightarrow 0 ]{} 0
\end{equation*}
}
%
Convergence: zero-stability and \textAndIndex{consistency} (small local errors made at each iteration),
%
where \textAndIndex{zero-stability} is ``small sensitivity to changes in initial condition''.
%
\makedefinition{Consistency.}{dfn:multiphysicsL15:200}{
%
A local error \( R_{n+1} \) can be defined as
%
\begin{equation*}%\label{eqn:multiphysicsL15:220}
R_{n+1} = \sum_{j = -1}^{k-1} \alpha_j x(t_{n-j}) - \Delta t \sum_{j=-1}^{k-1} \beta_j f(x(t_{n-j}), t_{n-j}).
\end{equation*}
%
The method is consistent if
%
\begin{equation*}%\label{eqn:multiphysicsL15:240}
\lim_{\Delta t} \lr{
   \max_n \Abs{ \inv{\Delta t} R_{n+1} } = 0
}
\end{equation*}
%
or \( R_{n+1} \sim O({\Delta t}^2) \)
}
%
%\EndNoBibArticle
