%
% Copyright � 2014 Peeter Joot.  All Rights Reserved.
% Licenced as described in the file LICENSE under the root directory of this GIT repository.
%
%\input{../blogpost.tex}
%\renewcommand{\basename}{multiphysicsL4}
%\renewcommand{\dirname}{notes/ece1254/}
%\newcommand{\keywords}{ECE1254H}
%\input{../peeter_prologue_print2.tex}
%
%\usepackage{kbordermatrix}
%
%\beginArtNoToc
%\generatetitle{ECE1254H Modeling of Multiphysics Systems.  Lecture 4: Modified nodal analysis.  Taught by Prof.\ Piero Triverio}
%%\subsection{Modified nodal analysis}
\index{modified nodal analysis}
%\label{chap:multiphysicsL4}
%
%\subsection{Disclaimer}
%
%Peeter's lecture notes from class.  These may be incoherent and rough.
%\chapter{Solving large systems}

The goal is to solve \textAndIndex{linear systems} of the form

\begin{equation}\label{eqn:multiphysicsL4:60}
   M \Bx = \Bb,
\end{equation}

possibly with thousands of elements.

\sectionAndIndex{Gaussian elimination}

\begin{equation}\label{eqn:multiphysicsL4:80}
   \kbordermatrix{
        & 1      & 2      & 3      \\
      1 & M_{11} & M_{12} & M_{13} \\
      2 & M_{21} & M_{22} & M_{23} \\
      3 & M_{31} & M_{32} & M_{33} \\
   }
\begin{bmatrix}
   x_1 \\
   x_2 \\
   x_3 \\
\end{bmatrix}
=
\begin{bmatrix}
   b_1 \\
   b_2 \\
   b_3 \\
\end{bmatrix}
\end{equation}

It's claimed for now, to be seen later, that back substitution is the fastest way to arrive at the solution, less computationally complex than completion the diagonalization.

Steps

\begin{equation}\label{eqn:multiphysicsL4:100}
   (1) \cdot \frac{M_{21}}{M_{11}} \implies
\begin{bmatrix}
      M_{21} &  \frac{M_{21}}{M_{11}}  M_{12} &  \frac{M_{21}}{M_{11}}  M_{13} \\
\end{bmatrix}
\end{equation}

\begin{equation}\label{eqn:multiphysicsL4:120}
   (2) \cdot \frac{M_{31}}{M_{11}} \implies
\begin{bmatrix}
      M_{31} &  \frac{M_{31}}{M_{11}}  M_{32} &  \frac{M_{31}}{M_{11}}  M_{33} \\
\end{bmatrix}
\end{equation}

This gives

\begin{equation}\label{eqn:multiphysicsL4:140}
\begin{bmatrix}
%\mathLabelBox
%{
      M_{11}
%}
%{Pivot}
      & M_{12} & M_{13} \\
   0 &  M_{22} - \frac{M_{21}}{
%\mathLabelBox
%{
   M_{11}
%}{Multiplier}
}  M_{12} & M_{23} - \frac{M_{21}}{M_{11}}  M_{13} \\
   0 &  M_{32} - \frac{M_{31}}{M_{11}}  M_{32} & M_{33} - \frac{M_{31}}{M_{11}}  M_{33} \\
\end{bmatrix}
\begin{bmatrix}
   x_1 \\
   x_2 \\
   x_3 \\
\end{bmatrix}
=
\begin{bmatrix}
   b_1 \\
   b_2 - \frac{M_{21}}{M_{11}} b_1 \\
   b_3 - \frac{M_{31}}{M_{11}} b_1
\end{bmatrix}.
\end{equation}

Here the \(M_{11}\) element is called the \textAndIndex{pivot}.  Each of the \(M_{j1}/M_{11}\) elements is called a \textAndIndex{multiplier}.  This operation can be written as

\begin{equation}\label{eqn:multiphysicsL4:160}
\begin{bmatrix}
      M_{11} & M_{12} & M_{13} \\
   0 &  M_{22}^{(2)} & M_{23}^{(3)} \\
   0 &  M_{32}^{(2)} & M_{33}^{(3)} \\
\end{bmatrix}
\begin{bmatrix}
   x_1 \\
   x_2 \\
   x_3 \\
\end{bmatrix}
=
\begin{bmatrix}
   b_1 \\
   b_2^{(2)} \\
   b_3^{(2)} \\
\end{bmatrix}.
\end{equation}

Using \( M_{22}^{(2)} \) as the pivot this time, form
\begin{equation}\label{eqn:multiphysicsL4:180}
\begin{bmatrix}
      M_{11} & M_{12} & M_{13} \\
   0 &  M_{22}^{(2)} & M_{23}^{(3)} \\
   0 &             0 & M_{33}^{(3)} - \frac{M_{32}^{(2)}}{M_{22}^{(2)}} M_{23}^{(2)} \\
\end{bmatrix}
\begin{bmatrix}
   x_1 \\
   x_2 \\
   x_3 \\
\end{bmatrix}
=
\begin{bmatrix}
   b_1 \\
   b_2 - \frac{M_{21}}{M_{11}} b_1 \\
   b_3 - \frac{M_{31}}{M_{11}} b_1
   - \frac{M_{32}^{(2)}}{M_{22}^{(2)}} b_{2}^{(2)} \\
\end{bmatrix}.
\end{equation}

\section{LU decomposition}
\index{LU decomposition}

Through Gaussian elimination, the system has been transformed from

\begin{equation}\label{eqn:multiphysicsL4:200}
   M x = b
\end{equation}

to
\begin{equation}\label{eqn:multiphysicsL4:220}
   U x = y.
\end{equation}

The Gaussian transformation written out in the form \( U \Bx = b \) is

\begin{equation}\label{eqn:multiphysicsL4:240}
U \Bx =
\begin{bmatrix}
   1 & 0 & 0 \\
   -\frac{M_{21}}{M_{11}} & 1 & 0 \\
   \frac{M_{32}^{(2)}}{M_{22}^{(2)}}
   \frac{M_{21}}{M_{11}}
-
\frac{M_{31}}{M_{11}}
&
-
\frac{M_{32}^{(2)}}{M_{22}^{(2)}}
& 1
\end{bmatrix}
\begin{bmatrix}
   b_1 \\
   b_2 \\
   b_3
\end{bmatrix}.
\end{equation}

As a verification observe that the
operation matrix \( K^{-1} \), where \( K^{-1} U = M \) produces the original system

\begin{equation}\label{eqn:multiphysicsL4:260}
\begin{bmatrix}
   1 & 0 & 0 \\
   \frac{M_{21}}{M_{11}} & 1 & 0 \\
   \frac{M_{31}}{M_{11}} &
   \frac{M_{32}^{(2)}}{M_{22}^{(2)}} &
   1
\end{bmatrix}
\begin{bmatrix}
   U_{11} & U_{12} & U_{13} \\
   0      & U_{22} & U_{23} \\
   0      & 0      & U_{33} \\
\end{bmatrix}
\Bx = \Bb
\end{equation}

\index{LU decomposition}
Using this LU decomposition is generally superior to standard Gaussian elimination, since it can be used for many different \(\Bb\) vectors, and cost no additional work after the initial factorization.

The steps are

\begin{dmath}\label{eqn:multiphysicsL4:280}
b
= M x
= L \lr{ U x}
\equiv L y.
\end{dmath}

The matrix equation \( L y = b \) can now be solved by substituting first  \( y_1 \), then \( y_2 \), and finally \( y_3 \).  This is called \textAndIndex{forward substitution}.

The final solution is

\begin{dmath}\label{eqn:multiphysicsL4:300}
   U x = y,
\end{dmath}

using \textAndIndex{back substitution}.

Note that this produced the vector \( y \) as a side effect of performing the Gaussian elimination process.

%\index{LU decomposition}
\makeexample{Numeric LU factorization.}{example:multiphysicsL4:1}{
%
% Copyright � 2014 Peeter Joot.  All Rights Reserved.
% Licenced as described in the file LICENSE under the root directory of this GIT repository.
%
%\input{../blogpost.tex}
%\renewcommand{\basename}{luExample}
%\renewcommand{\dirname}{notes/ece1254/}
%%\newcommand{\dateintitle}{}
%%\newcommand{\keywords}{}
%
%\input{../peeter_prologue_print2.tex}
%
%\usepackage{kbordermatrix}
%
%\beginArtNoToc
%
%\generatetitle{Numeric LU factorization example}
%\chapter{Numeric LU factorization example}
%\label{chap:luExample}
%
Looking at a numeric example is helpful to get a better feel for LU factorization before attempting a Matlab implementation, as it strips some of the abstraction away.
%
\begin{equation}\label{eqn:luExample:20}
M =
\begin{bmatrix}
5 & 1 & 1 \\
2 & 3 & 4 \\
3 & 1 & 2 \\
\end{bmatrix}.
\end{equation}
%
This matrix was picked to avoid having to think of selecting the right pivot row when performing the \( L U \) factorization.  The first two operations give
%
\begin{equation}\label{eqn:luExample:40}
\kbordermatrix{
&  &   & \\
&5 & 1 & 1 \\
\lr{ r_2 \rightarrow r_2 - \frac{2}{5} r_1 } & 0 &13/5  & 18/5 \\
\lr{ r_3 \rightarrow r_3 - \frac{3}{5} r_1 } & 0 &2/5 & 7/5 \\
}.
\end{equation}
%
The row operations (left multiplication) that produce this matrix are
%
\begin{equation}\label{eqn:luExample:60}
\begin{bmatrix}
1 & 0 & 0 \\
0 & 1 & 0 \\
-3/5 & 0 & 1 \\
\end{bmatrix}
\begin{bmatrix}
1 & 0 & 0 \\
-2/5 & 1 & 0 \\
0 & 0 & 1 \\
\end{bmatrix}
=
\begin{bmatrix}
1 & 0 & 0 \\
-2/5 & 1 & 0 \\
-3/5 & 0 & 1 \\
\end{bmatrix}.
\end{equation}
%
These operations happen to be commutative and also both invert simply.  The inverse operations are
\begin{equation}\label{eqn:luExample:80}
\begin{bmatrix}
1 & 0 & 0 \\
2/5 & 1 & 0 \\
0 & 0 & 1 \\
\end{bmatrix}
\begin{bmatrix}
1 & 0 & 0 \\
0 & 1 & 0 \\
3/5 & 0 & 1 \\
\end{bmatrix}
=
\begin{bmatrix}
1 & 0 & 0 \\
2/5 & 1 & 0 \\
3/5 & 0 & 1 \\
\end{bmatrix}.
\end{equation}
%
In matrix form the elementary matrix operations that produce the first stage of the Gaussian reduction are
%
\begin{equation}\label{eqn:luExample:100}
\begin{bmatrix}
1 & 0 & 0 \\
-2/5 & 1 & 0 \\
-3/5 & 0 & 1 \\
\end{bmatrix}
\begin{bmatrix}
5 & 1 & 1 \\
2 & 3 & 4 \\
3 & 1 & 2 \\
\end{bmatrix}
=
\begin{bmatrix}
5 & 1 & 1 \\
0 &13/5  & 18/5 \\
0 &2/5 & 7/5 \\
\end{bmatrix}.
\end{equation}
%
Inverted that is
%
\begin{equation}\label{eqn:luExample:120}
\begin{bmatrix}
5 & 1 & 1 \\
2 & 3 & 4 \\
3 & 1 & 2 \\
\end{bmatrix}
=
\begin{bmatrix}
1 & 0 & 0 \\
2/5 & 1 & 0 \\
3/5 & 0 & 1 \\
\end{bmatrix}
\begin{bmatrix}
5 & 1 & 1 \\
0 &13/5  & 18/5 \\
0 &2/5 & 7/5 \\
\end{bmatrix}.
\end{equation}
%
This is the first stage of the LU decomposition, although the U matrix is not yet in upper triangular form.  With the pivot row in the desired position already, the last row operation to perform is
%
\begin{equation}\label{eqn:luExample:140}
r_3 \rightarrow r_3 - \frac{2/5}{5/13} r_2 = r_3 - \frac{2}{13} r_2.
\end{equation}
%
The final stage of this Gaussian reduction is
%
\begin{equation}\label{eqn:luExample:160}
\begin{bmatrix}
1 & 0 & 0 \\
0 & 1 & 0 \\
0 & -2/13 & 1 \\
\end{bmatrix}
\begin{bmatrix}
5 & 1 & 1 \\
0 &13/5  & 18/5 \\
0 &2/5 & 7/5 \\
\end{bmatrix}
=
\begin{bmatrix}
5 & 1 & 1 \\
0 &13/5  & 18/5 \\
0 & 0 & 11/13 \\
\end{bmatrix}
= U,
\end{equation}
%
so the desired lower triangular matrix factor is
\begin{equation}\label{eqn:luExample:180}
\begin{bmatrix}
1 & 0 & 0 \\
2/5 & 1 & 0 \\
3/5 & 0 & 1 \\
\end{bmatrix}
\begin{bmatrix}
1 & 0 & 0 \\
0 & 1 & 0 \\
0 & 2/13 & 1 \\
\end{bmatrix}
=
\begin{bmatrix}
1 & 0 & 0 \\
2/5 & 1 & 0 \\
3/5 & 2/13 & 1 \\
\end{bmatrix}
= L.
\end{equation}
%
A bit of Matlab code easily verifies that the above manual computation recovers \( M = L U \)
%
\begin{verbatim}
l = [ 1 0 0 ; 2/5 1 0 ; 3/5 2/13 1 ] ;
u = [ 5 1 1 ; 0 13/5 18/5 ; 0 0 11/13 ] ;
m = l * u
\end{verbatim}
%
%\EndArticle
%\EndNoBibArticle

}

\index{LU decomposition!pivot}
\makeexample{Numeric LU factorization with pivots.}{example:multiphysicsL4:2}{
%
% Copyright � 2014 Peeter Joot.  All Rights Reserved.
% Licenced as described in the file LICENSE under the root directory of this GIT repository.
%
%\input{../blogpost.tex}
%\renewcommand{\basename}{luExample2}
%\renewcommand{\dirname}{notes/ece1254/}
%%\newcommand{\dateintitle}{}
%%\newcommand{\keywords}{}
%
%\input{../peeter_prologue_print2.tex}
%
%\beginArtNoToc
%
%\generatetitle{Numerical LU example where pivoting is required}
%\chapter{Numerical LU example where pivoting is required}
%\label{chap:luExample2}
%
%Given some trouble understanding how to apply the LU procedure where pivoting is required.
Proceeding with a factorization where pivots are required, does not produce an LU factorization that is the product of a lower triangular matrix and an upper triangular matrix.  Instead what is found is what looks like a permutation of a lower triangular matrix with an upper triangular matrix.  As an example, consider the LU reduction of
%
\begin{equation}\label{eqn:luExample2:20}
M \Bx =
\begin{bmatrix}
0 & 0 & 1 \\
2 & 0 & 4 \\
1 & 1 & 1
\end{bmatrix}
\begin{bmatrix}
x_1 \\
x_2 \\
x_3 \\
\end{bmatrix}.
\end{equation}
%
Since \( r_2 \) has the biggest first column value, that is the row selected as the pivot
%
\begin{equation}\label{eqn:luExample2:40}
M \Bx \rightarrow M' \Bx'
=
\begin{bmatrix}
2 & 0 & 4 \\
0 & 0 & 1 \\
1 & 1 & 1
\end{bmatrix}
\begin{bmatrix}
x_2 \\
x_1 \\
x_3 \\
\end{bmatrix}.
\end{equation}
%
This permutation can be expressed algebraically as a row permutation matrix operation
%
\begin{equation}\label{eqn:luExample2:60}
M' =
\begin{bmatrix}
0 & 1 & 0 \\
1 & 0 & 0 \\
0 & 0 & 1
\end{bmatrix}
M.
\end{equation}
%
With the pivot permutations out of the way, the row operations remaining for the Gaussian reduction of this column are
%
\begin{equation}\label{eqn:luExample2:80}
\begin{aligned}
r_2 & \rightarrow r_2 - \frac{0}{2} r_1 \\
r_3 & \rightarrow r_3 - \frac{1}{2} r_1 \\
\end{aligned},
\end{equation}
%
which gives
%
\begin{dmath}\label{eqn:luExample2:100}
M_1 =
\begin{bmatrix}
1 & 0 & 0 \\
0 & 1 & 0 \\
-1/2 & 0 & 1
\end{bmatrix}
M'
=
\begin{bmatrix}
1 & 0 & 0 \\
0 & 1 & 0 \\
-1/2 & 0 & 1
\end{bmatrix}
\begin{bmatrix}
2 & 0 & 4 \\
0 & 0 & 1 \\
1 & 1 & 1
\end{bmatrix}
=
\begin{bmatrix}
2 & 0 & 4 \\
0 & 0 & 1 \\
0 & 1 & -1
\end{bmatrix}.
\end{dmath}
%
Application of one more permutation operation gives the desired upper triangular matrix
%
\begin{dmath}\label{eqn:luExample2:120}
U = M_1' =
\begin{bmatrix}
1 & 0 & 0 \\
0 & 0 & 1 \\
0 & 1 & 0 \\
\end{bmatrix}
\begin{bmatrix}
2 & 0 & 4 \\
0 & 0 & 1 \\
0 & 1 & -1
\end{bmatrix}
=
\begin{bmatrix}
2 & 0 & 4 \\
0 & 1 & -1 \\
0 & 0 & 1
\end{bmatrix}.
\end{dmath}
%
This new matrix operator applies to the permuted vector
%
\begin{equation}\label{eqn:luExample2:140}
\Bx'' =
\begin{bmatrix}
x_2 \\
x_3 \\
x_1 \\
\end{bmatrix}.
\end{equation}
%
The matrix \( U \) has been constructed by the following row operations
%
\begin{dmath}\label{eqn:luExample2:160}
U =
\begin{bmatrix}
1 & 0 & 0 \\
0 & 0 & 1 \\
0 & 1 & 0 \\
\end{bmatrix}
\begin{bmatrix}
1 & 0 & 0 \\
0 & 1 & 0 \\
-1/2 & 0 & 1
\end{bmatrix}
\begin{bmatrix}
0 & 1 & 0 \\
1 & 0 & 0 \\
0 & 0 & 1 \\
\end{bmatrix}
M.
\end{dmath}
%
\( L U = M \) is sought, or
%
\begin{dmath}\label{eqn:luExample2:180}
L
\begin{bmatrix}
1 & 0 & 0 \\
0 & 0 & 1 \\
0 & 1 & 0 \\
\end{bmatrix}
\begin{bmatrix}
1 & 0 & 0 \\
0 & 1 & 0 \\
-1/2 & 0 & 1
\end{bmatrix}
\begin{bmatrix}
0 & 1 & 0 \\
1 & 0 & 0 \\
0 & 0 & 1 \\
\end{bmatrix}
M
= M,
\end{dmath}
%
or
%
\begin{dmath}\label{eqn:luExample2:200}
L
=
\begin{bmatrix}
0 & 1 & 0 \\
1 & 0 & 0 \\
0 & 0 & 1 \\
\end{bmatrix}
\begin{bmatrix}
1 & 0 & 0 \\
0 & 1 & 0 \\
1/2 & 0 & 1
\end{bmatrix}
\begin{bmatrix}
1 & 0 & 0 \\
0 & 0 & 1 \\
0 & 1 & 0 \\
\end{bmatrix}
=
\begin{bmatrix}
0 & 1 & 0 \\
1 & 0 & 0 \\
1/2 & 0 & 1
\end{bmatrix}
\begin{bmatrix}
1 & 0 & 0 \\
0 & 0 & 1 \\
0 & 1 & 0 \\
\end{bmatrix}
=
\begin{bmatrix}
0 & 0 & 1 \\
1 & 0 & 0 \\
1/2 & 1 & 0
\end{bmatrix}.
\end{dmath}
%
The \( L U \) factorization attempted does not appear to produce a lower triangular factor, but a permutation of a lower triangular factor?
%
When such pivoting is required it isn't obvious, at least to me, how to do the clever \( L U \) algorithm that outlined in class.  How can the operations be packed into the lower triangle when there is a requirement to actually have to apply permutation matrices to the results of the last iteration?
%
It seems that a \( LU \) decomposition of \( M \) cannot be performed, but an \( L U \) factorization of \( P M \), where \( P \) is the permutation matrix for the permutation \( 2,3,1 \) that was applied to the rows during the Gaussian operations.
%
Checking that \( L U \) factorization:
%
\begin{equation}\label{eqn:luExample2:220}
P M =
\begin{bmatrix}
0 & 1 & 0 \\
0 & 0 & 1 \\
1 & 0 & 0 \\
\end{bmatrix}
\begin{bmatrix}
0 & 0 & 1 \\
2 & 0 & 4 \\
1 & 1 & 1
\end{bmatrix}
=
\begin{bmatrix}
2 & 0 & 4 \\
1 & 1 & 1 \\
0 & 0 & 1 \\
\end{bmatrix}.
\end{equation}
%
The elementary row operation to be applied is
%
\begin{equation}\label{eqn:luExample2:240}
r_2 \rightarrow r_2 - \frac{1}{2} r_1,
\end{equation}
%
for
%
\begin{equation}\label{eqn:luExample2:260}
\lr{P M}_1 =
\begin{bmatrix}
2 & 0 & 4 \\
0 & 1 & -1 \\
0 & 0 & 1 \\
\end{bmatrix},
\end{equation}
%
The \( L U \) factorization is therefore
%
\begin{equation}\label{eqn:luExample2:280}
P M =
L U =
\begin{bmatrix}
1 & 0 & 0 \\
1/2 & 1 & 0 \\
0 & 0 & 1 \\
\end{bmatrix}
\begin{bmatrix}
2 & 0 & 4 \\
0 & 1 & -1 \\
0 & 0 & 1 \\
\end{bmatrix}.
\end{equation}
%
Observe that this can also be written as
%
\begin{equation}\label{eqn:luExample2:300}
M = \lr{ P^{-1} L } U.
\end{equation}
%
The inverse permutation is a \( 3,1,2 \) permutation matrix
%
\begin{equation}\label{eqn:luExample2:320}
P^{-1} =
\begin{bmatrix}
0 & 0 & 1 \\
1 & 0 & 0 \\
0 & 1 & 0 \\
\end{bmatrix}.
\end{equation}
%
It can be observed that the product \( P^{-1} L \) produces the not-lower-triangular matrix factor found earlier in \cref{eqn:luExample2:200}.
%
%\EndArticle
%\EndNoBibArticle

}

\makeexample{Final illustration of the LU algorithm with pivots by example.}{example:multiphysicsL4:3}{
%
% Copyright � 2014 Peeter Joot.  All Rights Reserved.
% Licenced as described in the file LICENSE under the root directory of this GIT repository.
%
%\input{../blogpost.tex}
%\renewcommand{\basename}{luAlgorithm}
%\renewcommand{\dirname}{notes/FIXMEwheretodirname/}
%%\newcommand{\dateintitle}{}
%%\newcommand{\keywords}{}
%
%\input{../peeter_prologue_print2.tex}
%
%\beginArtNoToc
%
%\generatetitle{Illustrating the LU algorithm with pivots by example}
%%\chapter{Illustrating the LU algorthm with pivots by example}
%%\label{chap:luAlgorithm}

Two previous examples of LU factorizations were given.  I found one more to be the key to understanding how to implement this as a Matlab algorithm, required for \cref{multiphysics:problemSet1:2}.

A matrix that contains both pivots and elementary matrix operations is

\begin{equation}\label{eqn:luAlgorithm:20}
M=
\begin{bmatrix}
0 & 0 & 2 & 1 \\
0 & 0 & 1 & 1 \\
2 & 0 & 2 & 0 \\
1 & 1 & 1 & 1
\end{bmatrix}
\end{equation}

The objective is to apply a sequence of row permutations or elementary row operations to \( M \) that put \( M \) into upper triangular form, while also tracking all the inverse operations.  When no permutations were required to produce \( U \), then a factorization \( M = L' U \) is produced where \( L' \) is lower triangular.

The row operations to be applied to \( M \) are

\begin{equation}\label{eqn:luAlgorithm:40}
U =
L_k^{-1}
L_{k-1}^{-1} \cdots
L_2^{-1}
L_1^{-1}
M,
\end{equation}

with

\begin{equation}\label{eqn:luAlgorithm:60}
L' = L_0 L_1 L_2 \cdots L_{k-1} L_k
\end{equation}

Here \( L_0 = I \), the identity matrix, and \( L_i^{-1} \) is either a permutation matrix interchanging two rows of the identity matrix, or it is an elementary row operation encoding the operation \( r_j \rightarrow r_j - M r_i \), where \( r_i \) is the pivot row, and \( r_j, j > i \) are the rows that the Gaussian elimination operations are applied to.

For this example matrix, the \( M_{1 1} \) value cannot be used as the pivot element since it is zero.  In general,
the row with the biggest absolute value in the column should be used.
In this case that is row 3.  The first row operation is therefore a \( 1,3 \) permutation.
For numeric stability, use

\begin{equation}\label{eqn:luAlgorithm:80}
L_1^{-1} =
\begin{bmatrix}
0 & 0 & 1 & 0 \\
0 & 1 & 0 & 0 \\
1 & 0 & 0 & 0 \\
0 & 0 & 0 & 1 \\
\end{bmatrix},
\end{equation}

which gives

\begin{equation}\label{eqn:luAlgorithm:100}
M \rightarrow
L_1^{-1}
M
=
\begin{bmatrix}
0 & 0 & 1 & 0 \\
0 & 1 & 0 & 0 \\
1 & 0 & 0 & 0 \\
0 & 0 & 0 & 1 \\
\end{bmatrix}
\begin{bmatrix}
0 & 0 & 2 & 1 \\
0 & 0 & 1 & 1 \\
2 & 0 & 2 & 0 \\
1 & 1 & 1 & 1 \\
\end{bmatrix}
=
\begin{bmatrix}
2 & 0 & 2 & 0 \\
0 & 0 & 1 & 1 \\
0 & 0 & 2 & 1 \\
1 & 1 & 1 & 1 \\
\end{bmatrix}.
\end{equation}

Computationally, avoiding the matrix multiplication that achieve this permutation is desired.  Instead just swap the two rows in question.

The inverse of this operation is the same permutation, so for \( L' \) the first stage computation is

\begin{equation}\label{eqn:luAlgorithm:120}
L \sim L_0 L_1 = L_1.
\end{equation}

As before, a matrix operation would be very expensive.  When the application of the permutation matrix is from the right, it results in an exchange of columns \(1,3\) of the \( L_0 \) matrix (which happens to be identity at this point).  So the matrix operation can be done as a column exchange directly using submatrix notation.

Now proceed down the column, doing all the non-zero row elimination operations required.  In this case, there is only one operation todo

\begin{equation}\label{eqn:luAlgorithm:140}
r_4 \rightarrow r_4 - \frac{1}{2} r_1.
\end{equation}

This has the matrix form

\begin{equation}\label{eqn:luAlgorithm:160}
L_2^{-1} =
\begin{bmatrix}
1 & 0 & 0 & 0 \\
0 & 1 & 0 & 0 \\
0 & 0 & 1 & 0 \\
-1/2 & 0 & 0 & 1 \\
\end{bmatrix}.
\end{equation}

The next stage of the \( U \) computation is

\begin{equation}\label{eqn:luAlgorithm:180}
M
\rightarrow L_2^{-1} L_1^{-1} M
=
\begin{bmatrix}
1 & 0 & 0 & 0 \\
0 & 1 & 0 & 0 \\
0 & 0 & 1 & 0 \\
-1/2 & 0 & 0 & 1 \\
\end{bmatrix}
\begin{bmatrix}
2 & 0 & 2 & 0 \\
0 & 0 & 1 & 1 \\
0 & 0 & 2 & 1 \\
1 & 1 & 1 & 1 \\
\end{bmatrix}
=
\begin{bmatrix}
2 & 0 & 2 & 0 \\
0 & 0 & 1 & 1 \\
0 & 0 & 2 & 1 \\
0 & 1 & 0 & 1 \\
\end{bmatrix}.
\end{equation}

Again, this is not an operation that should be done as a matrix operation.  Instead act directly on the rows in question with \cref{eqn:luAlgorithm:140}.

Note that the inverse of this matrix operation is very simple.
An amount \( r_1/2 \) has been subtracted from \( r_4 \), so to invert this all that is required is adding back \( r_1/2 \).  That is

\begin{equation}\label{eqn:luAlgorithm:200}
L_2
=
\begin{bmatrix}
1 & 0 & 0 & 0 \\
0 & 1 & 0 & 0 \\
0 & 0 & 1 & 0 \\
1/2 & 0 & 0 & 1 \\
\end{bmatrix}.
\end{equation}

Observe that when this is applied from the right to \( L_0 L_1 \rightarrow L_0 L_1 L_2\), the interpretation is a column operation

\begin{equation}\label{eqn:luAlgorithm:220}
c_1 \rightarrow c_1 + m c_4,
\end{equation}

In general, if application of the row operation

\begin{equation}\label{eqn:luAlgorithm:240}
r_j \rightarrow r_j - m r_i,
\end{equation}

to the current state of the matrix \( U \), requires application of the operation

\begin{equation}\label{eqn:luAlgorithm:260}
r_i \rightarrow r_i + m r_j,
\end{equation}

to the current state of the matrix \( L' \).

The next step is to move on to reduction of column 2, and for that only a permutation operation is required

\begin{equation}\label{eqn:luAlgorithm:280}
L_3 =
\begin{bmatrix}
1 & 0 & 0 & 0 \\
0 & 0 & 0 & 1 \\
0 & 0 & 1 & 0 \\
0 & 1 & 0 & 0 \\
\end{bmatrix},
\end{equation}

Application of a \( 2,4 \) row interchange to U, and a \( 2,4 \) column interchange to \( L' \) gives

\begin{equation}\label{eqn:luAlgorithm:300}
M \rightarrow
\begin{bmatrix}
2 & 0 & 2 & 0 \\
0 & 1 & 0 & 1 \\
0 & 0 & 2 & 1 \\
0 & 0 & 1 & 1 \\
\end{bmatrix}.
\end{equation}

The final operation is a regular row operation

\begin{equation}\label{eqn:luAlgorithm:320}
r_4 \rightarrow r_4 - \inv{2} r_3,
\end{equation}

with matrix
\begin{equation}\label{eqn:luAlgorithm:340}
L_4^{-1} =
\begin{bmatrix}
1 & 0 & 0 & 0 \\
0 & 1 & 0 & 0 \\
0 & 0 & 1 & 0 \\
0 & 0 & -1/2 & 1 \\
\end{bmatrix}
\end{equation}

The composite permutation performed so far is

\begin{equation}\label{eqn:luAlgorithm:360}
P = L_3 L_1 I.
\end{equation}

This should also be computed by performing row interchanges, not matrix multiplication.

To solve the system

\begin{equation}\label{eqn:luAlgorithm:380}
M \Bx = L' U \Bx = \Bb,
\end{equation}

solve the equivalent problem

\begin{equation}\label{eqn:luAlgorithm:400}
P L' U \Bx = P \Bb,
\end{equation}

To do this let \( \By = U \Bx \), for

\begin{equation}\label{eqn:luAlgorithm:420}
P L' \By = P \Bb.
\end{equation}

The matrix \( L = P L' \) is lower triangular, as \( P \) contained all the permutations that applied along the way (FIXME: this is a statement, not a proof, and not obvious).

The system

\begin{equation}\label{eqn:luAlgorithm:440}
L \By = P \Bb,
\end{equation}

can now be solved using forward substitution, after which the upper triangular system

\begin{equation}\label{eqn:luAlgorithm:460}
\By = U \Bx,
\end{equation}

can be solved using only back substitution.

%\EndArticle
%\EndNoBibArticle

}

%\EndArticle
%\EndNoBibArticle
