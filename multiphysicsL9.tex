%
% Copyright � 2014 Peeter Joot.  All Rights Reserved.
% Licenced as described in the file LICENSE under the root directory of this GIT repository.
%
% for template copy, run:
%
% ~/bin/ct multiphysicsL1  multiphysicsLectureN tl1
%
%\input{../blogpost.tex}
%\renewcommand{\basename}{multiphysicsL9}
%\renewcommand{\dirname}{notes/ece1254/}
%\newcommand{\keywords}{ECE1254H}
%\input{../peeter_prologue_print2.tex}
%
%\usepackage{kbordermatrix}
%\usepackage{easybmat}
%
%\beginArtNoToc
%\generatetitle{ECE1254H Modeling of Multiphysics Systems.  Lecture 9: Conjugate gradient methods.  Taught by Prof.\ Piero Triverio}
%\chapter{Conjugate gradient methods}
\label{chap:multiphysicsL9}

%\section{Disclaimer}
%
%Peeter's lecture notes from class.  These may be incoherent and rough.

\section{Conjugate gradient convergence}
\index{conjugate gradient!convergence}

For \( k \ll n \), convergence orders are given in \cref{tab:multiphysicsL9:20}.  A matrix norm, similar to \( \Norm{ \Bx } = \sqrt{ \Bx^\T \Bx } \), is defined

\captionedTable{Convergence.}{tab:multiphysicsL9:20}{
\begin{tabular}{|l|l|}
\hline
       & Full     \\ \hline
Direct & \(O(n^3)\)   \\ \hline
C.G.   & \(O(k n^2)\) \\ \hline
\end{tabular}
}

\begin{equation}\label{eqn:multiphysicsL9:40}
\Norm{\Bx}_M \equiv \sqrt{ \Bx^\T M \Bx }.
\end{equation}

Note that this norm is real valued for CG which only applies to positive definite matrices (or it will not converge), so this norm is real valued.

... lots on slides...

\begin{equation}\label{eqn:multiphysicsL9:60}
K(M) = \frac
{\sigma_{\mathrm{max}}}
{\sigma_{\mathrm{min}}}
\end{equation}

...

Some fast ways to estimate the conditioning number are required.

\sectionAndIndex{Gershgorin circle theorem}

\maketheorem{Gershgorin circle theorem.}{thm:multiphysicsL9:1}{
Given \( M \), for any eigenvalue of \( M \) there is an \( i \in [1, n] \) such that

\begin{equation*}
\Abs{ \lambda - M_{i i} } \le \sum_{j \ne i} \Abs{ M_{i j} }
\end{equation*}
}

Consider this in the complex plane for row \( i \)

\begin{equation}\label{eqn:multiphysicsL9:80}
\begin{bmatrix}
M_{i 1} &
M_{i 2} &  \cdots M_{i i} & \cdots M_{i n}
\end{bmatrix}
\end{equation}

This inequality covers a circular region in the complex plane as illustrated in \cref{fig:lecture9:lecture9Fig1} for a two eigenvalue system.

\imageFigure{../figures/ece1254-multiphysics/lecture9Fig1}{Gershgorin circles.}{fig:lecture9:lecture9Fig1}{0.3}

These are called Gershgorin circles.

\makeexample{Leaky bar.}{example:multiphysicsL9:1}{
For the leaky bar of \cref{fig:lecture9:lecture9Fig2}, the matrix is

%\imageFigure{../figures/ece1254-multiphysics/lecture9Fig2}{Leaky bar}{fig:lecture9:lecture9Fig2}{0.3}
\imageFigure{../figures/ece1254-multiphysics/leaky-bar.pdf}{Leaky bar.}{fig:lecture9:lecture9Fig2}{0.3}

\begin{equation}\label{eqn:multiphysicsL9:100}
M =
\begin{bmatrix}
2  & -1 &        &    &    \\
-1 & 3  & -1     &    &    \\
   & -1 & 3      & -1 &    \\
   &    &        & \ddots &    \\
   &    &        &    & -1     \\
   &    &        & -1 & 2    \\
\end{bmatrix}
\end{equation}

The Gershgorin circles are \cref{fig:lecture9:lecture9Fig3}.

\imageFigure{../figures/ece1254-multiphysics/lecture9Fig3}{Gershgorin circles for leaky bar.}{fig:lecture9:lecture9Fig3}{0.3}

This puts a bound on the eigenvalues

\begin{equation}\label{eqn:multiphysicsL9:120}
1 \le \lambda(M) \le 5,
\end{equation}

so that

\begin{equation}\label{eqn:multiphysicsL9:140}
K(M) =
\frac{\lambda_{\mathrm{max}}}
{\lambda_{\mathrm{max}}}
\le 5.
\end{equation}

On slides: example with smaller leakage to ground.
On slides: example with no leakage to ground.

These had, progressively larger and larger (possibly indefinite for the latter) conditioning number estimates.

The latter had the form of

\begin{equation}\label{eqn:multiphysicsL9:160}
M =
\begin{bmatrix}
     1  & -1  &  0  &  0 \\
    -1  &  2  & -1  &  0 \\
     0  & -1  &  2  &  1 \\
     0  &  0  & -1  &  1
\end{bmatrix}
\end{equation}

The exact eigenvalues for this system happens to be
\begin{equation}\label{eqn:multiphysicsL9:180}
\lambda \in
\{
 3.1069
 0.2833,
 1.3049 \pm 0.7545i
\}
\end{equation}

so the exact conditioning number is \( 3.1/0.28 \approx 11 \).

Compare this to the estimates, which are

\begin{equation}\label{eqn:multiphysicsL9:320}
\begin{aligned}
\Abs{ \lambda_1 - 1 } &\le 1 \\
\Abs{ \lambda_2 - 2 } &\le 2 \\
\Abs{ \lambda_3 - 2 } &\le 2 \\
\Abs{ \lambda_4 - 1 } &\le 1 \\
\end{aligned}
\end{equation}

These are two circles at \( z = 1 \) of radius 1, and two circles at \( z = 2 \) of radius 2, as plotted in \cref{fig:GershgorinL9:GershgorinL9Fig9}.

\imageFigure{../figures/ece1254-multiphysics/GershgorinL9Fig9}{Gershgorin circles for 4 eigenvalue system.}{fig:GershgorinL9:GershgorinL9Fig9}{0.3}
}

\section{Preconditioning}
\index{preconditioning}

Goal is to take

\begin{equation}\label{eqn:multiphysicsL9:280}
M \Bx = \Bb
\end{equation}

and introduce an easy to invert matrix \( P \) to change the problem to

\begin{equation}\label{eqn:multiphysicsL9:200}
P^{-1} M \Bx = P^{-1} \Bb.
\end{equation}

This system has the same solution, but allows for choosing \( P \) to maximize the convergence speed.

\section{Symmetric preconditioning}
\index{symmetric preconditioning}

Because the conjugate gradient methods requires a symmetric matrix, it is desirable to pick a preconditioning method that preserves the symmetric (and positive definite) nature of the matrix.  This is possible by splitting \( P \) into square root factors

\begin{equation}\label{eqn:multiphysicsL9:220}
P = P^{1/2} P^{1/2},
\end{equation}

and apply to \( M \Bx = \Bb \) as

\begin{equation}\label{eqn:multiphysicsL9:240}
P^{-1/2} M
\mathLabelBox{P^{-1/2} P^{1/2}}{\(= I\)} \Bx = P^{-1/2} \Bb.
\end{equation}

Now introduce a change of variables \( \By = P^{1/2} \Bx \), transforming the system to solve into

\begin{equation}\label{eqn:multiphysicsL9:260}
P^{-1/2} M P^{-1/2} \By = \Bb'.
\end{equation}

Some options
\begin{itemize}
\item Diagonal preconditioner: \( \BP = \text{diag}\{M\} \)
\item Incomplete LU or Cholesky factorization.  Cheap, approximate decomposition where a preconditioner \( M \simeq L U = P \) is picked.  An incomplete LU factorization would be easy to invert since lower or upper triangular matrices are easy to invert.  In Matlab the \( \text{ilu}() \) function can be used to do an incomplete LU factorization.
\item ... (many preconditioner are available).
\end{itemize}

For a symmetric positive definite matrix \( M \), an LU decomposition of the form \( M = L L^\T \), is called the \textAndIndex{Cholesky factorization}.

As an example consider the matrix

\begin{equation}\label{eqn:multiphysicsL9:340}
M = \left[
\begin{BMAT}(b){ccccc}{ccccc}
\begin{matrix} 2 & \\ & 2 \end{matrix} & & & \text{(small)} &\\
& \ddots & & & \\
& & \begin{matrix} 6 & \\ & 6 \end{matrix} & & \\
& & & \ddots & \\
& \text{(small)} & & & \begin{matrix} 50 & \\ & 50 \end{matrix} \\
\end{BMAT}
\right],
\end{equation}

for which a diagonal preconditioner can be used

\begin{equation}\label{eqn:multiphysicsL9:360}
P = \left[
\begin{BMAT}(b){ccccc}{ccccc}
\begin{matrix} 1/2 & \\ & 1/2 \end{matrix} & & & &\\
& \ddots & & & \\
& & \begin{matrix} 1/6 & \\ & 1/6 \end{matrix} & & \\
& & & \ddots & \\
& & & & \begin{matrix} 1/50 & \\ & 1/50 \end{matrix} \\
\end{BMAT}
\right].
\end{equation}

The preconditioned matrix will now have the form

\begin{equation}\label{eqn:multiphysicsL9:380}
M' = \left[
\begin{BMAT}(b){ccccc}{ccccc}
\begin{matrix} 1 & \\ & 1 \end{matrix} & & & \text{(small)} &\\
& \ddots & & & \\
& & \begin{matrix} 1 & \\ & 1 \end{matrix} & & \\
& & & \ddots & \\
& \text{(small)} & & & \begin{matrix} 1 & \\ & 1 \end{matrix} \\
\end{BMAT}
\right],
\end{equation}

so that the Gershgorin circles can all be found within a small radius of unity as sketched in \cref{fig:lecture9:lecture9Fig8}.

\imageFigure{../figures/ece1254-multiphysics/lecture9Fig8}{Gershgorin circles after preconditioning.}{fig:lecture9:lecture9Fig8}{0.3}

\section{Preconditioned conjugate gradient}
\index{conjugate gradient!preconditioned}

It is possible to avoid inverting the preconditioner by requiring that the LU decomposition of \( P \) be easily computed.  Then

\begin{equation}\label{eqn:multiphysicsL9:300}
P \Bz^k = \Br^k
\end{equation}

can be solved by successive forward and back substitution.

More on slides...

%\EndArticle
%\EndNoBibArticle
